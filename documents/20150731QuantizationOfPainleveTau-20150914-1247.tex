%%%%%%%%%%%%%%%%%%%%%%%%%%%%%%%%%%%%%%%%%%%%%%%%%%%%%%%%%%%%%%%%%%%%%%%%%%%%
\def\TITLE{\bf Painlev\'e系 の $\tau$ 函数の正準量子化について}
\def\PDFTITLE{Painlev\'e系 の $\tau$ 函数の正準量子化について}
\def\PDFAUTHOR{黒木玄}
\def\PDFSUBJECT{数学}
\def\AUTHOR{黒木 玄}
\def\AUTHOREN{Gen Kuroki}
\def\DATE{2015年9月14日 Version 1.4}
\def\ABSTRACT{}
%%%%%%%%%%%%%%%%%%%%%%%%%%%%%%%%%%%%%%%%%%%%%%%%%%%%%%%%%%%%%%%%%%%%%%%%%%%%
%\documentclass[12pt,twoside,dvipdfm]{jarticle}
\documentclass[12pt,twoside,dvipdfm]{msjproc}
\usepackage{amsmath,amssymb,amsthm,amscd}
%%%%%%%%%%%%%%%%%%%%%%%%%%%%%%%%%%%%%%%%%%%%%%%%%%%%%%%%%%%%%%%%%%%%%%%%%%%%
%\usepackage{hyperref}
\usepackage[dvipdfmx]{hyperref}
\usepackage{pxjahyper}
\hypersetup{%
 bookmarksnumbered=true,%
 colorlinks=true,%
 setpagesize=false,%
 pdftitle={\PDFTITLE},%
 pdfauthor={\PDFAUTHOR},%
 pdfsubject={\PDFSUBJECT},%
 pdfkeywords={TeX; dvipdfmx; hyperref; color;}}
\newcommand\arxivref[1]{\href{http://arxiv.org/abs/#1}{\tt arXiv:#1}}
\newcommand\TILDE{\textasciitilde}
\newcommand\US{\textunderscore}
%%%%%%%%%%%%%%%%%%%%%%%%%%%%%%%%%%%%%%%%%%%%%%%%%%%%%%%%%%%%%%%%%%%%%%%%%%%%
\usepackage[dvipdfm]{graphicx}
\usepackage[all]{xy}
%%%%%%%%%%%%%%%%%%%%%%%%%%%%%%%%%%%%%%%%%%%%%%%%%%%%%%%%%%%%%%%%%%%%%%%%%%%%%%
%\pagestyle{plain}
%\setlength{\oddsidemargin}{0cm}
%\setlength{\evensidemargin}{0cm}
%\setlength{\topmargin}{-1.6cm}
%\setlength{\textheight}{24.8cm}
%\setlength{\textwidth}{16cm}
\allowdisplaybreaks
%%%%%%%%%%%%%%%%%%%%%%%%%%%%%%%%%%%%%%%%%%%%%%%%%%%%%%%%%%%%%%%%%%%%%%%%%%%%%%
\usepackage[dvipdfm]{color}
\newcommand\red{\color{red}}
\newcommand\blue{\color{blue}}
\newcommand\green{\color{green}}
\newcommand\magenta{\color{magenta}}
\newcommand\cyan{\color{cyan}}
\newcommand\yellow{\color{yellow}}
\newcommand\white{\color{white}}
\newcommand\black{\color{black}}
\renewcommand\r{\red}
\renewcommand\b{\blue}
%%%%%%%%%%%%%%%%%%%%%%%%%%%%%%%%%%%%%%%%%%%%%%%%%%%%%%%%%%%%%%%%%%%%%%%%%%%%
\makeatletter
  \def\Ddots{\mathinner{\mkern1mu
      \raise\p@\hbox{.}\mkern2mu\raise4\p@\hbox{.}\mkern2mu
      \raise7\p@\vbox{\kern7\p@\hbox{.}}\mkern1mu}}
  \makeatother
%%%%%%%%%%%%%%%%%%%%%%%%%%%%%%%%%%%%%%%%%%%%%%%%%%%%%%%%%%%%%%%%%%%%%%%%%%%%
%\newcommand\N{{\mathbb N}} % natural numbers
\newcommand\Z{{\mathbb Z}} % rational integers
\newcommand\F{{\mathbb F}} % finite field
\newcommand\Q{{\mathbb Q}} % rational numbers
\newcommand\R{{\mathbb R}} % real numbers
\newcommand\C{{\mathbb C}} % complex numbers
\renewcommand\P{{\mathbb P}} % projective spaces
%%%%%%%%%%%%%%%%%%%%%%%%%%%%%%%%%%%%%%%%%%%%%%%%%%%%%%%%%%%%%%%%%%%%%%%%%%%%
%
% 定理環境
%
%\theoremstyle{plain} % 見出しをボールド、本文で斜体を使う
\theoremstyle{definition} % 見出しをボールド、本文で斜体を使わない
\newtheorem{theorem}{定理}
\newtheorem*{theorem*}{定理} % 番号を付けない
\newtheorem{prop}[theorem]{命題}
\newtheorem*{prop*}{命題}
\newtheorem{lemma}[theorem]{補題}
\newtheorem*{lemma*}{補題}
\newtheorem{cor}[theorem]{系}
\newtheorem*{cor*}{系}
\newtheorem{example}[theorem]{例}
\newtheorem*{example*}{例}
\newtheorem{axiom}[theorem]{公理}
\newtheorem*{axiom*}{公理}
\newtheorem{problem}[theorem]{問題}
\newtheorem*{problem*}{問題}
\newtheorem{summary}[theorem]{要約}
\newtheorem*{summary*}{要約}
\newtheorem{guide}[theorem]{参考}
\newtheorem*{guide*}{参考}
%
\theoremstyle{definition} % 見出しをボールド、本文で斜体を使わない
\newtheorem{definition}[theorem]{定義}
\newtheorem*{definition*}{定義} % 番号を付けない
%
%\theoremstyle{remark} % 見出しをイタリック、本文で斜体を使わない
\theoremstyle{definition} % 見出しをボールド、本文で斜体を使わない
\newtheorem{remark}[theorem]{注意}
\newtheorem*{remark*}{注意}
%
\numberwithin{theorem}{section}
\numberwithin{equation}{section}
\numberwithin{figure}{section}
\numberwithin{table}{section}
%
% 引用コマンド
%
\newcommand\secref[1]{第 \ref{#1} 節}
\newcommand\theoremref[1]{定理\ref{#1}}
\newcommand\propref[1]{命題\ref{#1}}
\newcommand\lemmaref[1]{補題\ref{#1}}
\newcommand\corref[1]{系\ref{#1}}
\newcommand\exampleref[1]{例\ref{#1}}
\newcommand\axiomref[1]{公理\ref{#1}}
\newcommand\problemref[1]{問題\ref{#1}}
\newcommand\summaryref[1]{要約\ref{#1}}
\newcommand\guideref[1]{参考\ref{#1}}
\newcommand\definitionref[1]{定義\ref{#1}}
\newcommand\remarkref[1]{注意\ref{#1}}
%
\newcommand\figureref[1]{図\ref{#1}}
\newcommand\figref[1]{\figureref{#1}}
\newcommand\tableref[1]{表\ref{#1}}
%
% \qed を自動で入れない proof 環境を再定義
%
\makeatletter
\renewenvironment{proof}[1][\proofname]{\par
%\newenvironment{Proof}[1][\Proofname]{\par
  \normalfont
  \topsep6\p@\@plus6\p@ \trivlist
  \item[\hskip\labelsep{\bfseries #1}\@addpunct{\bfseries.}]\ignorespaces
}{%
  \endtrivlist
}
\renewcommand{\proofname}{証明}
%\newcommand{\Proofname}{証明}
\makeatother
%
% 正方形の \qed を長方形に再定義
%
\makeatletter
\def\BOXSYMBOL{\RIfM@\bgroup\else$\bgroup\aftergroup$\fi
  \vcenter{\hrule\hbox{\vrule height.85em\kern.6em\vrule}\hrule}\egroup}
\makeatother
\newcommand{\BOX}{%
  \ifmmode\else\leavevmode\unskip\penalty9999\hbox{}\nobreak\hfill\fi
  \quad\hbox{\BOXSYMBOL}}
\renewcommand\qed{\BOX}
%\newcommand\QED{\BOX}
%%%%%%%%%%%%%%%%%%%%%%%%%%%%%%%%%%%%%%%%%%%%%%%%%%%%%%%%%%%%%%%%%%%%%%%%%%%%
\newcommand\g{\mathfrak{g}}
\newcommand\bor{\mathfrak{b}}
\newcommand\nil{\mathfrak{n}}
\newcommand\qP[1]{{\text{$q\mathrm{P}_{\text{#1}}$}}}
\newcommand\PP[1]{{\text{$\mathrm{P}_{\text{#1}}$}}}
\newcommand\arXiv[1]{{\tt arXiv:#1}}
\newcommand\bra{\langle}
\newcommand\ket{\rangle}
\newcommand\eps{\varepsilon}
\newcommand\WW{\widetilde{W}}
\newcommand\A{\mathcal{A}}
\newcommand\K{\mathcal{K}}
\newcommand\hK{{\widehat{\K}}}
\newcommand\B{\mathcal{B}}
\newcommand\diag{\mathop{\mathrm{diag}}\nolimits}
\newcommand\glhat{\mathop{\widehat{\mathrm{gl}}}\nolimits}
\newcommand\tL{{\widetilde{L}}}
\newcommand\tM{{\widetilde{M}}}
\newcommand\cG{{\mathcal{G}}}
\newcommand\tg{{\tilde{g}}}
\newcommand\Ad{\mathop{\mathrm{Ad}}\nolimits}
\newcommand\tC{{\widetilde{C}}}
\newcommand\ta{{\tilde{a}}}
\newcommand\tb{{\tilde{b}}}
\newcommand\tc{{\tilde{c}}}
\newcommand\ts{{\tilde{s}}}
\newcommand\tw{{\tilde{w}}}
\newcommand\tW{{\widetilde{W}}}
\newcommand\hL{{\widehat{L}}}
\newcommand\hM{{\widehat{M}}}
\newcommand\ha{{\hat{a}}}
\newcommand\hb{{\hat{b}}}
\newcommand\hf{{\hat{f}}}
\newcommand\ev{\varepsilon^\vee}
\newcommand\av{\alpha^\vee}
\newcommand\dv{\delta^\vee}
\newcommand\gauge{{\mathrm{gauge}}}
\newcommand\ad{\mathop{\mathrm{ad}}\nolimits}
\makeatletter\newcommand\qbinom{\genfrac[]\z@{}}\makeatother
\newcommand\Qv{{Q^\vee}}
\newcommand\co{c_{\mathrm{odd}}}
\newcommand\cO{{\mathcal{O}}}
%%%%%%%%%%%%%%%%%%%%%%%%%%%%%%%%%%%%%%%%%%%%%%%%%%%%%%%%%%%%%%%%%%%%%%%%%%%%
\begin{document}
%%%%%%%%%%%%%%%%%%%%%%%%%%%%%%%%%%%%%%%%%%%%%%%%%%%%%%%%%%%%%%%%%%%%%%%%%%%%
\title{\TITLE}
\author{\AUTHOR}{\AUTHOREN}
\date{\DATE}
\maketitle
%\begin{abstract}
%  \ABSTRACT
%\end{abstract}
%%%%%%%%%%%%%%%%%%%%%%%%%%%%%%%%%%%%%%%%%%%%%%%%%%%%%%%%%%%%%%%%%%%%%%%%%%%%
\begin{center}
この文書の最新版は次の場所からダウンロードできる. 

\href
{http://www.math.tohoku.ac.jp/~kuroki/LaTeX/20150731QuantizationOfPainleveTau.pdf}
{\small http://www.math.tohoku.ac.jp/{\TILDE}kuroki/LaTeX/20150731QuantizationOfPainleveTau.pdf}

さらに講演スライドのPDFファイルを次の場所からダウンロードできる.

\href
{http://www.math.tohoku.ac.jp/~kuroki/LaTeX/20150914QuantumPainleveTau.pdf}
{\small http://www.math.tohoku.ac.jp/{\TILDE}kuroki/LaTeX/20150914QuantumPainleveTau.pdf}

\end{center}

\begin{center}
Painlev\'e 系の量子化に関する
関連の文書を以下の場所からダウンロードできる.

\href
{http://www.math.tohoku.ac.jp/~kuroki/LaTeX/}
{http://www.math.tohoku.ac.jp/{\TILDE}kuroki/LaTeX/}
\end{center}
%%%%%%%%%%%%%%%%%%%%%%%%%%%%%%%%%%%%%%%%%%%%%%%%%%%%%%%%%%%%%%%%%%%%%%%%%%%%
\tableofcontents
%%%%%%%%%%%%%%%%%%%%%%%%%%%%%%%%%%%%%%%%%%%%%%%%%%%%%%%%%%%%%%%%%%%%%%%%%%%%
%\setcounter{section}{-1} % First section number = 0

\section{Painlev\'e系の量子化 (\qP{IV}を例に用いた解説)}

まず, Painlev\'e系の量子化について説明しよう%
\footnote{実は2次元量子共形場理論の holomorphic part に関する理論(conformal blocks
に関する理論)は量子 Painlev'e 系の理論そのものだとみなせるのだが, 
この論説では扱わない.  たとえば, 共形場理論が生まれた論文 \cite{BPZ} で
扱われているVirasoro代数のみを対称性に持つ共形場理論で
退化場 $\varphi_{12}(z)$, $\varphi_{21}(z)$ を考えた場合はちょうど単独 $2$ 階の
線形常微分作用素のモノドロミー保存変形の理論(Garnier系)の量子化になっている.
この場合に限らず, 一般にVirasoro代数は点付きコンパクトRiemann面の変形を記述している. 
その点の位置がちょうど特異点の場所に対応している.}.

多くの議論が対称化可能一般Cartan行列に付随する一般的な場合に拡張可能だが, 
表現論の言葉に不慣れな読者のために, 
$q$ 差分版のPainlev\'e IV 型方程式の対称形式 \qP{IV} を例に説明して行く.
この場合にはほとんどの結果を工夫のない直接的な計算で確認できる.

目標は量子化された Painlev\'e 系を表現論の言葉でうまく定式化し, 
量子 Painlev\'e 系の基礎になる代数構造の由来を
Kac-Moody代数や量子群の言葉を用いて説明することである.
\qP{IV}の場合には表現論に関係する部分が $3\times 3$ 行列を用いた
巧妙な計算に置き換わることになる.
量子群の $L$-operators による記述に慣れている読者であれば
この場合を見ればより一般の場合にどのようにすればよいかも理解できるはずである.

「q差分化」と「量子化」の区別については\secref{sec:quantization}を, 
古典極限については\secref{sec:classical-limit}に簡単な解説を書いておいた.
以下では「量子化」を「正準量子化」の意味で用いる.



\subsection{$q$-Painlev\'e IV 方程式の対称形式 \qP{IV}}

$q$ 差分版の Painlev\'e IV 方程式の対称形式 $\qP{IV}$ に関する詳しい記述については
文献 \cite{KNY-qPIV}, \cite{Noumi-AWG} を参照せよ.
さらに \qP{IV} の $A^{(1)}_2$ 型のアフィンWeyl群対称性の量子化とその一般化については
文献  \cite{Hasegawa-QB} を見よ. 
(量子化に関する文献を見れば古典極限によってPoisson構造の情報も得られる.)
この節の内容はそれらの文献からの引き写しである. 

まず, $A^{(1)}_2$ 型の一般Cartan行列(GCM) $[a_{ij}]_{i,j=0}^2$ と
反対称行列 $[b_{ij}]_{i,j=0}^2$ を次のように定める%
\footnote{反対称行列 $[b_{ij}]$ はクラスター代数の記述に用いられる
反対称行列と同じものである.}:
\begin{equation*}
  [a_{ij}]_{i,j=0}^2 = 
  \begin{bmatrix}
     2 & -1 & -1 \\
   -1 &  2 & -1 \\
   -1 & -1 & 2 \\
  \end{bmatrix},
  \quad
  [b_{ij}]_{i,j=0}^2 =
  \begin{bmatrix}
      0 & 1 & -1 \\
    -1 & 0 & 1   \\
     1 & -1 & 0 \\
  \end{bmatrix}.
\end{equation*}
さらに, $\C(q)$ 上 $a_i$, $F_i$ ($i=0,1,2)$ から生成される有理函数環を考え, 
そこに Poisson 構造を次のように入れる:
\begin{equation*}
 \{F_i, F_j\} = b_{ij} F_i F_j, \quad \{a_i,a_j\}=\{a_i,F_j\}=0.
\end{equation*}
周期性 $a_{i+3}=a_i$, $F_{i+3}=F_i$ によってインデックスを整数全体に拡張しておく.
$F_i$ を{\bf 従属変数}と呼び, $a_i$ を{\bf パラメーター変数}と呼ぶことにする%
\footnote{古典の場合にはパラメーター変数は数に特殊化されることが多い.
しかし, 量子化された場合には量子化された $\tau$ 変数と量子化されたパラメーター変数が
非可換なので, $\tau$ 変数を除外せずにパラメーター変数を数に特殊化することができなくなる.}.

$q$ 差分版の Painlev\'e IV 方程式の対称形式 \qP{IV} 
とは次のように定義される離散時間発展 $T_\qP{IV}$ のことである(\cite{KNY-qPIV}, \cite{Noumi-AWG}):
\begin{align*}
%  &
  T_\qP{IV}(F_i) = 
  a_i a_{i+1} F_{i+1}
  \frac{1+a_{i-1}F_{i-1}+a_{i-1}a_i F_{i-1}F_i}{1+a_i F_i+a_i a_{i+1} F_i F_{i+1}},
%  \\ &
  \quad
  T_\qP{IV}(a_i) = a_i.
\end{align*}
%この逆は次のように書ける:
%\begin{align*}
%  &
%  T_{\qP{IV}}^{-1}(F_i) = 
%  \frac{F_{i-1}}{a_i a_{i+1}}
%  \frac{a_i a_{i+1} + a_i F_{i+1} + F_i F_{i+1}}{a_{i-1}a_i + a_{i-1}F_i + F_{i-1}F_i},
%  \\ &
%  \quad
%  T_{\qP{IV}}^{-1}(a_i) = a_i.
%\end{align*}
この離散時間発展はPoisson構造を保つ.

$A^{(1)}_2$ 型の拡大アフィンWeyl群 $\WW(A^{(1)}_2)=\bra s_0,s_1,s_2,\pi\ket$ が
次の関係式で定義される:
\begin{equation*}
 s_i^2=1, \quad
 s_is_{i+1}s_i=s_{i+1}s_is_{i+1}, \quad
 \pi s_i = s_{i+1} \pi.
\end{equation*}
ただしインデックスを周期性 $s_{i+3}=s_i$ によって整数全体に拡張しておいた.

拡大アフィンWeyl群 $\WW(A^{(1)}_2)$ の作用を次のように定める:
\begin{equation*}
  s_i(a_j) = a_i^{-{a_{ij}}}a_j,  \quad 
  s_i(F_j) = F_j \left(\frac{a_i+F_i}{1+a_i F_i}\right)^{b_{ij}}, \quad
  \pi(a_i)=a_{i+1}, \quad
  \pi(F_i)=F_{i+1}.
\end{equation*}
この作用はPoisson構造を保ち, 離散時間発展 $T_{\qP{IV}}$ と可換であり, 
$q$ 差分版の Painlev\'e IV 方程式の対称性(B\"acklund変換)になっている.

実は以上の構造の背景には量子群が隠れている.
以下の節ではそのことを説明したい.

結論を先走って言うと, $q$ 差分版の Painlev\'e 方程式の従属変数 $F_i$ は
量子展開環の下三角部分の Chevalley 生成元 $\varphi_i$ の化身になっている%
\footnote{$\varphi$ は $q$-Serre 関係式を満たしている.}.

ただし,  $F_i$ が量子展開環のした三角部分の Chevalley 生成元 $\varphi_i$ の像に
直接なっているのではなく, 
$\varphi_i$ の余積を $\Delta(\varphi_i)=\varphi_{i1}+\varphi_{i2}$, 
$\varphi_{i1}=\varphi_i\otimes k_i$, $\varphi_{i2}=1\otimes\varphi_i$ 
と書くとき,  $a_i F_i$ が $\varphi_{i1}\varphi_{i2}^{-1}$ の像になっている
という若干複雑な事情になっている.
そして $s_i$ の作用は $\Delta(\varphi_i)$ の像 $f_i$ のべき $f_i^\gamma$ の作用を
用いて構成される.   詳しくは \cite{Kuroki-W} を参照して欲しい%
\footnote{変数 $\gamma$ によるべき $f_i^\gamma$ の構成の仕方について
は\secref{sec:powers1}に解説を書いておいた.}. 



\subsection{$\WW(A^{(1)}_2)\times\WW(A^{(1)}_1)$ 対称性}

この節の内容は文献 \cite{KNY-WxW}, \cite{KNY-qKP}, \cite{NY-RSK} の構成
を $(m,n)=(3,2)$ の場合に特殊化したものになっている.
互いに素な任意の $(m,n)$ の場合の量子化について
は \cite{Kuroki-WxW2010}, \cite{Kuroki-WxW2013} を見よ.

準周期性 $t_{i+3}=r^{-1} t_i$, $x_{i+3}=r^{-1} x_i$, $y_{i+3}=r^{-1} y_i$ を
満たす%
\footnote{このとき直後に定義される $a_i$ たちは $a_0a_1a_2=r$ をみたしている.
だから文献 \cite{KNY-qPIV} における $q$ はここでの $r$ に対応している.
記号 $q$ は量子群の変形パラメーターのために取っておくことにする.}%
変数 $t_i$, $x_i$, $y_i$ を用意し, 
$t_i^2 = x_i y_i$ が成立していると仮定し, 
Poisson 構造を次のように定める%
\footnote{この関係式は $3$ を3以上の任意の奇数 $m$ に
一般化しても $\mu=1,2,\dots,m-1$ について有効である.
ここで扱っている \qP{IV} のケースは $(n,m)=(3,2)$ の場合に対応しており, 
互いに素な任意の $(m,n)$ の場合に一般化される. 
そのとき変数 $x_i$, $y_i$ は $mn$ 個の変数 $x_{ik}$ に拡張されるが, 
$x_{ik}$ たちのPoisson括弧
も $\{x_{ik},x_{jl}\}=\eps_{ijkl} x_{ik}x_{jl}$, $\eps_{ijkl}=0,\pm1$ の形になるが, 
$\eps_{ijkl}$ がどのように $0,\pm1$ になっているかは複雑である.
詳しくは量子化された場合を扱っている \cite{Kuroki-WxW2010}, \cite{Kuroki-WxW2013} を見よ.
$n\geqq 3$ の場合のPoisson構造は量子化されて初めて明らかになった.}%
: $\mu=1,2$ について, 
\begin{alignat*}{3}
 &
 \{x_i, y_i\}=0, \quad
 \\ &
 \{x_i,x_{i+\mu}\} = (-1)^{\mu-1} x_i x_{i+\mu}, \quad
 & &
 \{x_i,y_{i+\mu}\} = - (-1)^{\mu-1} x_i y_{i+\mu}, \quad
 \\ &
 \{y_i,y_{i+\mu}\} = (-1)^{\mu-1} y_i y_{i+\mu}, \quad
 & &
 \{y_i,x_{i+\mu}\} = - (-1)^{\mu-1} y_i x_{i+\mu}, \quad
 \\ &
 \{t_i, \xi_j\}=0 \qquad (\xi=x,y,t).
\end{alignat*}
最後の行の関係式を忘れて, 変数 $t_i$ をPoisson中心元 $x_i y_i$ の平方根として導入してもよい.
このとき, $F_i$ と $a_i$ を
\begin{equation*}
 a_i = \frac{t_i}{t_{i+1}}, \quad F_i = \frac{x_{i+1}x_i}{t_{i+1}t_i}.
\end{equation*}
と定める. これらは周期性 $a_{i+3}=a_i$, $F_{i+3}=F_i$ を満たしており, 
前節のPoisson構造の定義式を満たしている.
この関係があるので $x_i$, $y_i$ たちをも{\bf 従属変数}と呼び, 
$t_i$ たちをも{\bf パラメーター変数}と呼ぶことにする.

$A^{(1)}_1$ 型の拡大アフィンWeyl群 $\WW(A^{(1)}_1)=\bra r_0,r_1,\varpi\ket$ が
次の関係式で定義される:
\begin{equation*}
 r_i^2=1, \quad
 \varpi r_i = r_{i+1} \varpi.
\end{equation*}
ただしインデックスを周期性 $r_{i+2}=r_i$ によって整数全体に拡張しておいた. 

拡大アフィンWeyl群の直積 $\WW(A^{(1)}_2)\times\WW(A^{(1)}_1)$ の $x_i$, $y_i$, $t_i$ への
作用を以下のように定めることができる%
\footnote{この意味がわからない天下りの式を直接扱うのは得策ではない. 
次節で解説するLax表示によるWeyl群作用の記述の方を使った方がよい.}%
:
\begin{align*}
 &
 s_i(x_i) 
 = x_i - (y_i x_i - y_{i+1}x_{i+1})(y_i+x_{i+1})^{-1}
 = (x_i+y_{i+1})x_{i+1}(y_i+x_{i+1})^{-1},
 \\ &
 s_i(x_{i+1})
 = x_{i+1} +(x_i+y_{i+1})^{-1}(y_i x_i - y_{i+1}x_{i+1})
 =(x_i+y_{i+1})^{-1} x_i (y_i+x_{i+1}),
 \\ &
 s_i(y_i) 
 = y_i - (x_i y_i - x_{i+1}y_{i+1})(x_i+y_{i+1})^{-1}
 = (y_i+x_{i+1})y_{i+1}(x_i+y_{i+1})^{-1},
 \\ &
  s_i(y_{i+1})
 = y_{i+1} +(y_i+x_{i+1})^{-1}(x_i y_i - x_{i+1}y_{i+1})
 =(y_i+x_{i+1})^{-1} y_i (x_i+y_{i+1}),
 \\ &
 s_i(x_{i+2})=x_{i+2},
 \quad
 s_i(y_{i+2})=y_{i+2}, 
 \\ &
 s_i(t_i)=t_{i+1},  \quad
 s_i(t_{i+1})=t_i,  \quad
 s_i(t_{j+2})=t_{j+2},
 \\ &
 \pi(x_i) = x_{i+1},  \quad
 \pi(y_i) = y_{i+1},  \quad
 \pi(t_i) = t_{i+1},
 \\[\medskipamount] &
 Q_i := y_{i+2}y_{i+1} + y_{i+2}x_i + x_{i+1}x_i, 
 \\ &
 r_1(x_i) 
 = x_i - r Q_{i+1}^{-1}(x_{i+3}x_{i+2}x_{i+1}-y_{i+4}y_{i+3}y_{i+2})
 = r^{-1} Q_{i+1}^{-1}y_i Q_i,
 \\ &
 r_1(y_i)
 = y_i + r(x_{i+2}x_{i+1}x_i-y_{i+3}y_{i+2}y_{i+1})Q_i^{-1}
 = r Q_{i+1} x_i Q_i^{-1},
 \\ &
 r_1(t_i) = t_i, \quad
 \varpi(x_i)=y_i,   \quad
 \varpi(y_i)=x_i,   \quad
 \varpi(t_i)=t_i.
\end{align*}
%$r_0=\varpi r_1 \varpi^{-1}$ なので, $\WW(A^{(1)}_1)$ の作用を定めるため
%には $r_1$ と $\varpi$ の作用を定めれば十分であることに注意せよ.
この作用は Poisson 構造と $F_i$, $a_i$ で生成される部分体を保ち, 
$s_i$, $\pi$ の $F_i$, $a_i$ への作用は前節に定義したものと一致している.

$U_1=r_1 \varpi$ とおく. 
このとき $U_1$ は以下を満たしている:
\begin{align*}
 &
 U_1(x_i) = r Q_{i+1} x_i Q_i^{-1},
 \qquad
 U_1(y_i) = r^{-1} Q_{i+1}^{-1}y_i Q_i,
 \qquad
 U_1(t_i) = t_i.%,
% \\[\medskipamount] &
% Q'_i := \varpi(Q_i) = x_{i+2}x_{i+1} + x_{i+2}y_i + y_{i+1}y_i, 
% \\ &
% U_1^{-1}(x_i) = r^{-1} {Q'_{i+1}}^{-1}x_i Q'_i,
% \qquad
% U_1^{-1}(y_i) = r Q'_{i+1} y_i {Q'_i}^{-1},
% \qquad
% U_1^{-1}(t_i) = t_i.
\end{align*}
これらの公式と
\begin{align*}
 &
 x_{i+1}x_i = t_i t_{i+1} F_i, \quad
 y_{i+2}x_i = t_i t_{i+2} F_{i+1}^{-1} F_i,
% \\ &
\quad
% x_{i+2}y_i = t_i t_{i+2} F_{i+1} F_i^{-1}, \quad
 y_{i+2}y_{i+1} = t_{i+1} t_{i+2}F_{i+1}^{-1}
\end{align*}
から, $U_1$ の $F_i$, $a_i$ への作用は
前節で定義した $q$ 差分版の Painlev\'e IV 方程式の離散時間発展 $T_\qP{IV}$ に
一致することも確かめられる%
\footnote{$U_1$ の $F_i$, $a_i$ への作用は $r$ の値によらない.}.

したがって, この節の内容を量子化可能ならば, 
$q$ 差分版 Painlev\'e IV 方程式を量子化できる.
そのためにはこの節で登場した変数 $t_i$, $x_i$, $y_i$ と
拡大Weyl群の作用を量子群の言葉で理解することが必要になる.  

\subsection{$\WW(A^{(1)}_2)\times\WW(A^{(1)}_1)$ の作用のLax表示}

量子化を始める前に $\WW(A^{(1)}_2)\times\WW(A^{(1)}_1)$ の作用のLax表示について
説明しておこう.

変数 $z,w$ を用意して, 
行列 $\Lambda_3(z)$, $\Lambda_2(w)$, $X(z)$, $Y(z)$, $V_i(w)$ を次のように定める:
\begin{align*}
 &
 \Lambda_3(z) =
 \begin{bmatrix}
  0 & 1 & 0 \\
  0 & 0 & 1 \\
  z & 0 & 0 \\
 \end{bmatrix},
 \quad
 \Lambda_2(w) =
 \begin{bmatrix}
  0 & 1 \\
  w & 0 \\
 \end{bmatrix},
 \\ &
 X(z) =
 \begin{bmatrix}
  x_1 & 1 & 0 \\
  0 & x_2 & 1 \\
  z & 0 & x_3 \\
 \end{bmatrix},
 \quad
 Y(z) =
 \begin{bmatrix}
  y_1 & 1 & 0 \\
  0 & y_2 & 1 \\
  z & 0 & y_3 \\
 \end{bmatrix},
 \quad
 V_i(w) =
 \begin{bmatrix}
  y_i &  1  \\
  w   & x_i \\
 \end{bmatrix}.
\end{align*}
$X(z)$, $Y(z)$, $V_i(z)$ を {\bf local $L$-operators} と呼ぶ.
さらに行列 $G_i$, $G'_i$ ($i=1,2$), $R_i$ ($i\in\Z$) を次のように定める:
\begin{align*}
 &
 G_1 =
 \begin{bmatrix}
  1   & 0 & 0 \\
  g_1 & 1 & 0 \\
  0 & 0 & 1 \\
 \end{bmatrix},
 \quad
 G_2 =
 \begin{bmatrix}
  1 & 0 & 0 \\
  0 & 1 & 0 \\
  0 & g_2 & 1 \\
 \end{bmatrix},
 \quad
 G'_i = \varpi(G_i),
 \quad
 R_i =
 \begin{bmatrix}
  1      & 0 \\
  \rho_i & 1 \\
 \end{bmatrix}.
\end{align*}
ここで
\begin{equation*}
 g_i = \frac{x_i y_i-x_{i+1}y_{i+1}}{x_i+y_{i+1}},
 \quad
 \rho_i = r \frac{x_{i+2}x_{i+1}x_{i}-y_{i+3}y_{i+2}y_{i+1}}{Q_i}.
\end{equation*}
$\varpi$ は $x_i$ と $y_i$ を交換する操作であった.

このとき, 前節で構成した拡大アフィンWeyl群の $x_i$, $y_i$ たちへの作用の
定義を以下のように書き直せる:
\begin{alignat*}{2}
 &
 s_i(X(z)) = G_i X(z) {G'_i}^{-1}, \qquad
 & & 
 s_i(Y(z)) = G'_i Y(z) G_i^{-1},
 \\ &
 \pi(X(z)) = \Lambda_3(z) X(rz) \Lambda_3(rz)^{-1}, \qquad
 & &
 \pi(Y(z)) = \Lambda_3(z) Y(rz) \Lambda_3(rz)^{-1},
 \\ &
 r_1(V_i(w)) = R_{i+1}^{-1} V_i(w) R_i,  \qquad
 & & 
 \varpi(V_i(w)) = \Lambda_2(w) V_i(w) \Lambda_2(w)^{-1}.
\end{alignat*}
これを拡大アフィンWeyl群の作用の{\bf Lax表示}と呼ぶ.
さらに, 以下の条件で上記の拡大アフィンWeyl群の $x_i$, $y_i$ への作用を
特徴付けることもできる:
\begin{align*}
 &
 r_1(X(z)Y(rz)) = X(z)Y(rz), 
 \\ &
 r_1(x_{i+2}x_{i+1}x_i) = y_{i+3}y_{i+2}y_{i+1}, \quad
 r_1(y_{i+3}y_{i+2}y_{i+1}) = x_{i+2}x_{i+1}x_i,
 \\ &
 \varpi(X(z)) = Y(z), \quad
 \varpi(Y(z)) = X(z),
 \\[\medskipamount] &
 s_i(V_{i+1}(w)V_i(w)) = V_{i+1}(w)V_i(w), \quad
 s_i(V_{i+2}(w))=V_{i+2}(w),
 \\ &
 s_i(x_i y_i) = x_{i+1}y_{i+1}, \quad 
 s_i(x_{i+1}y_{i+1})= x_i y_i,
 \\ &
 \pi(V_i(w)) = V_{i+1}(w).
\end{align*}
これらの事実は公式 \(
 Q_{i+1}x_i - y_{i+3}Q_i = x_{i+2}x_{i+1}x_i - y_{i+3}y_{i+2}y_{i+1}
\) などを使えば確認できる%
\footnote{この $(m,n)=(3,2)$ の場合に計算を地道にまとめておけば一般の場合に
どんな感じになっているかの感触がつかめる. 
うまく行く計算の仕組みを見付けるのは大変だが, 
ある特別な場合にうまく行くことを確認できれば, それを一般化するのは易しいことが多い.}.
そして, それらの事実を使うことによって
実際に拡大アフィンWeyl群の作用が定まっていることも確認できる%
\footnote{拡大アフィンWeyl群作用については, 
前節の天下り的な意味がよく分からない定義を出発点にするのではなく, 
この節のLax表示を用いた定義の方を出発点に採用した方がよい.}.


Weyl群作用のLax表示を初見の人のために計算のポイントがどこにあるかを解説しておく.
計算のポイントは $2\times 2$ 行列に関して
\begin{equation*}
 \begin{bmatrix}
   a & b \\
   0 & c \\
 \end{bmatrix}
 =
 \begin{bmatrix}
   1 & 0 \\
   g & 1 \\
 \end{bmatrix}
 \begin{bmatrix}
   a & 0 \\
   0 & c \\
 \end{bmatrix}
 \begin{bmatrix}
   1  & 0 \\
   g' & 0 \\
 \end{bmatrix}
\end{equation*}
および $g'=-g$ をみたす $g$ が $a,b,c$ から $g=(a-c)/b$ と一意に定まることである%
\footnote{$g'=-g$ を仮定しない幾何クリスタルへの一般化もある. 
その一般化はさらに非可換な場合に一般化される.
非可換な場合への一般化の例は\secref{sec:f^gamma}の最後の方および
\secref{sec:fhat}にある.}.
$s_i$ の作用のLax表示を記述する行列 $G_i$ は行列 $X(z)Y(rz)$ に
この計算を適用することによって得られる.



\subsection{量子展開環のChevalley生成元の像のべき $f_i^\gamma$ の作用}
\label{sec:f^gamma}

$q$ 差分版 Painlev\'e IV 方程式とその対称性の量子化を始めよう.
まず, 基礎になる非可換環を量子群の方法を用いて構成しよう.

$\B$ は生成元 $a_{ik}^{\pm1}$, $b_{ik}^{\pm1}$ ($i=1,2,3$, $k=1,2$) と
以下の関係式で定義される $\C(q)$ 上の代数であるとする:
\begin{align*}
 &
 a_{ik}a_{ik}^{-1} = a_{ik}^{-1}a_{ik} = 1, \quad
 b_{ik}b_{ik}^{-1} = b_{ik}^{-1}b_{ik} = 1
 \qquad (\xi=a_{ik},b_{ik}),
 \\ &
 a_{ik} b_{ik} = q^{-1} b_{ik} a_{ik}, \quad
 a_{ik} b_{i+1,k} = q b_{i+1,k} a_{ik}, \quad
 a_{ik} b_{i+2,k} = b_{i+2,k} a_{ik}, 
 \\ &
 a_{ik} a_{jk} = a_{jk} a_{ik}, \quad
 b_{ik} b_{jk} = b_{jk} b_{ik}, \quad
 \xi_{i1} \eta_{j2} = \eta_{j2} \xi_{i1} \qquad (\xi,\eta=a,b).
\end{align*}
ただし $a_{i+3,k}=a_{ik}$, $b_{i+3,k}=b_{ik}$ によって
インデックス $i$ を整数全体に拡張しておいた.

$A^{(1)}_2$ 型の量子 $R$ 行列 $R(z)$ を次のように定める:
\begin{align*}
 R(z)
 & = (q-q^{-1}z) \sum_i E_{ii}\otimes E_{ii}
 + (1-z)       \sum_{i\ne j} E_{ii}\otimes E_{jj}
 \\ & \,
 + (q-q^{-1})  \sum_{i<j} (E_{ij}\otimes E_{ji} + z E_{ji}\otimes E_{ij}).
\end{align*}
ここで $i,j$ は $1,2,3$ を動き, $E_{ij}$ は $(i,j)$ 成分のみが $1$ で
他の成分が $0$ であるような $3\times 3$ 行列(行列単位)であるとする. 
さらに行列 $L_k(z)$ を次のように定める:
\begin{equation*}
 L_k(z) =
 \begin{bmatrix}
    a_{1k} & b_{1k} & 0 \\
    0 & a_{2k} & b_{2k} \\
  z b_{3k} & 0 & a_{3k} \\
 \end{bmatrix}.
\end{equation*}
この $L_k(z)$ たちを {\bf local $L$-operators} と呼ぶ.
このとき代数 $\B$ の定義関係式は次の ``$RLL=LLR$'' 関係式に書き直される:
\begin{align*}
 &
 R(z/w)L_k(z)^1 L_k(w)^2 = L_k(z)^2 L_k(z)^1 R(z/w)  \quad (k=1,2), 
 \\ &
 L_1(z)^1 L_2(w)^2 = L_2(w)^2 L_1(z)^1.
\end{align*}
ただし $L_1(z)^1=L_1(z)\otimes 1$, $L_2(w)^2=1\otimes L_2(w)$ と定めた.

以上の構成は量子群の世界では標準的でよく知られている.

量子展開環 $U_q(\glhat_3)$ の下Borel部分代数 $U_q(\bor_-)$ は
生成元 $\kappa_i^{\pm 1}$, $\varphi_i$ ($i=1,2,3$) と次の関係式で定義される代数である:
\begin{align*}
 &
 \kappa_i\kappa_i^{-1} = \kappa_i^{-1}\kappa_i = 1, \quad
 \kappa_i\kappa_j = \kappa_j\kappa_i,
 \\ &
 \kappa_i \varphi_i \kappa_i^{-1} = q^{-1} \varphi_i, \quad
 \kappa_{i+1} \varphi_i \kappa_{i+1}^{-1} = q \varphi_i, \quad
 \kappa_{i+2} \varphi_i \kappa_{i+2}^{-1} = \varphi_i,
 \\ &
 \varphi_i^2 \varphi_{i\pm1} 
 - (q+q^{-1})\varphi_i\varphi_{i\pm1}\varphi
 + \varphi_{i\pm1}\varphi^2 = 0.
\end{align*}
ただしインデックスを $3$ 周期的に整数全体に拡張しておいた.
最後の関係式を $q$-Serre 関係式と呼び, 
$\varphi_i$ たちを量子展開環の下三角部分の {\bf Chevalley 生成元}と呼ぶことにする.
$\kappa_i^{\pm1}$ たちから生成される部分代数は {\bf Cartan 部分代数}と呼ばれる.


各 $k$ ごとに次によって代数準同型 $U_q(\bor_-)\to\B$ を定めることができる:
\begin{equation*}
  \kappa_i^{\pm1}\mapsto a_{ik}^{\pm1},  \qquad
  \varphi_i \mapsto a_{ik}^{-1} b_{ik}.
\end{equation*}
ゆえに $a_{ik}^{\pm 1}$ と $b_{ik}$ で生成される $\B$ の部分代数
は $U_q(\glhat_3)$ の下Borel部分代数の
テンソル積 $U_q(\bor_-)\otimes U_q(\bor_-)$ の
\begin{align*}
  &
  \kappa_i^{\pm1}\otimes 1 \mapsto a_{i1}^{\pm1},  \quad
  \varphi_i\otimes 1 \mapsto a_{i1}^{-1} b_{i1},
%  \\ &
\quad
  1\otimes \kappa_i^{\pm1} \mapsto a_{i2}^{\pm1},  \quad
  1\otimes\varphi_i\otimes 1 \mapsto a_{i2}^{-1} b_{i2}.
\end{align*}
で定まる代数準同型写像の像になっている.

よく知られているように, local $L$-operaors の積 $L(z):=L_1(z)L_2(z)$ は
量子展開環の余積に対応している.
$U_q(\bor_-)$ の余積が
\begin{equation*}
  \Delta(\kappa_i^{\pm1}) = \kappa_i^{\pm1}\otimes\kappa_i^{\pm1}, \qquad
  \Delta(\varphi_i) 
  = \varphi_i\otimes \kappa_i^{-1}\kappa_{i+1} + 1\otimes\varphi_i
\end{equation*}
で定義される. 行列 $L(z):=L_1(z)L_2(z)$ の $(i,i)$ 成分は $\Delta(\kappa_i)$ の
像に一致し, $(i,i+1)$ 成分を左から $(i,i)$ 成分で割ってできる元
\begin{equation*}
  f_i := a_{i1}^{-1}b_{i1}a_{i2}^{-1}a_{i+1,2} + a_{i2}^{-1}b_{i2}
\end{equation*}
は量子展開環の下三角部分のChevalley生成元の余積 $\Delta(\varphi_i)$ の像に一致する.
	


行列 $L(z) = L_1(z)L_2(z)$ の対角部分を
\begin{equation*}
  L_0=\diag(\ta_1,\ta_2,\ta_3), \qquad
  \ta_i = a_{i1}a_{i2} \quad (i=1,2,3)
\end{equation*}
と書き, 行列 $\tL(z) := L_1(z)L_2(z)L_0$ を次のように書くことにする%
\footnote{量子展開環の余積に対応する標準的な $L$-operator $L(z)=L_1(z)L_2(z)$ ではなく, 
対角部分 $L_0$ を二重にした $\tL(z)=L(z)L_0$ を主に使うことになることに注意せよ.}:
\begin{equation*}
 \tL(z) = L_1(z)L_2(z)L_0 =
 \begin{bmatrix}
     \ta_1^2 & b_1     & c_1 \\
   z c_2     & \ta_2^2 & b_2 \\
   z b_3     & z c_3   & \ta_3^2 \\
 \end{bmatrix},
 \quad
 \begin{cases}
  b_i = (a_{i1}b_{i2}+b_{i1}a_{i+1,2})a_{i+1,1}a_{i+1,2}, \\
  c_i = b_{i1}b_{i+1,2}a_{i+2,1}a_{i+2,2}.
 \end{cases}
\end{equation*}
さらに $\ta_i$, $b_i$, $c_i$ のインデックスを $3$ 周期的に整数全体に拡張しておく.

量子展開環の下三角部分のChevalley生成元の余積 $\Delta(\varphi_i)$ の
像 $f_i$ は次のように表わされるのであった:
\begin{equation*}
  f_i = \ta_i^{-1}b_i\ta_{i+1}^{-1} = f_{i1} + f_{i2}, \quad
  f_{i1} = a_{i1}^{-1}b_{i1}a_{i2}^{-1}a_{i+1,2}, \quad
  f_{i2} = a_{i2}^{-1}b_{i2}
\end{equation*}
$f_i$ たちは $\varphi_i$ の余積たちの像なので特に $q$-Serre 関係式を満たしている. 
ゆえに変数 $\gamma$ による $f_i$ のべき $f_i^\gamma$ を含む代数を
構成することができ(\secref{sec:powers1}), 次の公式が成立することを示せる
(\secref{sec:powers2}, \cite{Kuroki-W}, \cite{Kuroki-Tau}): $a_{ij}=a_{ji}=-1$ のとき
\begin{align*}
 &
 f_i^\gamma f_j f_i^{-\gamma}
 = q^{\pm\gamma}f_j+[\gamma]_q(f_i f_j - q^{\pm1} f_j f_i)f_i^{-1}
 = [1-\gamma]_q f_j + [\gamma]_q f_i f_j f_i^{-1},
 \\ &
 f_i^\beta f_j^{\beta+\gamma} f_i^\gamma
 = f_j^\gamma f_i^{\beta+\gamma} f_j^\beta.
\end{align*}
ここで $[\gamma]_q = (q^\gamma-q^{-\gamma})/(q-q^{-1})$.
後者の関係式は {\bf Verma 関係式}と呼ばれている.

$\ta_i^{\pm 1}$, $b_i$, $c_i$ で生成される $\B$ の部分代数
は $\ta_i^{\pm 1}$, $f_i$, $c_i$ で生成される部分代数
に一致しており, それらは以下の関係式を満たしていることを直接の計算で確認できる:
\begin{align*}
 &
 f_i f_{i+1} - q f_{i+1} f_i = (1-q^2) \ta_i^{-1}c_i\ta_{i+1}^{-1}, 
 \\ &
 f_i c_j = c_j f_i, \quad
 \ta_i\ta_j = \ta_j\ta_i, 
 \\ &
 c_i c_{i-2} = q c_{i-2} c_i, \quad
 c_i c_{i+2} = q^{-1} c_{i+2} c_i, 
 \\ &
 \ta_i     f_i = q^{-1} f_i \ta_i, \quad
 \ta_{i+1} f_i = q      f_i \ta_{i+1}, \quad
 \ta_{i+2} f_i = f_i\ta_{i+2}, 
 \\ &
 \ta_i     c_i = q^{-1} c_i \ta_i, \quad
 \ta_{i+1} c_i =        c_i \ta_{i+1}, \quad
 \ta_{i+2} c_i = q      c_i \ta_{i+2}.
\end{align*}
これらの関係式を使えば必要な公式はすべて計算できる.
特に重要なのは, $c_i$ たちが $f_j$ たちと可換になり, 
$c_i$ たちの積が $q$ べき因子の違いを除けば可換になることである.
この事実は後で変数 $f_i$ を変数変換するときに使われる.

$\tL(z)$ の成分で $f_i$ と非可換なの
は $\ta_i^2$, $\ta_{i+1}^2$, $b_{i-1}$, $b_{i+1}$ の4つだけであり, 
\begin{align*}
 &
 f_i^\gamma b_{i+1} f_i^{-\gamma}
 = b_{i+1} + (q^{-2\gamma}-1)\ta_i^2 b_i^{-1} c_i,
 \\ &
 f_i^\gamma b_{i-1} f_i^{-\gamma}
 = b_{i-1} + (q^{2\gamma}-1)c_{i-1}b_i^{-1}\ta_i^2,
 \\ &
 f_i^\gamma \ta_i f_i^{-\gamma} = q^\gamma \ta_i, 
 \\ &
 f_i^\gamma \ta_{i+1} f_i^{-\gamma} = q^{-\gamma} \ta_{i+1} 
\end{align*}
が成立していることも確認できる.

変換 $\Ad(f_i^\gamma)(x)=f_i^\gamma x f_i^{-\gamma}$ のLax表示を得る
ために次のようにおく:
\begin{equation*}
 \cG_{i,c}  = y_i((c^{ 2}-1)\ta_{i+1}^2 b_i^{-1}), \quad
 \cG'_{i,c} = y_i((c^{-2}-1)b_i^{-1}\ta_i^2).
\end{equation*} 
ただし,
\begin{align*}
 &
 y_1(a) =
 \begin{bmatrix}
   1 & 0 & 0 \\
   a & 1 & 0 \\
   0 & 0 & 1 \\
 \end{bmatrix},
 \quad
 y_2(a) =
 \begin{bmatrix}
   1 & 0 & 0 \\
   0 & 1 & 0 \\
   0 & a & 1 \\
 \end{bmatrix},
 \quad
 y_3(a) =
 \begin{bmatrix}
   1 & 0 & z^{-1}a \\
   0 & 1 & 0 \\
   0 & 0 & 1 \\
 \end{bmatrix}.
\end{align*}
このとき, $f_i^\gamma$ の作用に関する上の公式より次が成立している:
\begin{align*}
  f_i^\gamma \tL(z) f_i^{-\gamma} = \cG_{i,c} \tL(z) \cG'_{i,c}, \quad
  c = q^{-\gamma}.
\end{align*}
これを $\Ad(f_i^\gamma)$ の{\bf Lax表示}と呼ぶ%
\footnote{このタイプの公式を得るためには $L_1(z)L_2(z)$ そのものではなく, 
その対角部分 $L_0$ を二重化した $\tL(z)=L_1(z)L_2(z)L_0$ を使わなければいけない.}.




\subsection{不変部分代数への制限}
\label{sec:fhat}

代数 $\B$ の中心可逆元 $r$ を
\begin{equation*}
 r = c_1 c_3 c_2
\end{equation*}
と定め%
\footnote{より一般に local $L$-operators の行列としてのサイズ $3$ 
を $3$ 以上の奇数 $m$ に置き換えた場合には $r$ を \(
 r = c_1 c_3 \cdots c_m c_2 c_4 \cdots c_{m-1}
\) と定める.}, %
対角行列 $\tC$ と $t_i\in\B$ を次のように定める%
\footnote{行列のサイズが $3$ 以上の奇数 $m$ の場合には,
$\tC=\diag(\tc_1,\ldots,\tc_m)$ を次のように定める: 
\[
 \tC
 := \diag
  (1, 
   \co,                      c_1, 
   \co c_2,                  c_1c_3, 
   \co c_2c_4,               c_1c_3c_5, 
   \ldots, 
   \co c_2c_4\cdots c_{m-3}, c_1c_3\cdots c_m).
\]
ここで $\co=c_1c_3\cdots c_m$.}:
\begin{equation*}
 \tC = \diag(\tc_1,\tc_2,\tc_3) := \diag(1, c_1 c_3, c_1), \quad
 t_i = \tc_i \ta_i \tc_i^{-1}.
\end{equation*}
この $t_i$ と $\ta_i=a_{i1}a_{i2}$ は $q$ べき因子の違いを除いて一致する.
$\tc_i$ のインデックスは整数全体に拡張{\bf しない}ことにする.

このとき, 行列 $\hL(z) := \tC \tL(z) \tC^{-1}$ は次の形になる:
\begin{equation*}
 \hL(z) = \tC \tL(z) \tC^{-1} =
 \begin{bmatrix}
   t_1^2   & \hb_1 & 1     \\
   rz      & t_2^2 & \hb_2 \\
   zr\hb_3 & z     & t_3^2 \\
 \end{bmatrix},
 \qquad
 \begin{cases}
  \hb_1 = \tc_1 b_1 \tc_2^{-1}, \\
  \hb_2 = \tc_2 b_2 \tc_3^{-1}, \\
  \hb_3 = r^{-1} \tc_3 b_3 \tc_1^{-1}.
 \end{cases}
\end{equation*}
前節までは $3$ 周期的にインデックスを整数全体に拡張していたが, 
この節では次の{\bf 準}周期性によって $t_i$, $\hb_i$ のインデックスを整数全体に拡張しておく:
\begin{equation*}
 t_{i+3} = r^{-1}t_i, \quad
 \hb_{i+3} = r^{-1} \hb_i.
\end{equation*}

このとき, $t_i$ たちは, 互いに可換なだけではなく, $\hb_j$ たちとも可換になる%
\footnote{この事実がとても重要である.
$\ta_i^{\pm1}$, $b_i^{\pm1}$, $c_i^{\pm1}$ で生成される代数には, 
可逆な複素対角行列全体のなす群が
行列 $\tL(z)$ の相似変換の形で作用している.
その作用で不変な元全体のなす部分代数は $t_i^{\pm1}$, $\hb_i^{\pm1}$, $r^{\pm1}$ で生成される.
すなわち行列 $\hL(z)$ を考えることは可逆な対角行列による相似変換で
不変な部分代数を考えることに対応している.
$\hL(z)$ のすべての成分が $t_i$ と可換になるのはこの不変性の帰結である.}.
そこで $t_i$ たちを{\bf 量子化されたパラメーター変数}と呼ぶことにする.
$\hL(z)$ の成分で $f_i$ と非可換なのは $t_i^2$, $t_{i+1}^2$, $\hb_{i+1}$, $\hb_{i-1}$ 
の4つだけである. 

$\Ad(f_i^\gamma)$ の $\hL(z)$ の作用のLax表示を得るために次のようにおく:
\begin{equation*}
 G_{i,c} = \tC \cG_{i,c} \tC^{-1}, \quad
 G''_{i,c} = \tC \cG'_{i,c} \tC^{-1}.
\end{equation*}
後で $G'_{i,c}$ を別に定義するのでこのような記号法になっている.
このとき, 
\begin{align*}
 &
 G_{i,c} = y_i((c^2-1)t_{i+1}^2 \hb_i^{-1}),   \quad
 G''_{i,c} = y_i((c^{-2}-1)\hb_i^{-1}t_i^2)  \quad
 (i=1,2),
 \\ &
 G_{3,c} = y_3(r^{-1}(c^2-1)t_1^2\hb_3^{-1}), \quad
 G''_{3,c} = y_3(r^{-1}(c^{-2}-1)\hb_3^{-1}t_3^2).
\end{align*}
前節の結果より, 次の公式が成立している:
\begin{equation*}
  f_i^\gamma \hL(z) f_i^{-\gamma} = G_{i,c} \hL(z) G''_{i,c}, \quad
  c = q^{-\gamma}.
\end{equation*}
この公式より $\hL(z)$ が
文献 \cite{BK-GC1} で導入された unipotent crystal (幾何クリスタルの一種)
の量子化とみなせることがわかる.

量子化されたパラメーター変数 $a_i$ を次のように定めておく%
\footnote{パラメーター変数 $a_i$ は単純コルートに対応している.}:
\begin{equation*}
  a_i = t_i t_{i+1}^{-1}.
\end{equation*}
このとき, 量子化される前の幾何クリスタルの場合(\cite{BK-GC1}, \cite{BK-GC2})と同様
の方法で、この量子化された場合にも, 
拡大アフィンWeyl群 $\WW(A^{(1)}_2)=\bra s_0,s_1,s_2,\pi\ket$
の作用を $t_i$, $\hb_i$ たちから $\C(q,r)$ 上生成される斜体上に
次のように定めることができる:
\begin{equation*}
 s_i(\hL(z)) = G_{i,a_i} \hL(z) G_{i,a_i}^{-1}, \quad
 \pi(\hL(z)) = \Lambda_3(z) \hL(rz) \Lambda_3(r^2z)^{-1}.
\end{equation*}
$G_{i,a_i} = {G''_{i,a_i}}^{-1} = y_i((t_i^2-t_{i+1}^2)\hb_i^{-1})$ 
($i=1,2$) が成立していることに注意せよ%
\footnote{$s_i = \pi^{i-1}s_1\pi^{-(i-1)}$ なので $s_1$ と $\pi$ の作用が
定まっていれば $\WW(A^{(1)}_2)$ の作用を得るためには十分である.
そこで以下の説明では $s_0=s_3$ の作用に関する説明を暗黙のうちに省略する
ことがある.}.
この作用を具体的に書き下すと次のようになる:
\begin{align*}
 &
 s_i(t_i) = t_{i+1}, \quad
 s_i(t_{i+1}) = t_i, \quad
 s_i(t_{i+2}) = t_{i+2},
 \\ &
 s_i(\hb_i) = \hb_i, \quad
 s_i(\hb_{i+1}) = \hb_{i+1} + (t_i^2-t_{i+1}^2)\hb_i^{-1}, \quad
 s_i(\hb_{i-1}) = \hb_{i-1} - (t_i^2-t_{i+1}^2)\hb_i^{-1},
 \\ &
 \pi(t_j) = t_{j+1}, \quad
 \pi(\hb_j) = \hb_{j+1}.
\end{align*}
準周期性 $t_{i+3}=r^{-1}t_i$, $\hb_{i+3}=r^{-1}\hb_i$ に注意せよ.


以下の結果は\secref{sec:Sato-Wilson-2}において $\tau$ 函数へのWeyl群作用の
Sato-Wilson表示について説明するために使われる.

従属変数 $\hf_i$ を次のように定める
\footnote{これは $\hf_1$, $\hf_2$ のみに通用する公式.
$\hf_3$ についてはここだけで仮に $\tc_4=r$ とおけば
同じ公式で適切な定義が得られる.}:
\begin{equation*} 
 \hf_i = -(q-q^{-1})^{-1}\tc_i f_i \tc_{i+1}^{-1}.
\end{equation*}
このとき $\hb_i = -(q-q^{-1})t_i \hf_i t_{i+1}$ が成立している.

$\tc_i$ たちは $f_j$ たちと可換なので
\begin{equation*}
 \hf_i^\gamma f_j \hf_i^{-\gamma} = f_i^\gamma f_j f_i^{-\gamma}
\end{equation*}
を満たしている.  さらに $\tc_i$ たちは互いに $q$ べき因子の違いを除いて可換である.
ゆえに $\hf_i^\beta \hf_j^{\beta+\gamma} \hf_i^\gamma$ と
$\hf_j^\gamma \hf_i^{\beta+\gamma} \hf_j^\beta$ も $q$ べき因子の違いを
除いて等しい. このことから, 
上で構成したWeyl群作用を $f_i^\gamma$ ではなく, $\hf_i^\gamma$ を
用いて構成できそうである. 
しかし, $\hf_i$ たちは $f_i$ たちと違って $t_i$ たちと可換である.
ゆえに $t_i$ たちへのWeyl群作用は $\hf_i^\gamma$ 以外の手段を用いて
構成しなければいけない. そこで $\C(q,r)$ 上 $t_i$, $\hf_i$ で生成される
斜体 $\K$ の代数自己同型 $\ts_i$ を次のように定めておく:
\begin{equation*}
 \ts_i(t_i)=t_{i+1}, \quad
 \ts_i(t_{i+1})=t_i, \quad
 \ts_i(t_{i+2})=t_{i+2}, \quad
 \ts_i(\hf_j)=\hf_j.
\end{equation*} 
このとき $\K$ へのWeyl群作用を
\begin{equation*}
  s_i(x) = \Ad(\hf_i^\gamma)(\ts_i(x)) = \hf_i^\gamma\ts_i(x)\hf_i^{-\gamma}, \quad
  c = q^{-\gamma} = a_i = t_i t_i^{-1}
\end{equation*}
によって定めることでき, この作用は上で構成したWeyl群作用と一致している:
\begin{align*}
 &
 s_i(\hL(z)) = G_i \hL(z) G_i^{-1}, \quad
 G_i := G_{i,a_i} = y_i(g_i), \quad
 g_i = \frac{t_i^2-t_{i+1}^2}{\hb_i} = \frac{[\av_i]_q}{\hf_i}.
\end{align*}
この結果は\secref{sec:Sato-Wilson-2}で基本的な役目を果たす%
\footnote{斜体の元 $a$, $b$ が互いに可換なとき, 
$ab^{-1}=b^{-1}a$ を $a/b$ や $\frac{a}{b}$ と書くことにする.}.



\subsection{変数 $x_i$, $y_i$ の量子化}
\label{sec:xy}


対角行列 $\tC'$ を次のように定める:
\begin{equation*}
 \tC' = \diag(\tc_3 b_{31}, \tc_1 b_{11}, \tc_2 b_{21}).
\end{equation*}
このとき行列 $X(z):=\tC L_1(z) \tC'^{-1}$, $Y(rz):=\tC' L_2(z) L_0 \tC^{-1}$ は
次の形をしている%
\footnote{実際にはその形になるように $\tC'$ を定めた. 
適切な $\tC$ がすでに与えられているとき, $\tC'$ はその条件で一意に定まる.}:
\begin{align*}
 &
 X(z)=\tC L_1(z) \tC'^{-1} =
 \begin{bmatrix}
  x_1 & 1   & 0 \\
   0  & x_2 & 1 \\
   z  & 0   & x_3 \\
 \end{bmatrix},
 \\ &
 Y(rz)=\tC' L_2(z)L_0 \tC^{-1} =
 \begin{bmatrix}
  y_1 & 1   & 0  \\
   0  & y_2 & 1  \\
   rz & 0  & y_3 \\
 \end{bmatrix}.
\end{align*}
この式によって
量子化された従属変数 $x_i$, $y_i$ を定義する.
それらのインデックスを準周期性 $x_{i+3}=r^{-1}x_i$, $y_{i+3}=r^{-1}y_i$ 
によって整数全体に拡張しておく.

このとき, $\ha_i = t_i^2 = x_i y_i$ および以下の関係式が成立している: 
$\mu=1,2$ について, 
\begin{alignat*}{2}
 &
 x_i y_i = y_i x_i = t_i^2, \quad
 \\ &
 x_i x_{i+\mu} = q^{(-1)^{\mu-1}2} x_{i+\mu} x_i, \qquad
 & &
 x_i y_{i+\mu} = q^{-(-1)^{\mu-1}2} y_{i+\mu} x_i,
 \\ &
 y_i y_{i+\mu} = q^{(-1)^{\mu-1}2} y_{i+\mu} y_i, \qquad
 & &
 y_i x_{i+\mu} = q^{-(-1)^{\mu-1}2} x_{i+\mu} y_i,
 \\ &
 t_i \xi_j = \xi_j t_i \quad (\xi=x,y,t).
\end{alignat*}
この関係式の古典極限は
古典版の変数 $x_i,y_i,t_i$ のPoisson構造を再現する%
\footnote{$\hbar=q^2-1$ とおいて $\hbar\to 0$ の古典極限を考える.}.
このことより, この節の行列 $X(z)$, $Y(rz)$ は古典版の $q$ 差分化された
Painlev\'e IV 方程式の記述で用いた行列 $X(z)$, $Y(rz)$ の
量子化になっていることがわかる.


$x_i$, $y_i$ たちは以下の条件から一意に決定される:
\begin{equation*}
  X(z)Y(rz) = \tC \tL(z) \tC^{-1} = \hL(z), \quad
  r = c_1 c_3 c_2, \quad
  x_3 x_2 x_1 = b_{31}^{-1}a_{31}b_{21}^{-1}a_{21}b_{11}^{-1}a_{11}.
\end{equation*}
$r$ と $x_3 x_2 x_1$ に関する条件の右辺は代数 $\B$ の可逆な中心元であることにも注意せよ.
対角行列 $\tC$ の成分は $\tL(z)=L_1(z)L_2(z)L_0$ の成分 $c_i$ たちで書けていたの
で,  $x_3 x_2 x_1$ に関する条件と $\tL(z)$ の成分から $x_i$, $y_i$ たちは一意に決まる.
さらに $L_0$ は $L(z)=L_1(z)L_2(z)$ の対角部分だったので, 
$x_3 x_2 x_1$ に関する条件と $L(z)$ の成分から $x_i$, $y_i$, $t_i$ 
たちは一意に決まる. ($\tC=\diag(\tc_1,\tc_2,\tc_3)$, $L_0=\diag(\ta_1,\ta_2,\ta_3)$ であり, 
$t_i$ は $t_i=\tc_i\ta_i\tc_i^{-1}$ と定義されたのであった.) 


代数 $\B$ は $a_{ik}^{\pm1}$, $b_{ik}^{\pm1}$ から生成される代数なのであった.
$\B$ には群 $(\C^\times)^3\times(\C^\times)^3$ の元 $(g,g')$ を
次のように作用させることができる:
\begin{equation*}
 L_1(z)\mapsto g L_1(z) g'^{-1}, \quad
 L_2(z)\mapsto g'L_2(z) g^{-1}.
\end{equation*}
ただし $(\C^\times)^3$ の元 $g$, $g'$ と可逆な対角行列を同一視した.
この群作用を local $L$-operators $L_1(z)$, $L_2(z)$ の{\bf gauge変換}と呼ぶことにする.
このguage変換で不変な $\B$ の元全体のなす部分代数を $\B^\gauge$ と
書くことにする. $\B^\gauge$ は $x_i$, $y_i$, $t_i$, $r$ とその逆元たちで
生成される.  




\subsection{$\WW(A^{(1)}_2)\times\WW(A^{(1)}_1)$ の作用の量子化}


前節の定義より, 次が成立していることがすぐにわかる($i=1,2$):
\begin{align*}
 &
 \hL(z) = \tC L_1(z)L_2(z)L_0 \tC^{-1} = X(z)Y(rz),
 \\ &
 G_{i,c} = y_i(g_i(c)), \quad
 g_i(c) := \frac{(c^{ 2}-1)t_{i+1}^2}{x_i+y_{i+1}}, 
 \\ &
 G''_{i,c} = y_i(g''_i(c)), \quad
 g''_i(c) := \frac{(c^{-2}-1)t_i^2}{x_i+y_{i+1}}.
\end{align*}
さらに行列 $G'_{i,c}$ を次のように定める:
\begin{align*}
 G'_{i,c} = y_i(g'_i(c)), \quad
 g'_i(c)
 := \frac{(c^2-1)t_{i+1}^2}{c^2 t_{i+1}^2 x_i^{-1}+x_{i+1}}
 = - \frac{(c^{-2}-1)t_i^2}{c^{-2}t_i^2 y_{i+1}^{-1}+y_i}.
\end{align*}
このとき, $\Ad(f_i^\gamma)$ の $X(z)$, $Y(rz)$ への作用は次のように書ける:
\begin{align*}
 &
 f_i^\gamma X(z) f_i^{-\gamma} = G_{i,c} X(z) {G'_{i,c}}^{-1}, \quad
 f_i^\gamma Y(rz) f_i^{-\gamma} = G'_{i,c} Y(rz) G''_{i,c}, \quad
 c = q^{-\gamma}. 
\end{align*}
ゆえに, $x_i$, $y_i$ によって $\C(r,q)$ 上生成される斜体への
拡大アフィンWeyl群 $\WW(A^{(1)}_2)=\bra s_0,s_1,s_2,\pi\ket$ の作用を
次によって定めることができる:
\begin{align*}
 &
 s_i(X(z)) = G_{i,a_i} X(z) {G'_{i,a_i}}^{-1}, \quad
 s_i(Y(z)) = G'_{i,a_i} Y(rz) G''_{i,a_i}, \quad
 \\ &
 \pi(X(z)) = \Lambda(z)X(z)\Lambda(rz)^{-1}, \quad
 \pi(Y(rz)) = \Lambda(rz)Y(rz)\Lambda(r^2z)^{-1}.
\end{align*}
以下の公式より, 以上の結果は古典版の $q$ 差分化された Painlev\'e IV 方程式
の対称性(B\"acklund変換)の部分の量子化を与えていることがわかる:
\begin{align*}
 &
 G_{i,a_i} = y_i(g_i), \quad
 g_i = \frac{x_i y_i - x_{i+1}y_{i+1}}{x_i+y_{i+1}},
 \\ &
 G'_{i,a_i} = y_i(g'_i), \quad
 g'_i = \frac{y_i x_i - y_{i+1}x_{i+1}}{y_i+x_{i+1}},
 \\ &
 G''_{i,a_i}=G_{i,a_i}^{-1}
\end{align*}
この作用はパラメーター変数 $t_i$ たちにも自然に拡張される. 

以上で構成された $\WW(A^{(1)}_2)$ の作用の量子化の具体形は以下の通り:
\begin{align*}
 &
 s_i(x_i) 
 = x_i - (y_i x_i - y_{i+1}x_{i+1})(y_i+x_{i+1})^{-1}
 = (x_i+y_{i+1})x_{i+1}(y_i+x_{i+1})^{-1},
 \\ &
 s_i(x_{i+1})
 = x_{i+1} +(x_i+y_{i+1})^{-1}(y_i x_i - y_{i+1}x_{i+1})
 =(x_i+y_{i+1})^{-1} x_i (y_i+x_{i+1}),
 \\ &
 s_i(y_i) 
 = y_i - (x_i y_i - x_{i+1}y_{i+1})(x_i+y_{i+1})^{-1}
 = (y_i+x_{i+1})y_{i+1}(x_i+y_{i+1})^{-1},
 \\ &
  s_i(y_{i+1})
 = y_{i+1} +(y_i+x_{i+1})^{-1}(x_i y_i - x_{i+1}y_{i+1})
 =(y_i+x_{i+1})^{-1} y_i (x_i+y_{i+1}),
 \\ &
 s_i(x_{i+2})=x_{i+2},
 \quad
 s_i(y_{i+2})=y_{i+2}, 
 \\ &
 s_i(t_i)=t_{i+1},  \quad
 s_i(t_{i+1})=t_i,  \quad
 s_i(t_{j+2})=t_{j+2},
 \\ &
 \pi(x_i) = x_{i+1},  \quad
 \pi(y_i) = y_{i+1},  \quad
 \pi(t_i) = t_{i+1}.
\end{align*}
これらの公式は $q$ が一ヶ所も出て来ないように書かれているので
古典の場合の対応する公式と見かけ上完全に一致している.

残りの $\WW(A^{(1)}_1)=\bra r_0,r_1,\varpi\ket$ の部分の作用の量子化も
同様に構成される.
$d_i$, $d'_i$ を次のように定めるとすべての $x_i$, $y_i$, $t_i$ と可換である:
\begin{equation*}
  d_i  = x_{i+2}x_{i+1}x_i, \quad
  d'_i = y_{i+2}y_{i+1}y_i.
\end{equation*}
量子化された $V_i(w)$, $R_i$, $\rho_i$, $Q_i$ を古典の場合の公式をそのまま
用いて
\begin{align*}
  &
  V_i(w) =
  \begin{bmatrix}
    y_i & 1   \\
    w   & x_i \\
  \end{bmatrix},
  \quad
  R_i =
  \begin{bmatrix}
    1      & 0 \\
    \rho_i & 1 \\
  \end{bmatrix},
  \quad
  \rho_i = r \frac{d_i - d'_{i+1}}{Q_i},
  \\ &
  Q_i = y_{i+2}y_{i+1} + y_{i+2}x_i + x_{i+1}x_i.
\end{align*}
と定める. このとき $x_i$, $y_i$ によって $\C(r,q)$ 上生成される斜体への
拡大アフィンWeyl群 $\WW(A^{(1)}_1)=\bra r_0,r_1,\varpi\ket$
の作用を(古典の場合と同様の)次の式によって定めることができる:
\begin{equation*}
  r_1(V_i(z)) = R_{i+1}^{-1} V_i(z) R_i, \qquad
  \varpi(V_i(z)) = \Lambda_2(w)V_i(w)\Lambda_2(w)^{-1}.
\end{equation*}
この作用はパラメーター変数 $t_i$ にも自然に拡張される.

以上で構成された $\WW(A^{(1)}_1)$ の作用の量子化の具体形は以下の通り:
\begin{align*}
 &
 r_1(x_i) 
 = x_i - r Q_{i+1}^{-1}(x_{i+3}x_{i+2}x_{i+1}-y_{i+4}y_{i+3}y_{i+2})
 = r^{-1} Q_{i+1}^{-1}y_i Q_i,
 \\ &
 r_1(y_i)
 = y_i + r(x_{i+2}x_{i+1}x_i-y_{i+3}y_{i+2}y_{i+1})Q_i^{-1}
 = r Q_{i+1} x_i Q_i^{-1},
 \\ &
 r_1(t_i) = t_i,  \quad
 \varpi(x_i)=y_i,   \quad
 \varpi(y_i)=x_i,   \quad
 \varpi(t_i)=t_i.
\end{align*}
これらの公式も
古典の場合の対応する公式と見かけ上完全に一致している.

さらに, 以下の条件で以上で定義した2つの拡大アフィンWeyl群の $x_i$, $y_i$ への作用を
特徴付けることもできる:
\begin{align*}
 &
 r_1(X(z)Y(rz)) = X(z)Y(rz), 
 \\ &
 r_1(x_{i+2}x_{i+1}x_i) = y_{i+3}y_{i+2}y_{i+1}, \quad
 r_1(y_{i+3}y_{i+2}y_{i+1}) = x_{i+2}x_{i+1}x_i,
 \\ &
 \varpi(X(z)) = Y(z), \quad
 \varpi(Y(z)) = X(z),
 \\[\medskipamount] &
 s_i(V_{i+1}(w)V_i(w)) = V_{i+1}(w)V_i(w), \quad
 s_i(V_{i+2}(w))=V_{i+2}(w),
 \\ &
 s_i(x_i y_i) = x_{i+1}y_{i+1}, \quad 
 s_i(x_{i+1}y_{i+1})= x_i y_i,
 \\ &
 \pi(V_i(w)) = V_{i+1}(w).
\end{align*}
このことから2つの拡大アフィンWeyl群の作用が可換になることもわかる.
したがって, 以上によって, $\WW(A^{(1)}_2)\times\WW(A^{(1)}_1)$ の作用の量子化
が構成できたことになる.




\subsection{\qP{IV} の量子化}
\label{sec:qqPIV}

古典の場合の $q$ 差分版 Painlev\'e IV 方程式の離散時間発展は
$\WW(A^{(1)}_2)\times\WW(A^{(1)}_1)$ の作用における
$\WW(A^{(1)}_1)=\bra r_0,r_1, \varpi\ket$ の側の $U_1=r_1\varpi$ の作用に
一致しているのであった. 

量子化された $U_1=r_1\varpi$ の作用の仕方は次の通り:
\begin{align*}
 &
 U_1(x_i) = r Q_{i+1} x_i Q_i^{-1},
 \qquad
 U_1(y_i) = r^{-1} Q_{i+1}^{-1}y_i Q_i,
 \qquad
 U_1(t_i) = t_i.%,
% \\[\medskipamount] &
% Q'_i := \varpi(Q_i) = x_{i+2}x_{i+1} + x_{i+2}y_i + y_{i+1}y_i, 
% \\ &
% U_1^{-1}(x_i) = r^{-1} {Q'_{i+1}}^{-1}x_i Q'_i,
% \qquad
% U_1^{-1}(y_i) = r Q'_{i+1} y_i {Q'_i}^{-1},
% \qquad
% U_1^{-1}(t_i) = t_i.
\end{align*}
$Q_i$ は $Q_i=y_{i+2}y_{i+1} + y_{i+2}x_i + x_{i+1}x_i$ と定義されたのであった.


量子化されたパラメーター変数 $a_i$ は
古典の場合と同様に $a_i=t_i t_{i+1}^{-1}$ と定義されたのであった($t_i^2=x_i y_i=y_i x_i$).
量子化された従属変数 $F_i$ も古典の場合と同様に次のように定義しよう:
\begin{equation*}
  F_i = t_i^{-1} t_{i+1}^{-1} x_{i+1} x_i.
\end{equation*}
このとき, $F_{i+3}=F_i$, $a_{i+3}=a_i$ が成立しており, 
それらは以下の関係式を満たしている:
\begin{equation*}
  F_i F_j = q^{2 b_{ij}} F_j F_i,  \quad
 a_i a_j = a_j a_i, \quad 
 a_i F_j = F_j a_i.
\end{equation*}
たとえば $F_i F_{i+1}=q^2 F_{i+1} F_i$.
この関係式の古典極限は古典の場合の Poisson 構造を再現する.


$F_i$ の定義と $t_i^2= x_i y_i = y_i x_i$ と $x_i x_{i+1}=q^2 x_{i+1}x_i$ などより,
\begin{equation*}
  x_{i+1} x_i = t_i t_{i+1} F_i, \quad
  y_{i+2} x_i = t_i t_{i+2} F_i F_{i+1}^{-1} ,\quad
  y_{i+2} y_{i+1} = q^{-2} t_{i+1} t_{i+2} F_{i+1}^{-1}.
\end{equation*}	
ゆえに
\begin{align*}
  Q_i 
  & = q^{-2} t_{i+1} t_{i+2} F_{i+1}^{-1}
  + t_i t_{i+2} F_i F_{i+1}^{-1}
  + t_i t_{i+1} F_i
  \\
  & = ( 1 + q^2 a_i F_i + q^2 a_i a_{i+1} F_i F_{i+1} ) 
      q^{-2}t_{i+1}t_{i+2}F_{i+1}^{-1}
  \\
  Q_{i+1}
  & = ( 1 + q^2 a_{i+1} F_{i+1} + q^2 a_{i+1} a_{i-1} F_{i+1} F_{i-1} )
      q^{-2}r^{-1}t_{i+2}t_iF_{i-1}^{-1}
  \\
  Q_{i+2}
  & = (1 + q^2 a_{i-1}F_{i-1} + q^2 a_{i-1} a_i F_{i-1} F_i)
      q^{-2}r^{-2}t_it_{i+1}F_i^{-1}.
\end{align*}
これより, 特に
\begin{align*}
 &
 U_1(F_i)
 = r^2 Q_{i+2} F_i Q_i^{-1}
 \\
 &= (1 + q^2 a_{i-1}F_{i-1} + q^2 a_{i-1} a_i F_{i-1} F_i)
    a_i a_{i+1} F_{i+1}
    (1 + q^2 a_i F_i + q^2 a_i a_{i+1} F_i F_{i+1})^{-1}
\end{align*}
これが $q$ 差分版 Painlev\'e IV 方程式の{\bf 量子化された離散時間発展}である.

$s_i$, $\pi$ の作用は $r_i$, $\varpi$ の作用と可換だったので, 
$s_i$, $\pi$ の作用は $U_1$ の作用と可換である.
だから $s_i$, $\pi$ の作用は量子化された $q$ 差分版 Painlev\'e IV 方程式の
対称性(B\"acklund変換)になっている.
$s_i$, $\pi$ の作用は $F_i$ と $a_i$ たちで生成される斜体を保つ.
作用の具体的形は以下の通り:
\begin{align*}
 &
 s_i(F_i)=F_i, \quad
 s_i(F_{i-1})=F_{i-1}\frac{a_i+F_i}{1+a_i F_i}, \quad
 s_i(F_{i+1})=\frac{1+a_i F_i}{a_i+F_i}F_{i+1}, 
 \\ &
 s_i(a_i) = a_i^{-1}, \quad
 s_i(a_{i\pm1}) = a_ia_{i\pm1},
 \\ &
 \pi(F_i)=F_{i+1}, \quad \pi(a_i)=a_{i+1}.
\end{align*}
この形のWeyl群双有理作用の量子化は任意の対称化可能一般Cartan行列の場合に
一般化される(\cite{Hasegawa-QB}, \cite{Kuroki-W}).


\subsection{まとめ}

以上によって $q$ 差分版 Painlev\'e IV 方程式の離散時間発展とその $A^{(1)}_2$ 対称性が
量子化された.  量子化の過程で, $q$ 差分版 Painlev\'e IV 方程式の対称形式の
独立変数 $F_i$, $x_i$, $y_i$ たちとパラメーター変数 $a_i$, $t_i$ たちが量子群のどこに
どのように住んでいるかもわかった.

量子化された従属変数 $F_i$ とパラメーター変数 $a_i$ の積 $a_i F_i = y_{i+1}^{-1}x_i$ 
は $q$ のべき因子を除いて $\varphi_i\otimes\kappa_i^{-1}\kappa_{i+1}$ の像 $f_{i1}$ 
と $1\otimes \varphi_i$ の像 $f_{i2}$ の比 $f_{i2}^{-1}f_{i1}$ に等しい. 
すなわち $a_i F_i$ は $q$ のべき因子を除いて量子展開環の
下三角部分の Chevalley 生成元 $\varphi_i$ の余積の2つの項の比の像に等しい.

量子化されたパラメーター変数 $t_i$, $a_i$ のそれぞれは $q$ べき因子の違いを除いて
量子展開環の Cartan 部分代数の元 $\kappa_i$, $\kappa_i^{-1}\kappa_{i+1}$ の
余積の像に等しい.

要するに $q$ 差分化された Painlev\'e IV 方程式の対称形式の従属変数は
量子展開環の下三角部分のChevalley生成元から来ており,
パラメーター変数は量子展開環のCartan部分代数の生成元から来ている.

$q$ 差分化された Painlev\'e IV 方程式の量子化の対称性 $s_i$ の作用
は量子展開環の下三角部分のChevalley生成元の余積の像 $f_i$ のべき $f_i^\gamma$
の作用から来ている. ただし, $\gamma$ とパラメーター変数の関係
を $a_i=t_i/t_{i+1}=q^{-\gamma}$ としなければいけない.

$x_i$, $y_i$, $t_i$ 変数は local $L$-operators $L_1(z)$, $L_2(z)$ の可逆な
対角行列によるguage変換で不変な元であり, それらの逆元と $r^{\pm1}$ から
guage変換で不変な $\B$ の元全体のなす部分代数 $\B^\gauge$ が生成される.



%%%%%%%%%%%%%%%%%%%%%%%%%%%%%%%%%%%%%%%%%%%%%%%%%%%%%%%%%%%%%%%%%%%%%%%


\section{$\tau$ 函数の量子化}

前節では $q$ 差分版 Painlev\'e IV 方程式を例に Painlev\'e 系の量子化を
量子群を使って遂行する方法について説明した.
しかし, そこで量子化されたのは従属変数 $F_i$, $x_i$, $y_i$ たちと
パラメーター変数 $a_i$, $t_i$ たちだけであり, 
$\tau$ 函数は一切登場しなかった.

この節では $\tau$ 函数の量子化を扱うことにする.
より正確に言えば表現論と相性の良い量子化された {\bf $\tau$ 変数}を
導入し, その基本性質を調べる.

この文書で「量子化」は「Poisson 括弧を非可換環の交換子で置き換えること
およびその一般化」を意味しているのであった(\secref{sec:quantization}).
Painlev\'e 系の解に付随する $\tau$ 函数を考えてしまうと, 
その意味での量子化を遂行できない.
量子化できるのは Poisson 構造が定められた変数たちの方である.
だから, この文書のタイトルは「$\tau$ 函数の量子化」ではなく
「$\tau$ 変数の量子化」とするべきだったかもしれない.

Painlev\'e 系の従属変数とパラメーター変数は Painlev\'e 系の
舞台となる Poisson 多様体の座標変数であると考えられる.
ゆえにそこに新たに $\tau$ 変数を付け加えるということは,
Painlev\'e 系の舞台となる Poisson 多様体を拡張することを
意味している. 

$\tau$ 変数の量子化のためには $\tau$ 変数と従属変数やパラメーター変数の
あいだのPoisson括弧が適切に定義されていなければいけない.
しかし, 非常に残念なことに古典 Painlev\'e 系の論文で $\tau$ と他の変数のあいだ
に Poisson 括弧を導入しているものを見付けることができなかった.
だから, Poisson 括弧の情報無しに適切な量子 $\tau$ 変数の定義を
見付けなければいけない.

現時点でわかっていることは以下の通りである.



\subsection{量子 $\tau$ 変数の導入}

$q$ 差分版 Painlev\'e IV 方程式の量子化において
出発点になる基礎となる代数は
量子展開環 $U_q(\glhat_3)$ の下Borel部分代数 $U_q(\bor_-)$ だった. 
以下ではこの場合を例に量子化された $\tau$ 変数を導入しよう.
ただし, 可能な限り, 任意の対称化可能一般Cartan行列に付随する場合への一般化
の仕方が分かるように説明したい.

$\F=\C(q)$ とおき, 量子展開環などは $\F$ 上の代数として
定義されていると考えることにする.
以下で主に使うのは $U_q(\glhat_3)$ の下Borel部分代数全体ではなく, 
下三角部分 $U_q(\nil_-)=\bra\varphi_0,\varphi_1,\varphi_2\ket$
の方である.

まず, $\A$ は$U_q(\nil_-)$ のある剰余環を部分環として
含むOre整域であると仮定し, $Q(\A)$ はその分数斜体であるとする:
\begin{equation*}
 Q(\A) 
 = \{\, ab^{-1} \mid a,b\in\A,\ b\ne0 \,\}
 = \{\, b^{-1}a \mid a,b\in\A,\ b\ne0 \,\}.
\end{equation*}
たとえば\secref{sec:f^gamma}の $\B$ はそのような $\A$ の例になっている.
一般にアフィン型もしくは有限型の量子展開環の部分環の剰余環で整域に
なっているものはすべてOre整域になっている%
\footnote{我々が扱っているWeyl群作用は双有理作用なので, 変数に作用した結果が
変数の多項式ではなく, 変数の有理函数になってしまう. だから, その量子化を扱うためには
非可換環における分数の構成が必要になる. 非可換における分数の構成は特にOre整域に
おいて非常にうまく行くことが知られている.
Ore整域に関する解説を \cite{Kuroki-localization}, \cite{Kuroki-Ore} に書いておいた.
非可換分数の理論に不慣れな読者はそれらを参照して欲しい.}.

次に, 記号 $\ev_1,\ev_2,\ev_3,\dv$ で張られる自由 $\Z$ 加群 $\Qv$ を考える%
\footnote{$\dv$ はアフィンLie環の中心元 ``$c$'' に対応している.}.
$\Qv$ を{\bf コルート格子}と呼ぶことにする.
準周期性%
\footnote{この準周期性はパラメーター変数の準周期性 $t_{i+3}=r^{-1}t_i$ に対応している.
$\dv$ はアフィンLie環の中心元``c''に対応していたので, 準周期性は``レベル''の情報を
持っていると考えることもできる.}
\begin{equation*}
  \ev_{i+3} = \ev_i - \dv
\end{equation*}
によってインデックスを整数全体に拡張しておく.
$\Qv$ の群環の基底を $q^\gamma$ ($\gamma\in\Qv$) と書き, 
$\Qv$ の群環を $\F[q^\Qv]$ と書くことにする.
($\F[q^\Qv]$ はLaurent多項式環である.)

$\ev_i,\dv$ そのものおよび $q^{\ev_i},q^{\dv}$ を{\bf パラメーター変数}
と呼ぶことにする.

以下, $\K$ は $\A$ の分数斜体 $Q(\A)$ の部分斜体であると仮定する.

$\K$ と $\F[q^\Qv]$ のテンソル積代数を $\K[q^\Qv]=\K\otimes\F[q^Qv]$ と表わす.
$\K[q^\Qv]$ は $\K$ にパラメーター変数を添加して得られる代数である.
$\K[q^\Qv]$ の中でパラメーター変数 $q^\gamma$ ($\gamma\in\Qv$) はすべての元と可換である.
$\K[q^\Qv]$ もOre整域になるので, その分数斜体を $\K(q^\Qv)$ と表わす.

{\bf 単純コルートに対応するパラメーター変数} $\av_i$ を次のように定める:
\begin{equation*}
 \av_i = \ev_i-\ev_{i+1}.
\end{equation*}
たとえば, $\theta^\vee=\av_1+\av_2=\ev_1-\ev_3$ と
おくと $\av_0=\dv-\theta^\vee$ であり, 
周期性 $\av_{i+3}=\av_i$ が成立している.


コルート格子 $\Qv$ の双対格子を $P$ と書き, {\bf ウェイト格子}と呼び,  
$\ev_1,\ev_2,\ev_3,\dv$ の双対基底を $\eps_1,\eps_2,\eps_3,\Lambda_0$ と書く:
\begin{align*}
 &
 \bra\ev_i,\eps_j\ket=\delta_{ij}, \quad
 \bra\ev_i,\Lambda_0\ket=0, \quad
 \bra\dv,\eps_j\ket=0, \quad
 \bra\dv,\Lambda_0\ket=1.
\end{align*}
$\eps_i,\alpha_i, \Lambda_i\in X$ ($i\in\Z$) を次の条件によって定める:
\begin{align*}
 &
 \eps_{i+3}=\eps_i, \quad
 \Lambda_i = \Lambda_{i-1} + \eps_i,
 \\ &
 \alpha_i = \eps_i - \eps_{i+1} = 2\Lambda_i-\Lambda_{i-1}-\Lambda_{i+1}.
\end{align*}
このとき, 次が成立している:
\begin{align*}
 &
 \bra\av_i,\alpha_j\ket = a_{ij} \quad (i,j=0,1,2),
 \\ &
 \bra\av_i,\Lambda_j\ket = \delta_{ij} \quad (i,j=0,1,2),
 \qquad
 \bra\av_i,\Lambda_{j+3}\ket=\bra\av_i,\Lambda_j\ket \quad (i,j\in\Z).
\end{align*}
ここで $[a_{ij}]_{i,j=0}^2$ は $A^{(1)}_2$ 型の一般Cartan行列である.
そこで $\alpha_i$ を{\bf 単純ルート}と呼び, 
$\Lambda_i$ を{\bf 基本ウェイト}と呼び, 
$P$ の元を{\bf 整ウェイト}と呼ぶことにする.

任意の $\mu\in P$ に対して, 
パラメーター変数に作用する差分作用素 $\tau^\mu$ を
次のように定める:
\begin{equation*}
  \tau^\mu(\gamma) = \gamma + \bra\gamma,\mu\ket
  \quad (\gamma\in\Qv).
\end{equation*}
この作用は $q^\gamma$ のタイプのパラメーター変数に自然に拡張される:
\begin{equation*}
  \tau^\mu(q^\gamma) = q^{\bra\gamma,\mu\ket}q^\gamma
  \quad (\gamma\in\Qv).
\end{equation*}
そこで次の関係式によって $\K(q^\Qv)$ に $\tau^\mu$ ($\mu\in P$)
を添加してできる $\K(q^\Qv)$ に作用する $q$ 差分作用素環 $\K(q^\Qv)[\tau^P]$ が
定義される:
\begin{equation*}
 \tau^\lambda a = a \tau^\lambda, \quad
 \tau^\lambda q^\gamma = q^{\bra\gamma,\lambda\ket} q^\gamma \tau^\lambda, \quad
 \tau^\lambda \tau^\mu = \tau^{\lambda+\mu} \quad
 (a\in\K,\ \gamma\in\Qv,\ \lambda,\mu\in P).
\end{equation*}

実はこの差分作用素の意味で導入された記号 $\tau^\lambda$ たちが
我々が必要としている{\bf 量子化された $\tau$ 変数}である.
量子 $\tau$ 変数はパラメーター変数を動かす差分作用素なので, 
量子化された場合にパラメーター変数たちは $\tau$ 変数たちと非可換になる. 

特に基本的なのは基本ウェイトに対応する差分作用素 $\tau^{\Lambda_i}$ たち
である.  そこでそれらを $\tau_i$ と書き, 
{\bf 基本 $\tau$ 変数}と呼ぶことにする:
\begin{equation*}
  \tau_i := \tau^{\Lambda_i}  \qquad
 (\text{基本 $\tau$ 変数}).
\end{equation*}
基本 $\tau$ 変数 $\tau_i$ とは, 単純コルートに対応する
パラメーター変数 $\av_i$ の正準共役演算子 $\partial/\partial\av_i$ の
指数函数 $\exp(\partial/\partial\av_i)$ のことだと考えることもできる.
これが ``$\tau_i$'' の正体である!




\subsection{量子 $\tau$ 変数へのWeyl群作用}
\label{sec:tau-Weyl}

コルート格子 $\Qv$ とウェイト格子 $P$ には次によってWeyl群作用が定まる:
\begin{equation*}
 s_i(\gamma) = \gamma - \bra\gamma,\alpha_i\ket\av_i, \quad
 s_i(\lambda)=\lambda - \bra\av_i,\lambda\ket\alpha_i \quad
 (\gamma\in\Qv,\ \lambda\in P).
\end{equation*}
たとえば, $i=0,1,2$ について, 
\begin{align*}
 &
 s_i(\Lambda_i) = \Lambda_i-\alpha_i = \Lambda_{i-1}+\Lambda_{i+1}-\Lambda_i,
% \\ &
 \quad
 s_i(\Lambda_j) = \Lambda_j \quad (i\ne j).
\end{align*}
この作用を $\K$ の元を固定するように $\K(q^\Qv)[\tau^P]$ に拡張したもの
を $\ts_i$ と書く. すなわち, 以下の条件によって $\K(q^\Qv)[\tau^P]$ の
代数自己同型 $\ts_i$ を定める:
\begin{equation*}
 \ts_i(q^\gamma)=q^{s_i(\gamma)}, \quad
 \ts_i(\tau^\lambda)=\tau^{s_i(\lambda)}, \quad
 \ts_i(a) = a \quad
 (\gamma\in\Qv,\ \lambda\in P,\ a\in\K).
\end{equation*}
後でPainlev\'e系の対称性としての $\K(q^\Qv)[\tau^P]$ への
作用を $s_i$ と書くことになる.
それとは区別するために $\ts_i$ という記号を使った.

代数 $\A$ における量子展開環の下三角部分のChevalley生成元 $\varphi_i$ の
像を $f_i$ と書くことにする.
そして, 簡単のため, すべての $i$ について $f_i\ne 0$ であると仮定する%
\footnote{$f_i=0$ となる場合にはWeyl群作用を構成するときに $f_i^{\av_i}$ を
$1$ に置き換えればよい.  
$f_i^{\av_i}$ を含むVerma関係式の $f_i^{\av_i}$ を $1$ で置き換えて得られる
関係式は自明に成立している. 相空間の次元が低い Painlev\'e 系の対称性を
理解するためにはこのアイデアが必須である.}.

さらに, $Q(\A)$ の部分斜体 $\K$ は 
\begin{equation*}
  f_i^\gamma \K(q^\Qv) f_i^{-\gamma} \subset \K(q^\Qv)
  \quad (\gamma\in\Qv)
\end{equation*}
を満たしていると仮定する. 
たとえば, Ore整域 $\A$ が $U_q(\nil_-)$ の剰余環になっているとき, 
その分数斜体 $Q(\A)$ そのものを $\K$ とするとその条件が満たされている.
$\A$ が\secref{sec:f^gamma}の代数 $\B$ のとき, 
以下の斜体はその条件を満たしている:
\begin{itemize}
\item \secref{sec:f^gamma}の $\ta_i$, $b_i$, $c_i$ たちから生成される部分斜体,
\item \secref{sec:fhat}の $t_i$, $\hf_i$, たちから生成される部分斜体,
\item \secref{sec:xy}の $t_i$, $x_i$, $y_i$ たちから生成される部分斜体,
\item \secref{sec:qqPIV}の $a_i$, $F_i$ たちから生成される部分斜体.
\end{itemize}

Verma関係式より, $a_{ij}=a_{ji}=-1$ ならば 
\begin{equation*}
   f_i^{\av_i} f_j^{\av_i+\av_j} f_i^{\av_j}
 = f_j^{\av_j} f_i^{\av_i+\av_j} f_j^{\av_i}
\end{equation*}
が成立している(\secref{sec:powers2}). 
作用素として $\ts_i f^\gamma = f^{s_i(\gamma)} \ts_i$ ($f\in\A$, $\gamma\in\Qv$) 
が成立することを使うと,
\begin{equation*}
   f_i^{\av_i}\ts_i f_j^{\av_j}\ts_j f_i^{\av_i}\ts_i
 = f_j^{\av_j}\ts_j f_i^{\av_i}\ts_i f_j^{\av_j}\ts_j
\end{equation*}
が成立することがわかる. このことから, 斜体 $\K(q^\Qv)$ に
\begin{equation*}
  s_i(x) = \Ad(f_i^{\av_i})(\ts_i(x)) = f_i^{\av_i} \ts_i(x) f_i^{-\av_i}
  \quad (x\in\K(q^\Qv))
\end{equation*}
によってWeyl群の代数自己同型作用を定められることがわかる.

ここで, すべての $f_i$ が $\K$ に含まれていると仮定する.

このとき $s_i=\Ad(f_i^{\av_i})\ts_i$ の作用
は $q$ 差分作用素環 $\K(q^\Qv)[\tau^P]$ に拡張される.
公式 $\tau^\lambda f_i^{\gamma} = f_i^{\gamma+\bra\gamma,\lambda\ket}\tau^\lambda$ 
($\gamma\in\Qv$) などを使うと,
次が成立していることがわかる:
\begin{equation*}
 s_i(\tau^\lambda) = f_i^{\bra\av_i,\lambda\ket}\tau^{s_i(\lambda)},
\end{equation*}
特に基本 $\tau$ 変数への作用は以下のようになる:
\begin{equation*}
 s_i(\tau_i)=f_i\tau^{\Lambda_i-\alpha_i}
 = f_i \frac{\tau_{i-1}\tau_{i+1}}{\tau_i},
 \qquad 
 s_i(\tau_j)=\tau_j \quad (i,j=0,1,2;\ i\ne j).
\end{equation*}
たとえば, 整域 $\A$ が $U_q(\nil_-)$ の剰余環で $\K=Q(\A)$ のとき, 
以下の公式が成立していることがわかる:
\begin{align*}
 &
 s_i(q^\gamma) = q^{s_i(\gamma)},
 \quad
 s_i(\tau^\lambda) = f_i^{\bra\av_i,\lambda\ket}\tau^{s_i(\lambda)},
 \\ &
 s_i(f_j) = 
 \begin{cases}
  f_j & \qquad (i=j \ \text{or}\ a_{ij}=0), \\
  q^{\pm\av_i}f_j + [\av_i]_q(f_if_j-q^{\pm1}f_jf_i)f_i^{-1} & \qquad (a_{ij}=a_{ji}=-1).
 \end{cases}
\end{align*}
これらの公式(およびその対称化可能一般Cartan行列の場合への一般化)は
文献 \cite{NY-BWA} で構成されたWeyl群双有理作用の量子化かつ $q$ 差分化
になっている.




\subsection{Weyl群作用のLax-Sato-Wilson表示 (1)}
\label{sec:Sato-Wilson-1}

この節では $\A$ は\secref{sec:f^gamma}の $\B$ であるとし, 
$\K$ は
\begin{equation*}
  f_i:=(q^{-1}-q)^{-1}\ta_i^{-1}b_i\ta_{i+1}^{-1}, \quad
  d_i:=(q^{-1}-q)^{-1}\ta_i^{-1}c_i\ta_{i+2}^{-1}
\end{equation*}
から生成される分数斜体 $Q(\B)$ の部分斜体であると仮定する.
このとき, $f_i$, $d_i$ たちは
\begin{align*}
 &
 f_i f_{i+1} - q f_{i+1}f_i = q d_i,
 \\ &
 f_id_{i-1}=q d_{i-1}f_i, \quad
 f_id_i=q^{-1} d_i f_i, \quad
 f_id_{i+1}=d_{i+1}f_i,
 \\ &
 d_i d_{i+2} = q d_{i+2} d_i, \quad
 d_i d_{i-2} = q^{-1} d_{i-2} d_i
\end{align*}
を満たしている%
\footnote{$m=4$ の場合には例外的に $d_i$ どうしは可換になる.
$m\geqq4$ の場合には $f_id_{i-2}=q^{-1}d_{i-2}f_i$, $f_id_{i+1}=qd_{i+1}f_i$ となり, 
$m=3$ の場合には例外的に $f_i$ と $d_{i-2}=d_{i+1}$ が可換になる.}. %
インデックスの $3$ 周期性に注意せよ.
さらに\secref{sec:tau-Weyl}の記号のもとで $t_i = q^{-\ev_i}$, $r=q^{-\dv}$ とおき, 
$t=(t_1,t_2,t_3)$ とおく. このとき準周期性 $t_{i+3}=r^{-1}t_i$
が成立している. 
$z$ を $rz$ に移す差分作用素 $r^{z\partial/\partial z}$ を $T_{z,r}$ と書く:
\begin{equation*}
  T_{z,r}f(z) = r^{z\partial/\partial z}f(z) = f(rz).
\end{equation*}
行列 $D(t)$, $\tM(z)$ と行列値差分作用素 $M(z)$ を次のように定める: 
\begin{align*}
  &
  D(t) = \diag(t_1,t_2,t_3), 
  \\ &
  \tM(z) = L_0^{-1}\tL(z)L_0^{-1} = L_0^{-1}L_1(z)L_2(z)
  \\ &
  \phantom{\tM(z)}=
  \begin{bmatrix}
    1               & (q^{-1}-q) f_1 & (q^{-1}-q))d_1 \\
    z (q^{-1}-q)d_2 & 1              & (q^{-1}-q)f_2  \\
    z (q^{-1}-q)f_3 & z(q^{-1}-q)d_3 & 1              \\
  \end{bmatrix},
  \\ &
  M(z)
  = D(t)\tM(r^{-1}z)D(t)T_{z,r}^2
  \\ &
  \phantom{M(z)}=
  \begin{bmatrix}
    t_1^2     &   (q^{-1}-q)t_1f_1t_2 & (q^{-1}-q)t_1d_1t_3 \\
    z (q^{-1}-q)t_2d_2t_4 &   t_2^2     & (q^{-1}-q)t_2d_2t_3 \\
    z (q^{-1}-q)t_3f_3t_4 & z (q^{-1}-q)t_3d_3t_5 & t_3^2 \\
  \end{bmatrix}
  T_{z,r}^2.
\end{align*}
行列値差分作用素 $M(z)$ を $D(t)^2T_{z,r}^2$ に``対角化''しよう.
$\K[[z]]$ の元を成分に持つ $3\times 3$ 行列 $U(z)=[u_{ij}(z)]$ で, 
$U(0)$ が対角線が $1$ の上三角行列になり, 
\begin{equation*}
  M(z) = U(z)D(t)^2T_{z,r}^2U(z)^{-1}
\end{equation*}
を満たすものが唯一存在する.
$g_i\in\K$ と行列 $G_i$ を次のように定める:
\begin{equation*}
 g_i 
 = \frac{t_i^2-t_{i+1}^2}{(q^{-1}-q)t_if_it_{i+1}} 
 = \frac{[\av_i]_q}{f_i}, \qquad
 G_i = y_i(g_i).
\end{equation*}
このとき, $u_{i,i+1}(0) =  -g_i^{-1}$ でかつ, 
\secref{sec:tau-Weyl}の方法で定めたWeyl群作用について,
\begin{align*}
 &
 s_i(M(z)) = G_i M(z) G_i^{-1},
 \\ &
 \pi(M(z)) = (\Lambda_3(z)T_{z,r})M(z)(\Lambda_3(z)T_{z,r})^{-1}
\end{align*}
が成立している.  これをWeyl群作用の{\bf Lax表示}と呼ぶ.

$z_i=\tau^{\ev_i}$ とおき, 対角行列 $D_Z$ と行列 $Z(z)$ を
\begin{equation*}
  D_Z=\diag(z_1,z_2,z_3), \quad Z(z) = U(z)D_Z
\end{equation*} 
と定める. 行列 $S^g_i$, $S_i$ を次のように定める:
\begin{align*}
 &
 S^g_1 =
 \begin{bmatrix}
   0    & g_1^{-1} & 0 \\
   -g_1 & 0        & 0 \\
   0    & 0        & 1 \\
 \end{bmatrix},
 \quad
 S^g_2 =
 \begin{bmatrix}
   1 & 0 & 0 \\
   0 & 0 & g_2^{-1} \\
   0 & -g_2 & 0 \\
 \end{bmatrix},
 \\ &
 S_1 =
 \begin{bmatrix}
   0           & -[\av_1+1]_q & 0 \\
   [\av_1-1]_q & 0        & 0 \\
   0           & 0        & 1 \\
 \end{bmatrix},
 \quad
 S_2 =
 \begin{bmatrix}
   1 & 0           & 0 \\
   0 & 0           & -[\av_2+1]_q \\
   0 & [\av_2-1]_q & 0 \\
 \end{bmatrix}.
\end{align*}
このとき以下が成立している:
\begin{align*}
 &
 M(z)
 = Z(z)D((qt)T_{z,r})^2 Z(z)^{-1}
 = Z(z)D(qt)^2Z(r^2z)^{-1}T_{z,r}^2,
 \\[\medskipamount] &
 s_i(U(z)) = G_i U(z) S^g_i, 
 \\ &
 s_i(D_Z) = (S^g_i)^{-1} D_Z S_i, \
 \\ &
 s_i(D(t)T_{z,r}) = S_i^{-1} D(t)T_{z,r} S_i
 \\ &
 s_i(Z(z)) = G_i Z(z) S_i.
\end{align*}
$\pi$ の作用はどれも次の形になっている:
\begin{equation*}
  \pi(A(z)) = (\Lambda_3(z)T_{z,r})A(z)(\Lambda_3(z)T_{z,r})^{-1} \qquad
  (A(z)=U(z),D_Z,D(t)T_{z,r},Z(z)).
\end{equation*}
これをWeyl群作用の{\bf Sato-Wilson表示}と呼ぶ.



\subsection{基本 $\tau$ 変数へのWeyl群作用の結果の正則性}

前々節の構成が対称化可能一般Cartan行列に付随する場合にまで一般化され, 
次の定理が成立している.

\begin{theorem}[\cite{Kuroki-Tau}]
 基本 $\tau$ 変数へのWeyl群作用の結果は $f_i$, $q^{\pm\av_i}$ たちについて
 多項式になる.
 \qed
\end{theorem}

整ウェイト $\mu\in X$ が任意の $i$ について $\bra\av_i,\lambda\ket\geqq 0$
を満たしているとき, $\mu$ を{\bf ドミナント整ウェイト}と呼ぶ.
ドミナント整ウェイト全体の集合を $P_+$ と書く.
Weyl群を $W$ と書く.

ドミナント整ウェイト $\mu$ とWeyl群の元 $w$ によって $w(\mu)$ と
表わされる整ウェイト全体の集合を $WP_+$ と書く.
$\nu\in WP_+$ を $\nu=w(\mu)$, $w\in W$, $\mu\in P_+$ と書く
とき $w(\tau^\mu)$ は $\nu$ だけから決まり, $w$ と $\mu$ の取り方によらない.
そこで $\nu=w(\mu)\in WP_+$ に対して, $\tau(\nu) = w(\tau^\mu)$ とおく.

実際に証明できているのは次の結果である:
$\A$ が $U_q(\nil_-)$ の像に一致し, $\K=Q(\A)$ のとき,
すべての $f_i$ が $0$ でないならば,  
任意の $\nu\in WP_+$ に対して
\begin{equation*}
  \tau(\nu)=w(\tau^\mu) \in \A[q^\Qv]\tau^\nu
  \qquad(\nu=w(\mu),\ w\in W,\ \mu\in P_+).
\end{equation*}
証明には表現論におけるBGG圏 $\cO$ に関する理論を使う(\cite{Kuroki-Tau}).
証明の概略を以下で説明しよう.

以下, $\mu,\lambda\in P_+$ であるとする.

$w=s_{i_N}\cdots s_{i_2}s_{i_1}$ はWeyl群の元 $w$ の簡約表示であるとする.
このとき
\begin{align*}
 &
 \tw=\ts_{i_N}\cdots\ts_{i_2}\ts_{i_1}, \quad
 \gamma_k = s_{i_1}\cdots s_{i_{k-1}}(\av_{i_k}),
 \\ &
 X = f_{i_1}^{\gamma_1}\cdots f_{i_N}^{\gamma_N}, \quad
 Y = f_{i_1}^{\gamma_1+\bra\gamma_1,\mu\ket}
    \cdots f_{i_N}^{\gamma_N+\bra\gamma_N,\mu\ket}
\end{align*}
とおくと, $\tau(w(\mu))=w(\tau^\mu)$ は次のように表わされる:
\begin{equation*}
 w(\tau^\mu) = \tw\left(X^{-1}Y\right)\tau^{w(\mu)}.
\end{equation*}
ゆえに $Y$ が $X$ で左から割り切れること $Y\in X\A[q^\Qv]$ を示せばよい. 
そのためには, $X^{-1}Y\in\K(q^\Qv)$ なので, 
$\av_i$ たちに非負の整数を代入した場合に割り切れることを示せば十分である.
そしてそのためには $\A=U_q(\nil_-)$, $f_i=\varphi_i$ の場合に
割り切れることを示せば十分なので以下ではそのように仮定する. 

$\rho$ は $\bra\av_i,\rho\ket=1$ をみたすウェイトであるとし, 
Weyl群の元 $w$ とウェイト $\lambda$ に対して, 
$w$ の $\lambda$ への shifted action を
\begin{equation*}
 w\circ\lambda = w(\lambda+\rho)-\rho
\end{equation*}
と定める. このとき 
\begin{align*}
  &
  \lambda_k = (s_{i_{k-1}}\cdots s_{i_1})\circ\lambda,
%  \\ &
\quad
  f^\lambda_w = f_{i_N}^{\bra\av_{i_N},\lambda_N\ket+1}
         \cdots f_{i_1}^{\bra\av_{i_1},\lambda_1\ket+1}
\end{align*}
とおくと, $X$, $Y$ の積の順序を逆転し, 各 $\av_i$ に
非負の整数 $\bra\av_i,\lambda\ket$ を代入した結果は
それぞれ $f^\lambda_w$, $f^{\lambda+\mu}_w$ になる.
ゆえに $f^{\lambda+\mu}_w \in U_q(\nil_-)f^\lambda_w$ を示せばよい.

$f^{\lambda+\mu}_w \in U_q(\nil_-)f^\lambda_w$ の証明
が次の可換図式の存在に帰着することを示そう:
\begin{equation*}
\begin{CD}
  M(\lambda)\otimes L(\mu) @<<< M(w\circ\lambda)\otimes L(\mu) \\
  @AAA                          @AAA \\
  M(\lambda+\mu)           @<<< M(w\circ(\lambda+\mu)). \\
\end{CD}
\end{equation*}
この可換図式の内容について説明しよう.
$M(\lambda)$ と $L(\lambda)$ はそれぞれ最高ウェイト $\lambda$ のVerma加群と
その既約商加群であるとし, それぞれの最高ウェイトベクトルを $v_\lambda$, $u_\lambda$
と書くことにする.  このとき $f^\lambda_w v_\lambda$ は $M(\lambda)$ における
ウェイト $w\circ\lambda$ の特異ベクトル(定数倍を除いて唯一)になる.  
ゆえに $v_{w\circ\lambda}$ を $f^\lambda_w v_\lambda$ に移す $M(w\circ\lambda)$ 
から $M(\lambda)$ への準同型写像が得られる. 上の図式の横向きの矢線は
この種の準同型とこの種の準同型から誘導された準同型写像である. 
左側の上向きの矢線は $v_{\lambda+\mu}$ を $v_\lambda\otimes u_\mu$ に移す
準同型写像である.  上の図式を可換にするような右側の上向きの矢線(準同型写像)
が存在すると仮定する.

上の可換図式における $v_{w\circ(\lambda+\mu)}\in M(w\circ(\lambda+\mu))$ 
の $M(\lambda)\otimes L(\mu)$ における像を考える.  $M(\lambda+\mu)$ を
経由して得られる像は $\Delta(f^{\lambda+\mu}_w)(v_\lambda\otimes u_\mu)$ である
(ここで $\Delta$ は余積を表わす).  上の可換図式より, 
それは $M(w\circ\lambda)\otimes L(\mu)$ の $M(\lambda)\otimes L(\mu)$ に
おける像 $(U_q(\nil_-)f^\lambda_w v_\lambda)\otimes L(\mu)$ に含まれる:
\begin{equation*}
 (U_q(\nil_-)f^\lambda_w v_\lambda)\otimes L(\mu) \ni
 \Delta(f^{\lambda+\mu}_w)(v_\lambda\otimes u_\mu)
 = (a f^{\lambda+\mu}_w v_\lambda)\otimes u_\mu + (\text{lower terms}).
\end{equation*}
ここで $a$ は $0$ でない定数であり, 
``$(\text{lower terms})$'' の部分は $M(\lambda)$ のベクトルとウェイトが $\mu$ より
小さな $L(\mu)$ のウェイトベクトルのテンソル積の和である.
これより $U_q(\nil_-)f^\lambda_w \ni f^{\lambda+\mu}_w$ となることがわかる.

上の可換図式は Kac-Moody 代数の場合には \cite{DGK} の結果より存在することが
知られている(translation functor の理論より, \cite{KW} Section 2 も参照).
ゆえに量子展開環の場合にも \cite{EK-VI} の結果より存在することがわかる
($U(\g)$ のBGG圏 $\cO$ と $U_q(\g)$ のBGG圏 $\cO$ 
の braided tensor category として圏同値より).

要するに $\mu\in P_+$ に対する $w(\tau^\mu)$ が $f_i$ について多項式になる
という結果は任意の $\lambda\in P_+$ に
対して $M(\lambda+\mu)$ のウェイト $w\circ(\lambda+\mu)$ の特異ベクトル
が $M(\lambda)$ のウェイト $w\circ\lambda$ の特異ベクトルで割り切れる
という結果から導かれる.



\subsection{Weyl群作用のLax-Sato-Wilson表示 (2)}
\label{sec:Sato-Wilson-2}

この節では $\A$ は\secref{sec:f^gamma}の代数 $\B$ であるとし, 
$\K$ は\secref{sec:fhat}の $t_i$, $\hf_i$ たちで生成される斜体であるとし, 
前節までとは少し異なるWeyl群作用を $\K(q^\Qv)[\tau^P]$ 上に定める.

$U_q(\nil_-)$ のChevalley生成元の取り方には定数倍の不定性があり, 
公式 $s_i(\tau_i)=f_i\tau_i$ より, 
Weyl群の $\tau$ 変数への作用の仕方は Chevalley 生成元を定数倍すれば変わる.
$\tau$ 変数へのWeyl群作用にはもっと大きな不定性がある.

前節までの議論から $f_i$ たちが Verma 関係式を満たしていれば, 
$s_i=\Ad(f_i^{\av_i})\ts_i$ の形でWeyl群作用を定めるには十分である.
さらに $q$ べき因子の違いを除いて $A$ と $B$ が等しいことを $A\sim B$ と
書くことにすると, Verma関係式そのものがぴったり成立している必要はなく, 
$f_i$ たちが \(
      f_i^{\av_i} f_j^{\av_i+\av_j} f_i^{\av_j}
 \sim f_j^{\av_j} f_i^{\av_i+\av_j} f_j^{\av_j}
\) ($a_{ij}=a_{ji}=-1$) のタイプの弱いVerma関係式を満たしていれば, 
$s_i=\Ad(f_i^{\av_i})\ts_i$ によってWeyl群作用を定めることができる.
$f_i$ たちがVerma関係式を満たしており, 
$u_i\in Q(\A)$ たちの積は $q$ べき因子の違いを除いて可換であり, 
$f_iu_j=u_jf_i$ が成立しているとき, 
$u_if_i$ たちは弱いVerma関係式を満たしている.
だから $s_i=\Ad((u_if_i)^{\av_i})\ts_i$ によってWeyl群作用を
定めることもできる. このとき $s_i(\tau_i)=u_if_i\tau_i$ となる.

この構成を\secref{sec:fhat}の $\hf_i$ たちに適用しよう.
$\A$ は\secref{sec:f^gamma}の $\B$ であり, 
$\K$ は\secref{sec:fhat}の $t_i$, $\hf_i$ たちで生成される斜体であるとする.
$\hf_i$ は $f_i$ に $f_i$ たちと可換な $c_j^{\pm1}$ たちの積を
かけたものになっており, $c_j$ たちの積は $q$ べき因子の違いを除いて可換である.
さらに $\hf_i$ たちと $t_j$ たちは可換であり, $t_{i+3}=r^{-1}t_i$, 
$q^{-\ev_{i+3}}=q^{\dv}q^{-\ev_i}$ が成立しているので, 
$r=q^{-\dv}$, $t_i=q^{-\ev_i}$ と同一視することにする.
さらに, \secref{sec:fhat}と同様に $\ts_i$ を定めておく.
このように設定を変えることによって, $\K(q^\Qv)$ の代わりに $\K$ そのものを
扱えばよいことになる.
$\pi$ の $\K$ への作用は\secref{sec:fhat}と同様に定めておく.

$q$ 差分作用素環 $\K[\tau^P]$ に $s_i=\Ad(\hf_i^{\av_i})\ts_i$ に
よってWeyl群作用を定めることができる. このとき, 以下の公式が成立している%
\footnote{$\hb_i=-(q-q^{-1})t_i\hf_i t_{i+1}$ を使えば
それらの公式から $s_i(\hf_i)$ を計算することができる.}:
\begin{align*}
 &
 s_i(\tau_i) = \hf_i \frac{\tau_{i-1}\tau_{i+1}}{\tau_i}, \quad
 s_i(\tau_j) = \tau_j \quad
 (i,j=0,1,2,\ i\ne j),
 \\ &
 s_i(t_i) = t_{i+1}, \quad
 s_i(t_{i+1}) = t_i, \quad
 s_i(t_{i+2}) = t_{i+2},
 \\ &
 s_i(\hb_i) = \hb_i, \quad
 s_i(\hb_{i+1}) = \hb_{i+1} + (t_i^2-t_{i+1}^2)\hb_i^{-1}, \quad
 s_i(\hb_{i-1}) = \hb_{i-1} - (t_i^2-t_{i+1}^2)\hb_i^{-1},
 \\ &
 \pi(t_j) = t_{j+1}, \quad
 \pi(\hb_j) = \hb_{j+1}.
\end{align*}
新しい公式は最初の行だけである, 
残りの公式はすでに\secref{sec:fhat}で得られている.
\secref{sec:fhat}では $s_i$ の $t_i$ と $\hb_i$ たちへの作用を
行列 $\hL(z)$ を用いて記述する公式(Lax表示)が得られている.
以下では $s_i$ の $\tau$ 変数たちへの作用の行列表示(Sato-Wilson表示)
について説明する.

以下の構成は\secref{sec:Sato-Wilson-1}と完全に同様である.
$t=(t_1,t_2,t_3)$ とおき,
\begin{equation*}
 D(t)=\diag(t_1,t_2,t_3)
\end{equation*}
と定め, $z$ を $rz$ に移す差分作用素を $T_{z,r}$ と書き, 
行列値差分作用素 $\hM(z)$ を
\begin{equation*}
  \hM(z) = \hL(z)T_{z,r}^2
\end{equation*}
と定める. この行列値差分作用素 $\hM(z)$ を $(D(t)T_{z,r})^2$ に``対角化''しよう.
$\K[[z]]$ の元を成分に持つ $3\times 3$ 行列 $U(z)=[u_{ij}(z)]$ 
で $U(0)$ が対角成分がすべて $1$ の上三角行列になり, 
\begin{equation*}
  \hM(z) = U(z)(D(t)T_{z,r})^2U(z)^{-1},
  \quad
  \text{すなわち}\ \hL(z)=U(z)D(t)^2U(r^2z)^{-1}
\end{equation*}
を満たすものが一意に存在する.  
この一意性を使うと $\hL(z)$ へのWeyl群作用の公式から $U(z)$ へのWeyl群作用の
公式を得ることができる. 結果を以下で説明しよう.

$g_i\in\K$ と行列 $G_i$ を次のように定める($i=1,2$):
\begin{equation*}
 g_i = \frac{t_i^2-t_{i+1}^2}{\hb_i} = \frac{[\av_i]_q}{\hf_i}, \quad
 G_i = y_i(g_i).
\end{equation*}
このとき, $u_{i,i+1}(0) =  -g_i^{-1}$ となり, 
\begin{align*}
 &
 s_i(\hM(z)) = G_i \hM(z) G_i^{-1},
 \\ &
 \pi(\hM(z)) = (\Lambda_3(z)T_{z,r})\hM(z)(\Lambda_3(z)T_{z,r})^{-1}
\end{align*}
が成立している. これをWeyl群作用の{\bf Lax表示}と呼ぶ.

$z_i=\tau^{\eps_i}$ とおき, 対角行列 $D_Z$ と行列 $Z(z)$ を
\begin{equation*}
  D_Z=\diag(z_1,z_2,z_3), \quad Z(z) = U(z)D_Z
\end{equation*} 
と定める. 行列 $S^g_i$, $S_i$ を次のように定める:
\begin{align*}
 &
 S^g_1 =
 \begin{bmatrix}
   0    & g_1^{-1} & 0 \\
   -g_1 & 0        & 0 \\
   0    & 0        & 1 \\
 \end{bmatrix},
 \quad
 S^g_2 =
 \begin{bmatrix}
   1 & 0 & 0 \\
   0 & 0 & g_2^{-1} \\
   0 & -g_2 & 0 \\
 \end{bmatrix},
 \\ &
 S_1 =
 \begin{bmatrix}
   0           & -[\av_1+1]_q & 0 \\
   [\av_1-1]_q & 0        & 0 \\
   0           & 0        & 1 \\
 \end{bmatrix},
 \quad
 S_2 =
 \begin{bmatrix}
   1 & 0           & 0 \\
   0 & 0           & -[\av_2+1]_q \\
   0 & [\av_2-1]_q & 0 \\
 \end{bmatrix}.
\end{align*}
このとき以下が成立している:
\begin{align*}
 &
 \hM(z)
 = Z(z)(D(qt)T_{z,r})^2 Z(z)^{-1}
 = Z(z)D(qt)^2Z(r^2z)^{-1}T_{z,r}^2,
 \\[\medskipamount] &
 s_i(U(z)) = G_i U(z) S^g_i, 
 \\ &
 s_i(D_Z) = (S^g_i)^{-1} D_Z S_i, \
 \\ &
 s_i(D(t)T_{z,r}) = S_i^{-1} D(t)T_{z,r} S_i
 \\ &
 s_i(Z(z)) = G_i Z(z) S_i.
\end{align*}
$\pi$ の作用はどれも次の形になっている:
\begin{equation*}
  \pi(A(z)) = (\Lambda_3(z)T_{z,r})X(z)(\Lambda_3(z)T_{z,r})^{-1} \qquad
  (A(z)=U(z),D_Z,D(t)T_{z,r},Z(z)).
\end{equation*}
これをWeyl群作用の{\bf Sato-Wilson表示}と呼ぶ.




\subsection{まとめ}

量子化された $\tau$ 函数(より正確に言えば $\tau$ 変数)は
単純コルートに対応するパラメーター変数 $\av_i$ の正準共役 $\partial/\partial\av_i$
の指数函数 $\tau_i=exp(\partial/\partial\av_i)$ のことである.

量子化されたWeyl群作用は $s_i=\Ad(f_i^{\av_i})\ts_i$ の形式で構成されたので
それをそのまま $\tau_i$ たちに作用させれば, Weyl群作用が量子化された $\tau$ 
変数まで拡張される.

その作用のもとで $\mu\in P_+$ に対する $w(\tau^\mu)$ は
従属変数 $f_i$ たちとパラメーター変数 $q^{\pm\av_i}$ たち 
について多項式になる.

$\tau$ 変数へのWeyl群作用はSato-Wilson表示を持つ. 


%%%%%%%%%%%%%%%%%%%%%%%%%%%%%%%%%%%%%%%%%%%%%%%%%%%%%%%%%%%%%%%%%%%%%%%

\section{付録: 「量子化」と「変数べき $f^\gamma$ の構成法」について}

この付録では(正準)量子化と古典極限についておよび非可換環における変数 $\gamma$
によるべき $f^\gamma$ の構成法について解説する.


\subsection{$q$ 差分化と量子化の区別}
\label{sec:quantization}

「$q$ 差分化」と「量子化」という用語を厳密に区別して用いたい.

微分 $df(x)/dx$ を用いた事柄をパラメーター $q$ で変形して,  $q$ 差分
\begin{equation*}
 T_{x,q} f(x) = \frac{f(x)-f(qx)}{x(1-q)}
\end{equation*}
を用いた事柄に拡張することおよびその適切な一般化を {\bf $q$ 差分化}と呼ぶことにする.
逆に $q\to 1$ で $T_{q,x}\to d/dx$ の極限を取る操作を {\bf 微分極限}と呼ぶことにする.
そして, 微分を用いて記述される数学的対象を{\bf 微分版}と呼び, 
$q$ 差分を用いて記述される数学的対象を{\bf $q$ 差分版}と呼ぶことにする.



Poisson括弧 $\{p,x\}=1$ を持つ代数に関する事柄を交換関係
\begin{equation*}
 [p,x] = px - xp = 1
\end{equation*} 
を満たす非可換環に関する事柄に拡張することおよびその適切な一般化を{\bf 正準量子化}と呼ぶことにする.
以下では簡単のため正準量子化を単に{\bf 量子化}と呼ぶことにする.

たとえば, 座標 $x$ を持つ直線の余接束上の函数環の自然な量子化は $x$ と $d/dx$ で生成される
微分作用素環である. 

たとえば, Lie代数 $\g$ の普遍展開環 $U(\g)$ は $\g$ から生成される対称代数 $S(\g)$ 
(これは $\g^*$ 上の函数環とみなされる)の量子化である.
さらに $\g$ がKac-Moody代数ならば, 量子展開環 $U_q(\g)$ は普遍展開環 $U(\g)$ の
代数としての $q$ 差分化であるとみなせる.  
代数として $U_q(\g)$ を $U(\g)$ の量子化とはみなさない.
$U(\g)$ は不変微分作用素環とみなせるので微分版の数学的対象とみなされ,  
それとの対比で $U_q(\g)$ は代数として $q$ 差分版の数学的対象とみなされる.
代数として量子展開環 $U_q(\g)$ は対称代数 $S(\g)$ の量子化かつ $q$ 差分化になっている.

一方, $U_q(\g)$ の余積は非余可換であり, $q\to 1$ の極限で余可換な $U(\g)$ の
余積が再現されるので, 量子展開環 $U_q(\g)$ は余代数として普遍展開環 $U(\g)$ の
量子化だとみなせる.

量子展開環 $U_q(\g)$ は代数として普遍展開環 $U(\g)$ の $q$ 差分化になっており, 
余代数としては普遍展開環 $U(\g)$ の量子化になっている. 
代数としての $q$ 差分化と余代数としての量子化のためのパラメーターの両方が
同じ $q$ になっているので混乱しないように注意が必要である.




\subsection{古典極限とPoisson構造}
\label{sec:classical-limit}

パラメーター $\hbar$ を含む可換とは限らない環 $A_\hbar$ に対して, 
$\hbar\to 0$ の極限で得られる環(すなわち $\hbar$ で生成されるイデアル
で $A_\hbar$ を割ってできる剰余環 $A_\hbar/\hbar A_\hbar$)を $A_0$ と書き, 
$a\in A_\hbar$ の $A_0$ での像を $\bar{a}$ と書くことにする.

$A_0$ は可換環になると仮定する. 
このとき $A_\hbar$ は $A_0$ の量子化であるという.
仮定より, 任意の $a,b\in A_\hbar$ の交換子 $[a,b]=ab-ba$ は $\hbar$ で割り切れる.
任意の $a,b\in A_\hbar$ に対して,  
$\hbar^{-1}[a,b]\in A_\hbar$ の $A_0$ での像は $a$, $b$ の像だけから決まることがわかり, 
可換環 $A_0$ に Poisson 構造を
\(
 \{\bar{a}, \bar{b}\} = \overline{\hbar^{-1}[a,b]}
\)
によって定めることができる.
このように構成された Poisson 代数 $A_0$ を $A_\hbar$ の{\bf 古典極限}と呼ぶことにする.

たとえば, $p=d/dx$ と $x$ から $\C[\hbar]$ 上生成されるパラメーター $\hbar$ 付きの
微分作用素環 $\C[\hbar, x, d/dx]$ を考え, 
$p=\hbar d/dx$ と $x$ から $\C[\hbar]$ 上生成される部分代数を $A_\hbar$ と書くことにする.
このとき $A_0=A_\hbar/\hbar A_\hbar$ は $\bar{p}$ と $\bar{x}$ で
生成される可換な多項式環になり,  $\hbar^{-1}[p,x]=1$ なので, 
$A_0=\C[\bar{p},\bar{x}]$ には $\{\bar{p},\bar{x}\}=1$ で自然に Poisson 構造が定まる.
これが上の意味での古典極限の典型例である.  

$\C$ 上のLie環 $\g$ の普遍展開環 $U(\g)$ の古典極限は以下のように構成される.
まず, $U(\g)$ の $\C[\hbar]$ による係数拡大 $U(\g)\otimes\C[\hbar]$ を考える.
そして, $\hbar A$ ($A\in\g$) たちから $\C[\hbar]$ 上生成されるその部分代数を $A_\hbar$
とする.  このとき $A_0=A_\hbar/\hbar A_\hbar$ は可換環になり, 
$\hbar\g$ から $\C$ 上生成される対称代数 $S(\hbar\g)\cong S(\g)$ に同型になる.
これを $U(\g)$ の古典極限と呼ぶ.
さらに, $A,B\in\g$ に対して $[A,B]=C$ が成立しているとき, 
$\hbar^{-1}[\hbar A, \hbar B]=\hbar C$ となるので, 
$\hbar A$, $\hbar B$, $\hbar C$ の $A_0$ での像をそれぞれ $a$, $b$, $c$ と
書くと $A_0$ において $\{a,b\}=c$ が成立している.
要するに Lie 環におけるLieブラケット $[\ ,\ ]$ をそのままPoisson括弧 $\{\ ,\ \}$ に
置き換えるだけで古典極限における Poisson 構造が得られる.

生成元 $x$, $y$ と基本関係式 $xy=qyx$ で定義される $\C[q]$ 上の代数 $A_q$ を考える.
(古典極限は $q\to 1$ で得られる.  $\hbar=q-1$ とおいて考えよ.)
このとき $[x,y]=xy-yx=(q-1)yx$が成立しているので,  $A_{q=1}=A_q/(q-1)A_q$ は
可換環になり,  $A_{q=1}$ に自然に入る Poisson 構造は $\{\bar{x},\bar{y}\}=\bar{x}\bar{y}$ を
満たしている.  このことから $\{a,b\}=ab$ 型のPoisson構造の量子化は $xy=qyx$ 型の交換関係
で定義される代数を用いて構成可能なことがわかる.  この例は $q$ 差分化された場合の量子化の
構成において最も基本的である.

古典極限の形だけがわかっていても, 可能な量子化は無数にある.
我々が欲しいのは古典極限のレベルで成立している多くの結果が保たれるような
良い量子化である.
$q$ 差分化されていない微分版の場合には, 
手頃な手間で可能な直線的な手計算のみで良い量子化を構成可能なことが多いが, 
$q$ 差分化された場合はそうではない.
背景に隠れている代数構造を明らかにしないと良い量子化を作ることは難しい.

しかも実際には古典極限におけるPoisson構造の定義式が明らかになっていない場合が多数ある.
そのような場合には量子化はさらに困難になる%
\footnote{古典Painlev\'e系の研究者の方々にお願い. 
Poisson構造も論文に書いて下さい.  お願い致します.}.




\subsection{非可換環の元の変数によるべきを導入する方法}
\label{sec:powers1}

非可換環の元 $f$ の変数 $\gamma$ によるべき $f^\gamma$ を純代数的に
構成する方法を解説しておく.
さらに次の節で量子展開環の下三角部分のChevalley生成元のべきが満たす
関係式の証明も簡単に解説しておく.

$A$ は可換とは限らない環であるとする.

$X$ は空でない集合であるとする.
このとき, べき集合 $A^X$ には自然に環構造が入る.
以下では, この状況の下で, $A$ と
単射準同型写像 $A\to A^X$, $a\mapsto (a)_{x\in X}$ による像を同一視することにする.
これによって $A$ は $A^X$ の部分環とみなされる.

変数 $\gamma$ による $A$ の可逆元 $f$ のべき $f^\gamma$ は
環 $A^\Z$ の元として $f^\gamma = (f^k)_{k\in\Z}$ と定義される.
任意の $k\in\Z$ について $f^k$ が満たしている関係式を $f^\gamma$ 
も満たしている.  $A^\Z$ に埋め込まれた $A$ と $f^\gamma$ から
生成された $A^\Z$ の部分環を考えれば, $A$ に $f$ の変数 $\gamma$ 
によるべき $f^\gamma$ を付け加えた環が得られる.

より一般に, 変数の組 $\gamma=(\gamma_1,\ldots,\gamma_N)$ の
整係数多項式 $p(\gamma)$ に対して,  $f^{p(\gamma)}$ は
$A^{\Z^N}$ の元として \(
  f^{p(\gamma)}
  = \left(f^{p(k)}\right)_{k\in\Z^N}
\) と定義される. 
このように定義された $f$ のべきは変数 $\gamma_i$ に任意の整数の組を
代入したときに成立しているすべての関係式をみたしている. 
たとえば
\begin{equation*}
 f^\gamma f^{-\gamma} = f^{-\gamma}f^\gamma = 1,  \quad
 f^\lambda f^\mu = f^{\lambda+\mu}
\end{equation*}
が成立している. さらに, もしも $A$ の可逆元 $f$, $g$ と中心可逆元 $c$ が
\begin{equation*}
 f g = c g f
\end{equation*} 
という関係式を満たしていれば
\begin{equation*}
 f^\beta g^\gamma = c^{\beta\gamma} g^\gamma f^\beta
\end{equation*}
が成立している. 

他にも, たとえば, もしも $A$ の可逆元 $f,g$ が
\begin{equation*}
 f^k g^{k+l} f^l = g^l f^{k+l} g^k
  \quad (k,l\in\Z)
\end{equation*}
という関係式を満たしていれば
\begin{equation*}
  f^\beta g^{\beta+\gamma} f^\gamma
 =g^\gamma f^{\beta+\gamma} g^\beta.
\end{equation*}
も成立している. 

このように単純な処方箋で, 
自然な関係式を満たしている変数 $\gamma$ によるべき $f^\gamma$ を
非可換環に自由に付け加えることができる.




\subsection{Chevalley生成元のべきが満たす関係式}
\label{sec:powers2}

いつものように $q$ 二項係数を次のように定めておく:
\begin{align*}
 &
 [x]_q = \frac{q^x-q^{-x}}{q-q^{-1}}, \quad
 [k]_q! = [1]_q[2]_q\cdots[k]_q,
 \\ &
 \qbinom{x}{k}_q = \frac{[x]_q[x-1]_q\cdots[x-k+1]_q}{[k]_q!} \quad
 (k=0,1,2,\ldots).
\end{align*}

$\C(q)$ 上の代数 $\A$ の可逆元 $f,g$ はを任意に取り, 
非負の整数 $k$ に対する $\ad_q(f)^k(g)\in\A$ を次のように定める:
\begin{equation*}
  \ad_q(f)(g)= fg - q^{-1} gf, \quad
  \ad_q(f)^{k+1}(g) = f\ad_q(f)^k(g) - q^{2k-1}\ad_q(f)^k(g) f.
\end{equation*}
このとき, すべての整数 $l$ について
次が成立することを帰納法で示せる%
\footnote{$l$ が正の場合には両辺に右から $f^l$ をかけて
得られる公式を帰納法で証明し, $l$ が負の場合には両辺に左から $f_i^{-l}$ を
かけて得られる公式を証明する.}:
\begin{equation*}
  f^l g f^{-l} = 
  \sum_{k=0}^\infty q^{(k-1)(l-k)}\qbinom{l}{k}_q \ad_q(f)^k(g) f^{-k}.
\end{equation*}
$k>l$ のとき $q$ 二項係数は $0$ になるので右辺は実際には有限和である.

ここで, 次の $q$-Serre 関係式が成立していると仮定する:
\begin{equation*}
 \ad_q(f)^2(g) = f^2 g - (q+q^{-1})fgf + gf^2 = 0.
\end{equation*}
このとき, 上の公式より次が成立する:
\begin{equation*}
  f^l g f^{-l} = q^{-l}g + [l]_q(fg-q^{-1}gf)f^{-1}.
\end{equation*}
ゆえに前節の構成によって変数によるべき $f^\gamma$ は次を満たしている:
\begin{equation*}
 f^\gamma g f^{-\gamma} 
 = q^{-\gamma} g + [\gamma]_q(fg-q^{-1}gf)f^{-1}.
\end{equation*}
この右辺は次のように変形できる:
\begin{equation*}
 f^\gamma g f^{-\gamma}
 = [1-\gamma]_q g + [\gamma]_q fgf^{-1}
 = q^\gamma g + [\gamma]_q(fg-qgf)f^{-1}.
\end{equation*}

さらに, 次の $q$-Serre 関係式も成立していると仮定する:
\begin{equation*}
 g^2 f - (q+q^{-1})gfg + fg^2 = 0.
\end{equation*}
このとき量子展開環の表現論によって任意の非負の整数 $k,l$ に対して
\begin{equation*}
  f^k g^{k+l} f^l = g^l f^{k+l} g^k
\end{equation*}
が成立することが知られている%
\footnote{Verma加群のあいだの準同型がどれだけ
存在するかに関する結果の一部である.
$q$-Serre関係式を用いた直接の計算によって証明することもそう難しくない.}.
これを{\bf Verma関係式}と呼ぶ%
\footnote{より一般のVerma関係式とその証明については
教科書 \cite{Lusztig} の Proposition 39.3.7 を参照せよ.}.
$k,l$ が両方非負の整数の場合のVerma関係式から, 
$k,l$ のどちらか片方もしくは両方が
負の整数の場合にも同じ公式が成立することが導かれる%
\footnote{たとえば $k$ が負で $k+l$ が非負の場合のVerma関係式は, 
両辺に左から $f^{-k}$ をかけて 
右から $g^{-k}$ をかけると, 
非負の整数べきのみを含むVerma関係式の形になる.}.

ゆえに, 前節の構成によって, 
整数べきを変数べきで置き換えた次の公式が成立している:
\begin{equation*}
  f^\beta g^{\beta+\gamma} f^\gamma
 =g^\gamma f^{\beta+\gamma} g^\beta.
\end{equation*}
これもVerma関係式と呼ぶ.

%%%%%%%%%%%%%%%%%%%%%%%%%%%%%%%%%%%%%%%%%%%%%%%%%%%%%%%%%%%%%%%%%%%%%%%

\begin{thebibliography}{99}

\bibitem{BPZ}
Belavin, A.\ A., Polyakov, A.\ M., and Zamolodchikov, A.\ B. 
Infinite conformal symmetry in two-dimensional quantum field theory. 
Nuclear Phys.\ B 241 (1984), no.~2, 333-380.

\bibitem{BK-GC1}
Berenstein, Arkady and Kazhdan, David. 
Geometric and unipotent crystals. 
GAFA 2000 (Tel Aviv, 1999).  
Geom.\ Funct.\ Anal.\  2000,  Special Volume, Part I, 188--236.
\arxivref{math/9912105}

\bibitem{BK-GC2}
Berenstein, Arkady and Kazhdan, David. 
Geometric and unipotent crystals.\ II.\ 
From unipotent bicrystals to crystal bases. 
Quantum groups, 13–-88, Contemp.\ Math., 433, 
Amer.\ Math.\ Soc., Providence, RI, 2007.
\arxivref{math/0601391}

\bibitem{DGK}
Deodhar, Vinay V., Gabber, Ofer, and Kac, Victor. 
Structure of some categories of representations 
of infinite-dimensional Lie algebras. 
Adv.\ in Math.\ 45 (1982), no.~1, 92--116.

\bibitem{EK-VI}
Etingof, Pavel and Kazhdan, David. 
Quantization of Lie bialgebras. 
VI. Quantization of generalized Kac-Moody algebras. 
Transform.\ Groups 13 (2008), no.~3--4, 527--539. 
\arxivref{math/0004042}

\bibitem{Hasegawa-QB}
Hasegawa, Koji.
Quantizing the B\"acklund transformations of Painlev\'e equations and 
the quantum discrete Painlev\'e VI equation. 
{\em Exploring new structures and natural constructions in mathematical physics}, 275-288, 
Adv.\ Stud.\ Pure Math., 61, {\em Math.\ Soc.\ Japan}, Tokyo, 2011.
\arxivref{math/0703036}

\bibitem{KW}
Kac, Victor G.\ and Wakimoto, Minoru.
Modular invariant representations of infinite-dimensional Lie algebras and superalgebras.
Proc.\ Nat.\ Acad.\ Sci.\ U.S.A.\ 85 (1988), no.~14, 4956--4960.

\bibitem{KNY-qPIV}
Kajiwara, Kenji, Noumi, Masatoshi, and Yamada, Yasuhiko.
A study on the fourth q-Painlev\'e equation.
J.\ Phys.\ A 34 (2001), no.~41, 8563-8581. 
\arxivref{nlin/0012063}

\bibitem{KNY-WxW}
Kajiwara, Kenji, Noumi, Masatoshi, and Yamada, Yasuhiko.
Discrete dynamical systems with $W(A^{(1)}_{m-1}(1)\times A^{(1)}_{n-1})$ symmetry.
Lett.\ Math.\ Phys.\ 60 (2002), no.~3, 211-219. 
\arxivref{nlin/0106029}

\bibitem{KNY-qKP}
Kajiwara, Kenji, Noumi, Masatoshi, and Yamada, Yasuhiko.
$q$-Painlev\'e systems arising from $q$-KP hierarchy.
Lett.\ Math.\ Phys.\ 62 (2002), no.~3, 259-268. 
\arxivref{nlin/0112045}

\bibitem{Kuroki-localization}
黒木玄.
非可換環の局所化の演習問題.
演習問題集 10 pages, 2007年.
\\ \href
{http://www.math.tohoku.ac.jp/~kuroki/LaTeX/20070216_localization.pdf}
{http://www.math.tohoku.ac.jp/{\TILDE}kuroki/LaTeX/20070216{\US}localization.pdf}

\bibitem{Kuroki-Ore}
黒木玄.
Ore集合の作り方.
ノート 15~pages, 2010年作成.
\\ \href
{http://www.math.tohoku.ac.jp/~kuroki/LaTeX/20101116OreSets.pdf}
{http://www.math.tohoku.ac.jp/{\TILDE}kuroki/LaTeX/20101116OreSets.pdf}

\bibitem{Kuroki-WxW2010}
黒木玄.
量子 $\tW(A^{(1)}_{m-1})\times\tW(A^{(1)}_{n-1})$ 双有理作用.
ノート 15~pages, 2010年作成.
\\ \href
{http://www.math.tohoku.ac.jp/~kuroki/LaTeX/20100630_WxW.pdf}
{http://www.math.tohoku.ac.jp/{\TILDE}kuroki/LaTeX/20100630{\US}WxW.pdf}

\bibitem{Kuroki-WxW2013}
黒木玄.
互いに素な $m$, $n$ に対する
拡大アフィンWeyl群の直積 $\WW(A^{(1)}_{m-1})\times\widetilde{W}(A^{(1)}_{n-1})$ の
双有理作用の量子化. 講演スライド, 2013年作成.
\\ \href
{http://www.math.tohoku.ac.jp/~kuroki/LaTeX/20130322WxW.pdf}
{http://www.math.tohoku.ac.jp/{\TILDE}kuroki/LaTeX/20130322WxW.pdf}

\bibitem{Kuroki-W}
Kuroki, Gen. 
Quantum groups and quantization of Weyl group symmetries of Painlev\'e systems. 
{\rm Exploring new structures and natural constructions in mathematical physics}, 289-325, 
Adv.\ Stud.\ Pure Math., 61, {\em Math.\ Soc.\ Japan}, Tokyo, 2011.
\arxivref{0808.2604}

\bibitem{Kuroki-Tau}
Kuroki, Gen.
Regularity of quantum tau-functions generated by quantum birational Weyl group actions.
\arxivref{1206.3419}

\bibitem{Lusztig}
Lusztig, George. 
Introduction to quantum groups. 
Reprint of the 1994  edition.  
Modern Birkhauser Classics. 
Birkhauser/Springer, New York, 2010.\ xiv+346 pp. 

\bibitem{Noumi-AWG}
Noumi, Masatoshi.
Affine Weyl group approach to Painlev\'e equations.
Proceedings of the International Congress of Mathematicians, 
Vol.~III (Beijing, 2002), 497-509, Higher Ed.\ Press, Beijing, 2002. 
\arxivref{math-ph/0304042}

\bibitem{NY-BWA}
Noumi, Masatoshi and Yamada, Yasuhiko.
Birational Weyl group action arising from a nilpotent Poisson algebra. 
Physics and combinatorics 1999 (Nagoya), 287--319, 
World Sci.\ Publ., River Edge, NJ, 2001. 
\arxivref{math.QA/0012028}

\bibitem{NY-RSK}
Noumi, Masatoshi and Yamada, Yasuhiko.
Tropical Robinson-Schensted-Knuth correspondence and birational Weyl group actions. 
{\em Representation theory of algebraic groups and quantum groups}, 371-442,
Adv.\ Stud.\ Pure Math., 40, {\em Math.\ Soc.\ Japan}, Tokyo, 2004. 
\arxivref{math-ph/0203030}

\end{thebibliography}

%%%%%%%%%%%%%%%%%%%%%%%%%%%%%%%%%%%%%%%%%%%%%%%%%%%%%%%%%%%%%%%%%%%%%%%%%%%%
\end{document}
%%%%%%%%%%%%%%%%%%%%%%%%%%%%%%%%%%%%%%%%%%%%%%%%%%%%%%%%%%%%%%%%%%%%%%%%%%%%
