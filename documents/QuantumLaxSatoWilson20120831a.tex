%%%%%%%%%%%%%%%%%%%%%%%%%%%%%%%%%%%%%%%%%%%%%%%%%%%%%%%%%%%%%%%%%%%%%%%%%%%%
\def\TITLE{\bf Sato-Wilson formalism for the quantum birational Weyl group actions of type $A$}
\def\AUTHOR{Gen Kuroki\thanks{\ADDRESS.}}
\def\DATE{June 15, 2012, \quad Version 0.11\thanks{
This is an unfinished draft
with typographies and logical errors. 
Please do not redistribute this draft printed on papers.
The latest version can be retrieved from \href
{http://www.math.tohoku.ac.jp/\~kuroki/LaTeX/QuantumLaxSatoWilson.pdf}
{here}.
%{http://www.math.tohoku.ac.jp/{\textasciitilde}kuroki/LaTeX/QuantumLaxSatoWilson.pdf}.
The twitter account of the author is 
\href{https://twitter.com/genkuroki}{genkuroki}.
Any comments are welcome.}}
% 2012-05-16 Ver.0.00 Created
% 2012-05-25 Ver.0.01 6 pages
% 2012-05-26 Ver.0.02 7 pages, Proof of A_{n-1} Sato-Wilson added
% 2012-05-26 Ver.0.03 8 pages, Proof of A_{n-1} Lax added
% 2012-05-27 Ver.0.04 8 pages, Remarks in Section 1 added, minor corrections
% 2012-05-27 Ver.0.05 10 pages
% 2012-05-29 Ver.0.06 12 pages, A_\infty case added
% 2012-05-29 Ver.0.07 13 pages, A^{(1)}_{n-1} for n\geqq 3 case partially added
% 2012-06-01 Ver.0.08 14 pages
% 2012-06-04 Ver.0.09 16 pages, A^{(1)}_{n-1} for n\geqq 3 case completed
% 2012-06-06 Ver.0.10 minor corrections
% 2012-06-15 Ver.0.11 definition of S_i changed and \eta_i and \hs_i removed!
\def\ADDRESS{Mathematical Institute, Tohoku University, Sendai 980-8578, Japan}
%%%%%%%%%%%%%%%%%%%%%%%%%%%%%%%%%%%%%%%%%%%%%%%%%%%%%%%%%%%%%%%%%%%%%%%%%%%%
\def\ABSTRACT{
We present the the Lax and the Sato-Wilson formalisms for the $q$-difference version 
of the quantized birational Weyl group actions 
of type $A_{n-1}$, of type $A_\infty$, and of type $A^{(1)}_{n-1}$.
}
%%%%%%%%%%%%%%%%%%%%%%%%%%%%%%%%%%%%%%%%%%%%%%%%%%%%%%%%%%%%%%%%%%%%%%%%%%%%
\def\ACKNOWLEDGEMENTS{
This work was supported by Grant-in-Aid for Scientific Research No.~23540003.
}
%%%%%%%%%%%%%%%%%%%%%%%%%%%%%%%%%%%%%%%%%%%%%%%%%%%%%%%%%%%%%%%%%%%%%%%%%%%%
\documentclass[12pt,twoside]{article}
\usepackage{amsmath,amssymb,amsthm}
\usepackage{hyperref}
\pagestyle{headings}
\setlength{\oddsidemargin}{0cm}
\setlength{\evensidemargin}{0cm}
\setlength{\topmargin}{-1.6cm}
\setlength{\textheight}{24cm}
\setlength{\textwidth}{16cm}
\allowdisplaybreaks
%%%%%%%%%%%%%%%%%%%%%%%%%%%%%%%%%%%%%%%%%%%%%%%%%%%%%%%%%%%%%%%%%%%%%%%%%%%%
\newcommand\A{{\mathcal A}}
\newcommand\B{{\mathcal B}}
\newcommand\G{{\mathcal G}}
\newcommand\K{{\mathcal K}}
\newcommand\M{{\mathcal M}}
\newcommand\N{{\mathcal N}}
\renewcommand\L{{\mathcal L}}
\renewcommand\O{{\mathcal O}}
\renewcommand\b{{\mathfrak b}}
\renewcommand\d{\partial}
\newcommand\g{{\mathfrak g}}
\newcommand\h{{\mathfrak h}}
\newcommand\n{{\mathfrak n}}

\newcommand\tA{{\widetilde A}}
\newcommand\tD{{\widetilde D}}
\newcommand\tK{{\widetilde K}}
\newcommand\tL{{\widetilde L}}
\newcommand\tP{{\widetilde P}}
\newcommand\tU{{\widetilde U}}
\newcommand\tW{{\widetilde W}}
\newcommand\tcA{\widetilde{\mathcal A}}
\newcommand\ts{{\tilde s}}
\newcommand\tw{{\widetilde w}}

\newcommand\bars{{\overline s}}
\newcommand\barG{{\overline G}}
\newcommand\barS{{\overline S}}

\newcommand\av{\alpha^\vee}
\newcommand\eps{\varepsilon}
\newcommand\epsv{\eps^\vee}
\newcommand\deltav{\delta^\vee}
\newcommand\Qv{Q^\vee}

\newcommand\arxivref[1]{\href{http://arxiv.org/abs/#1}{\tt arXiv:#1}}
\newcommand\bra{\langle}
\newcommand\ket{\rangle}
\newcommand\Hom{\mathop{\mathrm{Hom}}\nolimits}
\newcommand\Aut{\mathop{\mathrm{Aut}}\nolimits}
\makeatletter\newcommand\qbinom{\genfrac[]\z@{}}\makeatother
\newcommand\ad{\mathop{\mathrm{ad}}\nolimits}
\newcommand\Ad{\mathop{\mathrm{Ad}}\nolimits}
\newcommand\pr{\mathop{\mathrm{pr}}\nolimits}
\newcommand\isom{\cong}
\renewcommand\setminus{\smallsetminus}
\newcommand\pa{{\mathrm{pa}}}
\newcommand\bs{{\mathbf s}}
\newcommand\be{{\mathbf e}}
\newcommand\Oq[1]{\O_{q,#1}}
%\newcommand\intpart{{\mathrm{int}}}
\newcommand\intpart{P}
\newcommand\Oint{\O_\intpart}
\newcommand\Ointq{\O_{\intpart,q}}
\newcommand\Ointg{\Oint^{\mathrm{g}}}
\newcommand\Ointqg{\Ointq^{\mathrm{g}}}
\newcommand\Kint{K_{\intpart}}
\newcommand\Kintg{\Kint^{\mathrm{g}}}
\newcommand\Ob{\mathop{\mathrm{Ob}}\nolimits}
\newcommand\Peq[1]{\mathrm{P}_{\mathrm{#1}}}
\newcommand\PI{\Peq{I}}
\newcommand\PII{\Peq{II}}
\newcommand\PIII{\Peq{III}}
\newcommand\PIV{\Peq{IV}}
\newcommand\PV{\Peq{V}}
\newcommand\PVI{\Peq{VI}}
\newcommand\MOD{\mathop{\mathrm{mod}}\nolimits}
\newcommand\lie{\mathrm}
\newcommand\ot{\otimes}
\newcommand\diag{\mathop{\mathrm{diag}}\nolimits}
%%%%%%%%%%%%%%%%%%%%%%%%%%%%%%%%%%%%%%%%%%%%%%%%%%%%%%%%%%%%%%%%%%%%%%%%%%%%
%\newcommand\N{{\mathbb N}} % natural numbers
\newcommand\Z{{\mathbb Z}} % rational integers
\newcommand\F{{\mathbb F}} % finite field
\newcommand\Q{{\mathbb Q}} % rational numbers
\newcommand\R{{\mathbb R}} % real numbers
\newcommand\C{{\mathbb C}} % complex numbers
%\renewcommand\P{{\mathbb P}} % projective spaces
%%%%%%%%%%%%%%%%%%%%%%%%%%%%%%%%%%%%%%%%%%%%%%%%%%%%%%%%%%%%%%%%%%%%%%%%%%%%
%
% theorem environments
%
\theoremstyle{plain} % bold header and slanted body
%\theoremstyle{definition} % bold header and normal body
\newtheorem{theorem}{Theorem}
\newtheorem*{theorem*}{Theorem}
\newtheorem{prop}[theorem]{Proposition}
\newtheorem*{prop*}{Proposition}
\newtheorem{lemma}[theorem]{Lemma}
\newtheorem*{lemma*}{Lemma}
\newtheorem{cor}[theorem]{Corollary}
\newtheorem*{cor*}{Corollary}
\newtheorem{example}[theorem]{Example}
\newtheorem*{example*}{Example}
\newtheorem{axiom}[theorem]{Axiom}
\newtheorem*{axiom*}{Axiom}
\newtheorem{problem}[theorem]{Problem}
\newtheorem*{problem*}{Problem}
\newtheorem{summary}[theorem]{Summary}
\newtheorem*{summary*}{Summary}
\newtheorem{guide}[theorem]{Guide}
\newtheorem*{guide*}{Guide}
%
\theoremstyle{definition} % bold header and normal body
\newtheorem{definition}[theorem]{Definition}
\newtheorem*{definition*}{Definition}
%
%\theoremstyle{remark} % slanted header and normal body
\theoremstyle{definition} % bold header and normal body
\newtheorem{remark}[theorem]{Remark}
\newtheorem*{remark*}{Remark}
%
\numberwithin{theorem}{section}
\numberwithin{equation}{section}
\numberwithin{figure}{section}
\numberwithin{table}{section}
%
% refs
%
\newcommand\secref[1]{Section \ref{#1}}
\newcommand\theoremref[1]{Theorem \ref{#1}}
\newcommand\propref[1]{Proposition \ref{#1}}
\newcommand\lemmaref[1]{Lemma \ref{#1}}
\newcommand\corref[1]{Corollary \ref{#1}}
\newcommand\exampleref[1]{Example \ref{#1}}
\newcommand\axiomref[1]{Axiom \ref{#1}}
\newcommand\problemref[1]{Problem \ref{#1}}
\newcommand\summaryref[1]{Summary \ref{#1}}
\newcommand\guideref[1]{Guide \ref{#1}}
\newcommand\definitionref[1]{Definition \ref{#1}}
\newcommand\remarkref[1]{Remark \ref{#1}}
%
\newcommand\figureref[1]{Figure \ref{#1}}
\newcommand\tableref[1]{Table \ref{#1}}
%
% proof environment without \qed
%
\makeatletter
\renewenvironment{proof}[1][\proofname]{\par
%\newenvironment{Proof}[1][\Proofname]{\par
  \normalfont
  \topsep6\p@\@plus6\p@ \trivlist
  \item[\hskip\labelsep{\bfseries #1}\@addpunct{\bfseries.}]\ignorespaces
}{%
  \endtrivlist
}
\renewcommand{\proofname}{Proof}
%\newcommand{\Proofname}{Proof}
\makeatother
%
% \qed
%
\makeatletter
\def\BOXSYMBOL{\RIfM@\bgroup\else$\bgroup\aftergroup$\fi
  \vcenter{\hrule\hbox{\vrule height.85em\kern.6em\vrule}\hrule}\egroup}
\makeatother
\newcommand{\BOX}{%
  \ifmmode\else\leavevmode\unskip\penalty9999\hbox{}\nobreak\hfill\fi
  \quad\hbox{\BOXSYMBOL}}
\renewcommand\qed{\BOX}
%\newcommand\QED{\BOX}
%%%%%%%%%%%%%%%%%%%%%%%%%%%%%%%%%%%%%%%%%%%%%%%%%%%%%%%%%%%%%%%%%%%%%%%%%%%%
\begin{document}
%%%%%%%%%%%%%%%%%%%%%%%%%%%%%%%%%%%%%%%%%%%%%%%%%%%%%%%%%%%%%%%%%%%%%%%%%%%%
\title{\TITLE}
\author{\AUTHOR}
\date{\DATE}
\maketitle
\begin{abstract}
  \ABSTRACT
\end{abstract}
\tableofcontents
%%%%%%%%%%%%%%%%%%%%%%%%%%%%%%%%%%%%%%%%%%%%%%%%%%%%%%%%%%%%%%%%%%%%%%%%%%%%
\setcounter{section}{-1} % First section number = 0

\section{Introduction}

In \cite{Kuroki2008}, 
the author canonically  
the birational Weyl group action 
arising from a nilpotent Poisson algebra 
proposed by Noumi and Yamada in \cite{NY0012028}
and also constructed its $q$-difference deformation. 
At that time he was not able to quantize the $\tau$-functions 
generated by the birational Weyl group action.
After that, in \cite{Kuroki2012a}, he succeeded 
in quantizing the $\tau$-functions
and showed the regularity (or polynomiality) 
of the quantum $\tau$-functions.
However he did not mention the Lax and Sato-Wilson formalisms 
for the quantum birational Weyl group action.  

In this paper, we shall present the Lax and the Sato-Wilson formalisms
for the $q$-difference version of the quantum birational Weyl group actions
of type $A$.  
First we shall construct the Lax and the Sato-Wilson formalisms 
for the quantum birational Weyl group action of type $A_{n-1}$ 
(\secref{sec:A_{n-1}}).
Second, taking the inductive limit of them,
we shall get the formalisms for the action of type $A_\infty$
(\secref{sec:A_{infinity}}).  
Third, by the $n$-periodic reduction, we shall also obtain the formalisms 
for the actions of type $A^{(1)}_{n-1}$ for $n\geqq 3$
(\secref{sec:A^{(1)}_{n-1}}).
Fourth, we shall describe the formalisms of the action of type $A^{(1)}_1$
(\secref{sec:A^{(1)}_1}).

%%%%%%%%%%%%%%%%%%%%%%%%%%%%%%%%%%%%%%%%%%%%%%%%%%%%%%%%%%%%%%%%%%%%%%%%%%%%%%

\paragraph{Notation and Conventions.}
When $X = \sum_k a_k\ot b_k$, we set 
$X^{12}=\sum_k a_k\ot b_k\ot 1$, 
$X^{13}=\sum_k a_k\ot 1\ot b_k$, and
$X^{23}=\sum_k 1\ot a_k\ot b_k$.
The $q$-numbers, the $q$-factorials, and the $q$-binomial coefficients are defined by
\begin{align*}
 &
 [a]_q = \frac{q^a-q^{-a}}{q-q^{-1}}, \qquad
 [k]_q! = [1]_q[2]_q\cdots[k]_q, \quad
 \\ &
 \qbinom{a}{k}_q =
 \frac{[a]_q[a-1]_q[a-2]_q\cdots[a-k+1]_q}{[k]_q!}
 \quad  (k\in\Z_{\geqq0}).
\end{align*}
The $q$-commutator is given by $[A,B]_q=AB-qBA$.
The associative algebra is always with unit $1$. 
The matrix units are denoted by $E_{ij}$.
We shall often denote the unit matrix by $1$.
For any algebra homomorphism $f$ from an algebra $R$ to an algebra $R'$,
we shall denote by the same symbol $f$
the induced mapping from the set of matrices over $R$
to the set of matrices over $R'$.
If $a$ and $s$ are mutually commutative elements of an algebra 
and $s$ is invertible, then we denote $s^{-1}a=as^{-1}$ 
by the notation of a fraction, $a/s$.

%%%%%%%%%%%%%%%%%%%%%%%%%%%%%%%%%%%%%%%%%%%%%%%%%%%%%%%%%%%%%%%%%%%%%%%%%%%%

\paragraph{Acknowledgements.}
\ACKNOWLEDGEMENTS

%%%%%%%%%%%%%%%%%%%%%%%%%%%%%%%%%%%%%%%%%%%%%%%%%%%%%%%%%%%%%%%%%%%%%%%%%%%%

\section{Quantum birational Weyl group action of type $A_{n-1}$}
\label{sec:A_{n-1}}

%%%%%%%%%%%%%%%%%%%%%%%%%%%%%%%%%%%%%%%%%%%%%%%%%%%%%%%%%%%%%%%%%%%%%%%%%%%%

\subsection{Quantum algebras of type $A_{n-1}$}
\label{sec:quantum-alg-A_{n-1}}

In this subsection, we deal with the lower triangular part of 
the quantum group of type $A_{n-1}$ for a positive integer $n$.

We define the $R$-matrix of type $A_{n-1}$ by
\begin{equation}
 R 
 = \sum_i q E_{ii}\ot E_{ii}
 + \sum_{i\ne j} E_{ii}\ot E_{jj}
 + \sum_{i<j} (q-q^{-1}) E_{ij}\ot E_{ji},
 \label{eq:def-R}
\end{equation}
where $i$ and $j$ run through $1,2,\ldots,n$.
Then the $R$-matrix $R$ satisfies the Yang-Baxter equation
$R^{12}R^{13}R^{23} = R^{23}R^{13}R^{12}$.
Let $L$ be the upper triangular matrix of size $n$ with 
non-commutative indeterminate entries $L_{ij}$,
namely $L = \sum_{i\leqq j} L_{ij} E_{ij}$.

Let $\B_-$ be the associative algebra over $\C(q)$ 
generated by $L_{ij}$ ($1\leqq i\leqq j\leqq n$) 
and $L_{ii}^{-1}$ ($1\leqq i\leqq n$)
with the following fundamental relations:
\begin{equation}
 R L^1L^2 = L^2L^1R,
 \quad L_{ii} L_{ii}^{-1} = L_{ii}^{-1}L_{ii} = 1,
 \label{eq:RLL=LLR}
\end{equation}
where $L^1=L\ot 1$ and $L^2=1\ot L$. 
Explicitly the relation $R L^1L^2=L^2L^1R$ is equivalent to
the following conditions:
\begin{align*}
 &
 i\leqq j<k\leqq l \ \text{or}\ k<i\leqq j<l 
 \implies L_{ij}L_{kl} = L_{kl}L_{ij},
 \\ &
 k<i\leqq j \implies L_{ij}L_{kj}=q L_{kj}L_{ij},
 \\ &
 i\leqq j<l \implies L_{ij}L_{il}=q^{-1}L_{il}L_{ij},
 \\ &
 k<i\leqq l<j \implies 
 L_{ij}L_{kl} - L_{kl}L_{ij} = (q-q^{-1}) L_{kj}L_{il}.
\end{align*}
The matrix $L$ is called {\em the $L$-operator of type $A_{n-1}$}.

\begin{remark}
 When $L=\sum_{i,j} L_{ij} E_{ij}$ is not upper triangular,
 the relation $R L^1L^2=L^2L^1R$ is equivalent to
 the following relations:
 \begin{align*}
  &
  L_{il}L_{kj} = L_{il}L_{kj}, \quad
  L_{ij}L_{kl}-L_{kl}L_{ij} = (q-q^{-1})L_{il}L_{kj},
  \\ &
  L_{ij}L_{il} = q L_{il}L_{ij}, \quad
  L_{ij}L_{kj} = q L_{kj}L_{ij}
  \qquad\qquad\qquad\quad (k<i, l<j).
 \end{align*}
 These relations are summarized in the following diagram:
 \begin{equation*}
  \begin{array}{ccc}
    L_{kl}   & \leftarrow & L_{kj}  \\
    \uparrow & \nwarrow   & \uparrow \\
    L_{il}   & \leftarrow & L_{ij} \\
  \end{array},
  \qquad (k<i, l<j),
 \end{equation*}
 where the vertical and horizontal arrows stand for 
 the relations of type $L_{ij}L_{il} = q L_{il}L_{ij}$, 
 the sloping arrow stands for the relation 
 $L_{ij}L_{kl}-L_{kl}L_{ij} = (q-q^{-1})L_{il}L_{kj}$, 
 and the other combination without a connecting arrow commutes.
 The fundamental relations of $\B_-$ are 
 the specialization of these relations to the case where $L$ is upper triangular.
 \qed
\end{remark}

Denote by $D_L$ the diagonal part of $L$, namely $D_L=\sum_i L_{ii}E_{ii}$.
We define the unipotent upper triangular matrix $\tL$ by $\tL=D_L^{-1}L$
and denote its $(i,j)$-entry by $f_{ij}$:
\begin{equation*}
 \tL = D_L^{-1}L = 1 + \sum_{i<j} f_{ij}E_{ij}, \quad 
 f_{ij} = L_{ii}^{-1}L_{ij}.
\end{equation*}
We call the matrix $\tL$ {\em the quasi $L$-operator of type $A_{n-1}$}.

Let $\N_-$ be the subalgebra of $\B_-$ generated by $f_{ij}$ ($i<j$).
We define $f_i\in\N_-$ by
\begin{equation}
 f_i = (q-q^{-1})^{-1} f_{i,i+1}.
 \label{eq:f_i}
\end{equation}
Then we have
\begin{equation}
 f_j f_{ij} - q^{-1} f_{ij} f_j = f_{i,j+1} \quad (i<j).
 \label{eq:f_{ij}f_j}
\end{equation}
and hence the algebra $\N_-$ is generated by $f_i$'s.
Moreover they satisfy the following $q$-Serre relations:
\begin{equation}
 f_i^2 f_{i\pm1} - (q+q^{-1}) f_if_{i\pm1}f_i + f_{i\pm1}f_i^2=0, \qquad
 f_if_j=f_jf_i \quad (|i-j|\geqq 2).
 \label{eq:f_i-q-Serre}
\end{equation}
Under the condition where $L_{ii}$'s are invertible, 
we can derive the fundamental relations of $\B_-$ from
\eqref{eq:f_i}, \eqref{eq:f_{ij}f_j}, \eqref{eq:f_i-q-Serre},  
and the following relations:
\begin{equation*}
 L_{ii}f_{ij}=q^{-1}f_{ij}L_{ii}, \quad
 L_{jj}f_{ij}=q     f_{ij}L_{jj}, \quad
 L_{kk}f_{ij}=f_{ij}L_{kk} \quad
 (i<j, k\ne i,j).
\end{equation*}
Therefore the algebra $\N_-$ is isomorphic to  
the upper triangular part $U_q(\n_-)$ 
of the $q$-difference deformation $U_q(\g)$ of
the universal enveloping algebra $U_q(\g)$ 
of the Kac-Moody algebra $\g$ of type $A_{n-1}$.
(See also Section II of \cite{DF}.)

In the sequel, we identify $\N_-$ with $U_q(\n_-)$
and denote $\N_-=U_q(\n_-)$ simply by $U_-$. 
We call the Chevalley generators $f_i$ of $U_-$ 
{\em the quantum dependent variables}.

We denote by $Q(R)$ the skew field of fractions of an Ore domain $R$.
The algebra $U_-$ is a Noetherian domain 
(Sections 7.3 and 7.4 of \cite{Jos-1995}).
Therefore $U_-$ is an Ore domain.
We obtain the skew field $K=Q(U_-)$ of fractions of $U_-$. 

%%%%%%%%%%%%%%%%%%%%%%%%%%%%%%%%%%%%%%%%%%%%%%%%%%%%%%%%%%%%%%%%%%%%%%%%%%%%

\subsection{Quantum birational Weyl group action}
\label{sec:Weyl-A_{n-1}}

The Weyl group $W=W(A_{n-1})$ of type $A_{n-1}$ is defined to be the group
generated by $s_1,s_2,\ldots,s_{n-1}$ with the following fundamental relations:
\begin{equation}
 s_i s_{i+1}s_i = s_{i+1}s_i s_{i+1}, \qquad
 s_i s_j = s_j s_i \quad (|i-j|\geqq 2), \qquad
 s_i^2 = 1.
 \label{eq:Weyl}
\end{equation}
Then $W$ is isomorphic to the permutation group of $\{1,2,\ldots,n\}$ 
by sending $s_i$ to the transposition $(i,i+1)$.

Let $\Qv$ be the free $\Z$-module generated by $\eps_1,\eps_2,\ldots,\eps_n$
and put $P=\Hom(\Qv,\Z)$.
Denote the canonical pairing between $\Qv$ and $P$ by $\bra\,\ ,\ \ket$.
We call $\Qv$ the coroot lattice and $P$ the weight lattice.
Denote the dual basis of $\{\epsv_i\}_{i=1}^n$ by $\{\eps_i\}_{i=1}^n$.
We define the simple coroots $\av_i$ ($i=1,\ldots,n-1$) by $\av_i=\epsv_i-\epsv_{i+1}$
and the simple roots $\alpha_i$ ($i=1,\ldots,n-1$) by $\alpha_i=\eps_i-\eps_{i+1}$.
The fundamental weights $\Lambda_i$ ($i=0,1,\ldots,n$) 
and the Weyl vector $\rho$ are given by
$\Lambda_i = \eps_1+\eps_2+\cdots+\eps_i$ and $\rho=\sum_{i=1}^{n-1}\Lambda_i$.
Then we have
\begin{equation}
 \bra\av_i,\alpha_j\ket =
 \begin{cases}
   2 & \text{if $i=j$},  \\
  -1 & \text{if $|i-j|=1$}, \\
   0 & \text{otherwise}, \\ 
 \end{cases}
 \quad
 \bra\av_i,\Lambda_j\ket = \delta_{ij},
 \quad
 \bra\av_i,\rho\ket = 1.
 \label{eq:a_{ij}}
\end{equation}
The matrix $[a_{ij}]_{i=1}^{n-1}$ is defined by $a_{ij}=\bra\av_i,\alpha_j\ket$ and
called the Cartan matrix of type $A_{n-1}$.

The Weyl group acts on $\Qv$ and $P$ by
\begin{equation*}
 s_i(\beta) = \beta - \bra\beta,\alpha_i\ket\av_i \quad (\beta\in\Qv), \qquad
 s_i(\lambda) = \lambda - \bra\av_i,\lambda\ket\alpha_i \quad (\lambda\in P).
 \label{eq:Weyl-Q-P}
\end{equation*}
Then the Weyl group actions on $\Qv$ and $P$ preserve the canonical pairing
between them.
 
Denote by $K[q^\beta|\beta\in\Qv]$ the associative algebra over 
the skew field $K=Q(U_-)$ generated by symbols $q^\beta$
for $\beta\in\Qv$ with the following fundamental relations:
\begin{equation*}
 q^\beta f_i = f_i q^\beta, \quad
 q^\beta q^\gamma = q^{\beta+\gamma}, \quad
 q^0 = 1 \quad (\beta,\gamma,0\in\Qv).
\end{equation*}
It coincides with the Laurent polynomial ring over $K$ 
generated by $\{q^{\pm\epsv_i}\}_{i=1}^n$ 
and hence is an Ore domain.
Denote by $K^\pa$ the skew filed of fractions 
of $K[q^\beta|\beta\in\Qv]=K[q^{\pm\epsv_1},\ldots,q^{\pm\epsv_n}]$.
We call $q^{\epsv_i}$'s {\em the parameter variables}. 
The algebra $U_-^\pa$ %(resp.\ $\tU_-^\pa$) 
is defined to be the subalgebra of $K^\pa$
generated by $U_-$ %(resp.\ $\tU_-$) 
and $\{q^{\pm\epsv_i}\}_{i=1}^n$.
Here $(\ )^\pa$ stands for {\em an algebra with parameter variables}.

For each $\lambda\in P$, 
we define the algebra homomorphism $\phi_\lambda:U_-^\pa\to U_-$ 
by $\phi_\lambda(q^\beta)=q^{\bra\beta,\lambda\ket}$ ($\beta\in\Qv$) 
and $\phi_\lambda(f_i)=f_i$.
Using Theorem 2.1 of \cite{S-1971}, we can prove that 
$S_\lambda = \{\, a\in U_-^\pa \mid \phi_\lambda(a)\ne 0\,\}$
is an Ore set in $U_-^\pa$. Therefore we obtain the localization
$U_-^\pa[S_\lambda^{-1}]\subset K^\pa$.
Define the algebra $U_{-,(P)}^\pa$ to be the intersection 
of $U_-[S_\lambda^{-1}]$ for all $\lambda\in P$.
(For details, see \cite{Kuroki2012a}.)
The algebra homomorphism $\phi_\lambda$ is uniquely extended to
the algebra homomorphism $\phi_\lambda:U_{-,(P)}^\pa\to K$,
which substitutes $\bra\beta,\lambda\ket$ in $\beta\in\Qv$.

Let $D(K^\pa)$ be the associative algebra generated by $K^\pa$ and $\{\tau^\lambda\}_{\lambda\in P}$
with the following defining relations:
\begin{align}
 &
 \tau^\lambda\tau^\mu = \tau^{\lambda+\mu}, \quad
 \tau^0 = 1 \quad (\lambda,\mu,0\in P),
 \notag
 \\ &
 \tau^\lambda f_i = f_i \tau^\lambda, \quad
 \tau^\lambda q^\beta = q^{\beta + \bra\beta,\lambda\ket} \tau^\lambda \quad
 (\lambda\in P, \beta\in\Qv). 
 \label{eq:tau-def}
\end{align}
Then we have $D(K^\pa) = \bigoplus_{\lambda\in P} K^\pa \tau^\lambda$.
The symbol $D({\ }^\pa)$ stands for 
{\em a difference operator algebra with respect to the parameter variables}.
We define {\em the quantum $\tau$-variables} $\tau_i$ 
by $\tau_i=\tau^{\Lambda_i}$
($i=0,1\ldots,n$) and call $\tau^\lambda$'s 
{\em the quantum Laurent $\tau$-monomials}.
(Note that $\tau_0=1$.)
Let $D(U_-^\pa)$ (resp.\ $D(U_{-,(P)}^\pa)$) 
be the subalgebra of $D(K^\pa)$ generated by 
$U_-^\pa$ (resp.\ $U_{-,(P)}^\pa$) and $\{\tau^\lambda\}_{\lambda\in P}$.

The following proposition is a special case of the general result 
of \cite{Kuroki2012a} (see also \cite{Kuroki2008}).

\begin{prop}
\label{prop:QWGA-A_{n-1}}
 For each $i=1,2,\ldots,n-1$, 
 the algebra automorphism $\bs_i$ of $D(U_{-,(P)}^\pa)$ can be given by
 \begin{align*}
  &
  \bs_i(f_j)
  = q^{-\av_i} f_j + [\av_i]_q [f_i,f_j]_{q^{-1}} f_i^{-1}
  \quad (j=i\pm1),
 \quad
  \bs_i(f_j) = f_j 
  \quad (j\ne i\pm 1), 
 \\ &
  \bs_i(q^\beta) = q^{s_i(\beta)} 
  \quad (\beta\in\Qv),
 \qquad
  \bs_i(\tau^\lambda) 
  = f_i^{\bra\av_i,\lambda\ket}\tau^{s_i(\lambda)} 
  \quad (\lambda\in P).
 \end{align*}
 Then the mapping $s_i\mapsto\bs_i$ defines the Weyl group actions 
 on $D(U_{-,(P)}^\pa)$, $U_{-,(P)}^\pa$, $D(K^\pa)$, and $K^\pa$.
 \qed
\end{prop}

\begin{definition}
 The Weyl group actions 
 on $D(U_{-,(P)}^\pa)$, $U_{-,(P)}^\pa$, $D(K^\pa)$, and $K^\pa$ 
 obtained by \propref{prop:QWGA-A_{n-1}}
 are called {\em the quantum birational Weyl group actions of type $A_{n-1}$}.
 Denote the action of $w\in W$ on $x\in D(K^\pa)$ by $w(x)$.
 \qed 
\end{definition}

\begin{remark}
For $w\in W$, we define the algebra automorphism $\tw$ 
of $D(K^\pa)$ by 
\begin{equation*}
 \tw(f_i) = f_i, \quad 
 \tw(q^\beta) = q^{w(\beta)}, \quad
 \tw(\tau^\lambda) = \tau^{w(\lambda)} \quad
 (\beta\in\Qv, \lambda\in P).
\end{equation*}
This defines the Weyl group action on $D(K^\pa)$, 
which is called {\em the tilde action}.
In \cite{Kuroki2008} and \cite{Kuroki2012a}, the author constructs 
the quantum birational Weyl group action by
\begin{equation*}
 s_i(x) = f_i^{\av_i} \ts_i(x) f_i^{-\av_i}
 \quad (x\in D(K^\pa)),
\end{equation*}
where $f_i^{\av_i}$ denotes a fractional power of $f_i$.
For details, see \cite{Kuroki2012a}.
\qed
\end{remark}

\begin{remark}
\label{remark:fractional-power}
 Applying the main result of \cite{Kuroki2012a} to the quantum birational
 Weyl group action of type $A_{n-1}$, 
 we obtain that $w(\tau_i)\in D(U_-^\pa)$ for $w\in W$ and $i=1,2,\ldots,n-1$.
 More precisely, 
 there exists a unique non-commutative polynomial $\phi_{i,w}$ 
 in $\{f_i, q^{\pm\av_i}\}_{i=1}^{n-1}$ 
 such that $w(\tau_i)=\phi_{i,w}\tau^{w(\Lambda_i)}$.
 Such a result is called {\em the regularity of the quantum $\tau$-functions}
 in \cite{Kuroki2012a}.
 \qed 
\end{remark}

%%%%%%%%%%%%%%%%%%%%%%%%%%%%%%%%%%%%%%%%%%%%%%%%%%%%%%%%%%%%%%%%%%%%%%%%%%%%

\subsection{Lax formalism}
\label{sec:Lax-A_{n-1}}

For the notational simplicity, we shall denote $q^{-\epsv_i}$ by $t_i$
and $q^{-\av_i}$ by $a_i$:
\begin{equation*}
 t_i = q^{-\epsv_i}, \quad 
 a_i = q^{-\av_i} = t_i/t_{i+1}.
\end{equation*}
We should be careful of the minus signs in the exponents. 
Then we have
\begin{align}
  &
  s_i(f_j)
  = a_i      f_j + \frac{a_i^{-1}-a_i}{q-q^{-1}} [f_i,f_j]_{q^{-1}} f_i^{-1}
% \notag
% \\ &
% \hphantom{s_i(f_j)}
  = a_i^{-1} f_j + \frac{a_i^{-1}-a_i}{q-q^{-1}} [f_i,f_j]_q        f_i^{-1}
 \notag
 \\ &
  \hphantom{s_i(f_j)}
  = \frac{q a_i-q^{-1}a_i^{-1}}{q-q^{-1}} f_j 
  + \frac{a_i^{-1}-a_i}{q-q^{-1}} f_i f_j f_i^{-1}
  \qquad (j=i\pm1), 
 \label{eq:s_i(f_j)-1}
 \\ &
  s_i(f_j) = f_j 
  \qquad (j\ne i\pm 1),
 \label{eq:s_i(f_j)-2}
 \\ &
  s_i(t_i) = t_{i+1}, \quad
  s_i(t_{i+1}) = t_i, \quad
  s_i(t_j) = t_j \quad (j\ne i,i+1),
 \label{eq:s_i(t_j)}
 \\ &
  s_i(\tau_i) 
  = f_i\frac{\tau_{i-1}\tau_{i+1}}{\tau_i} 
  \qquad (i=1,\ldots,n-1),
 \notag
 \\ & 
  s_i(\tau_j) = \tau_j
  \qquad (i\ne j).
 \notag 
\end{align}
These formulas uniquely characterize the action of $s_i$ on $D(K^\pa)$.

Define the diagonal matrix $D_t$ of the parameter variables by 
\begin{equation*}
 D_t = \sum_{i=1}^n t_i E_{ii} = \diag(t_i)_{i=1}^n.
\end{equation*}
Recall that, in \secref{sec:quantum-alg-A_{n-1}}, 
the quasi $L$-operator $\tL$ is defined by $\tL=D_L^{-1}L$, 
where $D_L$ is the diagonal part of the $L$-operator $L$.
We introduce {\em the $M$-operator} by
\begin{equation*}
 M = D_t \tL D_t 
   = D_t^2 + \sum_{i<j}t_i t_j f_{ij}E_{ij}
   = [m_{ij}]_{i,j=1}^n.
\end{equation*}
Define {\em the $G$-matrices} $G_i$ ($i=1,2,\ldots,n-1$) by
\begin{equation*}
 G_i = 1 + g_i E_{i+1,i}, \quad
 g_i = \frac{t_i^2-t_{i+1}^2}{m_{i,i+1}}
     = \frac{a_i-a_i^{-1}}{f_{i,i+1}}
     = -\frac{[\av_i]_q}{f_i}.
\end{equation*}
Here we should remember that $a_i=q^{-\av_i}$ (not $a_i=q^{\av_i}$).
Then we can obtain the following theorem by straightforward calculations.

\begin{theorem}[Lax formalism for type $A_{n-1}$]
\label{theorem:Lax-A_{n-1}}
 We have
 \begin{equation*}
   s_i(M)=G_i M G_i^{-1} \quad \text{for $i=1,2,\ldots,n-1$}.
 \end{equation*}
 These formulas with \eqref{eq:s_i(t_j)} uniquely characterize 
 the quantum birational Weyl group action of type $A_{n-1}$ 
 on $K^\pa$.
\end{theorem}

\begin{proof}
Using \eqref{eq:f_i}, \eqref{eq:f_{ij}f_j}, and \eqref{eq:f_i-q-Serre},
we can write down the explicit formulas for 
the actions of $s_i$ on $f_{kl}$'s as below:
\begin{align*}
 &
 s_i(f_{ki}) = a_i f_{ki} - (a_i-a_i^{-1})f_{k,i+1}f_{i,i+1}^{-1}
 \qquad (k<i),
 \\ &
 s_i(f_{k,i+1}) = a_i^{-1}f_{k,i+1}
 \qquad (k<i),
 \\ &
 s_i(f_{i+1,l}) = a_i^{-1}f_{i+1,l} + (a_i-a_i^{-1})f_{i,i+1}^{-1}f_{il}
 \qquad (l>i+1),
 \\ &
 s_i(f_{il}) = a_i f_{il}
 \qquad (l>i+1),
 \\ &
 s_i(f_{kl}) = f_{kl} \qquad \text{for other $f_{kl}$}.
\end{align*}
These formulas with \eqref{eq:s_i(t_j)}
immediately lead to $s_i(M)=G_i M G_i^{-1}$.
The last statement of the theorem is clear.
\qed
\end{proof}

\begin{remark}[The $q\to 1$ limit]
 We shall sketch how to obtain 
 the limit of \theoremref{theorem:Lax-A_{n-1}} as $q\to 1$.
 Do not confuse it with a classical limit.
 Define $x_{ji}$ by $f_{ij}=(q-q^{-1})x_{ji}$ ($i<j$).
 Then we have $[x_{j+1,j}, x_{ji}]_q = x_{j+1,i}$ ($i<j$).
 Therefore, after taking the limit as $q\to 1$, 
 we obtain the relations 
 $[x_{ij}, x_{kl}] = \delta_{jk}x_{il}-\delta_{li}x_{kj}$
 ($i<j$, $k<l$),
 which are the fundamental relations of the matrix units.
 Setting $q=e^{\hbar/2}$, we have
 \begin{equation*}
   M = 1 + \hbar \M + O(\hbar^2), \quad 
   G_i = \G_i + O(\hbar), 
 \end{equation*}
 where $\M$ and $\G_i$ are defined by
 \begin{equation*}
  \M = -\sum_i \epsv_i E_{ii} + \sum_{i<j} x_{ji} E_{ij}, \quad
  \G_i = 1 - \frac{\epsv_i-\epsv_{i+1}}{x_{i+1,i}} E_{i+1,i}.
 \end{equation*}
 The limit of the quantum birational Weyl group
 action of type $A_{n-1}$ as $q\to 1$ can be written in the form 
 $s_i(\M)=\G_i \M \G_i^{-1}$.
 This is the straightforward canonical quantization of Theorem 7.1 
 in \cite{Noumi}.
 \theoremref{theorem:Lax-A_{n-1}} is both the quantization
 and the $q$-difference analogue of Theorem 7.1 in \cite{Noumi}.
 \qed
\end{remark}

\begin{remark}[Quantum unipotent crystal structure on $M$]
 In this remark, we shall deform the algebra automorphisms 
 $\bs_i$ ($i=1,2,\ldots,n-1$) and 
 construct a quantum unipotent crystal structure on the $M$-operator.

 For any element $c$ of the center of $(K^\pa)^\times$, 
 we can define the algebra automorphism $\be_i^c$ of $K^\pa$ 
 as follows.  We define the actions of $\be_i^c$ on $f_j$'s by
 \begin{align*}
  &
  \be_i^c(f_j) 
  = c f_j      + \frac{c^{-1}-c}{q-q^{-1}}[f_i,f_j]_{q^{-1}} f_i^{-1}
  =  c^{-1} f_j + \frac{c^{-1}-c}{q-q^{-1}}[f_i,f_j]_q        f_i^{-1}
  \\ & \hphantom{\be_i^c(f_j)}
  = \frac{q c-q^{-1}c^{-1}}{q-q^{-1}}f_j 
  + \frac{c^{-1}-c}{q-q^{-1}}f_jf_jf_j^{-1}
  \qquad (j=i\pm1),
  \\ &
  \be_i^c(f_j) = f_j \qquad (j\ne i\pm1).
 \end{align*}
 If we formally denote $c$ by $q^{-\gamma}$, then
 we have $\be_i^c(f_j)=f_i^\gamma f_j f_i^{-\gamma}$.
 (For the construction of fractional powers of $f_i$, see \cite{Kuroki2012a}.)
 The actions of $\be_i^c$ on $t_j$'s are given by
 \begin{equation*}
  \be_i^c(t_i) = c^{-1}t_i, \quad
  \be_i^c(t_{i+1}) = ct_{i+1}, \quad
  \be_i^c(t_j) = t_j \quad (j\ne i,i+1).
 \end{equation*}
 Equivalently, we set $\be_i^c(q^\beta)=c^{\bra\beta,\alpha_i\ket}q^\beta$
 for $\beta\in\Qv$.
 Then the specialization of $c$ at $q^{-\av_i}$ gives $\be_i^c = \bs_i$.
 Putting $\alpha_i(M) = m_{ii}/m_{i+1.i+1}$, 
 $\psi_i(M) = m_{i,i+1}/m_{ii}$, and
 $y_i(a) = \exp(aE_{i+1,i})$,
 we obtain
 \begin{equation*}
   \be_i^c(M) 
   = y_i\left(\frac{c^2-1}{\alpha_i(M)\psi_i(M)}\right)\cdot
     M \cdot
     y_i\left(\frac{c^{-2}-1}{\psi_i(M)}\right).
 \end{equation*}
 Therefore we can regard the upper triangular matrix $M$ 
 as a quantum unipotent crystal. 
 Compare the above formula with Equation (3.8) of \cite{BK2000}.
 \qed
\end{remark}

%%%%%%%%%%%%%%%%%%%%%%%%%%%%%%%%%%%%%%%%%%%%%%%%%%%%%%%%%%%%%%%%%%%%%%%%%%%%

\subsection{Sato-Wilson formalism}
\label{sec:Sato-Wilson-A_{n-1}}

Since $M$-operator is the upper triangular matrix with
mutually distinct diagonal entries, 
it can be uniquely diagonalized 
by the unipotent upper triangular matrix $U$:
\begin{equation*}
 M = U D_t^2 U^{-1}, \quad
 U = 1 + \sum_{i<j} u_{ij} E_{ij},
\end{equation*}
where $u_{ij}$'s are given by
\begin{equation}
 u_{ij} 
 = \sum_{r=1}^{j-i}(-1)^r
   \sum_{i=i_0<i_1<\cdots<i_r=j}
   \frac{m_{i_0i_1}m_{i_1i_2}\cdots m_{i_{r-1}i_r}}
        {(t_{i_0}^2-t_j^2)(t_{i_1}^2-t_j^2)\cdots(t_{i_{r-1}}^2-t_j^2)}
 \quad (i<j).
 \label{eq:u_{ij}}
\end{equation}
In particular, we have 
$u_{i,i+1} = -m_{i,i+1}/(t_i^2-t_{i+1}^2)=-g_i^{-1} = f_i/[\av_i]_q$. 

The uniqueness of $U$ and \theoremref{theorem:Lax-A_{n-1}} show that
$s_i(U)$ is written in the following form:
\begin{equation}
 s_i(U) = G_i U S_i^g, \quad
 S_i^g = g_i^{-1} E_{i,i+1} - g_i E_{i+1,i} + \sum_{k\ne i,i+1} E_{kk}.
 \label{eq:s_i(U)}
\end{equation}
In fact, the matrix $G_i U S_i^g$ is unipotent upper triangular 
and \theoremref{theorem:Lax-A_{n-1}} leads to
\begin{equation*}
 s_i(M) 
 = G_i M G_i^{-1}
 = G_i U D_t^2 (G_i U)^{-1}
 = G_i U S_i^g s_i(D_t^2) (G_i U S_i^g)^{-1}.
\end{equation*}
On the other hand, we have $s_i(M)=s_i(U)s_i(D_t^2)s_i(U)^{-1}$.
Therefore we obtain $s_i(U)=G_i U S_i^g$.
In order to show that $G_i U S_i^g$ is unipotent upper triangular, 
it is sufficient to calculate its $2$-by-$2$ part for $(i,i+1)$:
\begin{equation*}
 \begin{bmatrix}
   1   & 0 \\
   g_i & 1 \\ 
 \end{bmatrix}
 \begin{bmatrix}
   1   &  -g_i^{-1} \\
   0   &   1        \\
 \end{bmatrix}
 \begin{bmatrix}
   0   & g_i^{-1} \\
  -g_i &  0       \\
 \end{bmatrix}
 =
 \begin{bmatrix}
    1   &  -g_i^{-1} \\
   g_i  &  0         \\
 \end{bmatrix}
 \begin{bmatrix}
   0   & g_i^{-1} \\
  -g_i &  0       \\
 \end{bmatrix}
 =
 \begin{bmatrix}
   1   &  g_i^{-1} \\
   0   &   1        \\
 \end{bmatrix}. 
\end{equation*}

Define {\em the $z$-variables} by $z_i = \tau^{\eps_i}$ ($i=1,2,\ldots,n$) 
and the diagonal matrix $D_Z$ of the $z$-variables by
\begin{equation*}
 D_Z = \sum_{i=1}^n z_i E_{ii} = \diag(z_i)_{i=1}^n,
 \qquad z_i = \tau^{\eps_i} = \frac{\tau_i}{\tau_{i-1}}.
\end{equation*}
We define {\em the $Z$-operator} by
\begin{equation*}
 Z = U D_Z = D_Z + \sum_{i<j} u_{ij}z_j E_{ij} = [z_{ij}]_{i,j=1}^n.
\end{equation*}
Since $z_i t_j = q^{-\delta_{ij}} t_j z_i$, 
we have $M = q^2 Z D_t^2 Z^{-1}$. 
The actions of $s_i$ on $z_j$'s are explicitly written in the following form:
\begin{equation}
 s_i(z_i)= f_i z_{i+1}, \quad
 s_i(z_{i+1}) = f_i^{-1} z_i, \quad
 s_i(z_j) = z_j \quad (j\ne i,i+1).
 \label{eq:s_i(z_j)}
\end{equation}
Define the matrices $S_i$ ($i=1,2,\ldots,n-1$) by
\begin{equation*}
 S_i = 
 -[\av_i-1]_q^{-1} E_{i,i+1} + [\av_i+1]_q E_{i+1,i}
 + \sum_{k\ne i,i+1} E_{kk}.
\end{equation*}

\begin{theorem}[Sato-Wilson formalism for type $A_{n-1}$]
\label{theorem:Sato-Wilson-A_{n-1}}
 We have
 \begin{equation}
  s_i(Z) = G_i Z S_i, \quad 
  s_i(D_t) = S_i^{-1} D_t S_i 
  \quad \text{for $i=1,2,\ldots,n-1$}.
  \label{eq:s_i(Z)}
 \end{equation}
 These formulas uniquely characterize the whole of 
 the quantum birational Weyl group action of type $A_{n-1}$
 on $D(K^\pa)$.
\end{theorem}

\begin{proof}
 Equation \eqref{eq:s_i(t_j)} is equivalent to $s_i(D_t)=S_i D_t S_i^{-1}$.
 Because of \eqref{eq:s_i(U)}, 
 the formula $s_i(Z)=G_i Z S_i$ is equivalent to
 $s_i(D_Z) = (S_i^g)^{-1} D_Z S_i$ and hence equivalent to \eqref{eq:s_i(z_j)}.
 In order to show the last equivalence, it is sufficient to calculate 
 the $2$-by-$2$ part of $(S_i^g)^{-1} D_Z S_i$ for $(i,i+1)$:
 \begin{align*}
  &
  -g_i^{-1} z_{i+1} 
  = [\av_i]_q^{-1} f_i z_{i+1}
  = f_i z_{i+1} [\av_i+1]_q^{-1}
  = s_i(z_i)[\av_i+1]_q^{-1},
  \\ &
  g_i z_i 
  = - [\av_i]_q f_i^{-1} z_i
  = - f_i^{-1}z_i [\av_i-1]_q
  = - s_i(z_{i+1}) [\av_i-1]_q,
  \\ &
  \begin{bmatrix}
     0  & -g_i^{-1} \\
    g_i &   0       \\
  \end{bmatrix}
  \begin{bmatrix}
    z_i & 0       \\
     0  & z_{i+1} \\
  \end{bmatrix}
  \begin{bmatrix}
       0     & -[\av_i-1]^{-1} \\
   [\av_i+1] &       0         \\
  \end{bmatrix}
  \\ &
  =
  \begin{bmatrix}
     0      & -g_i^{-1} z_{i+1} \\
    g_i z_i &   0       \\
  \end{bmatrix}
  \begin{bmatrix}
       0     & -[\av_i-1]^{-1} \\
   [\av_i+1] &       0         \\
  \end{bmatrix}
  =
  \begin{bmatrix}
   s_i(z_i) &    0         \\
      0     & s_i(z_{i+1}) \\
  \end{bmatrix}.
 \end{align*}
 We have shown Equation \eqref{eq:s_i(Z)}.
 Equation \eqref{eq:s_i(Z)} implies $s_i(M)=G_i M G_i^{-1}$.
 Therefore \theoremref{theorem:Lax-A_{n-1}} leads to 
 the last statement of the theorem.
 \qed
\end{proof}

\begin{remark}
 \theoremref{theorem:Sato-Wilson-A_{n-1}} is both the quantization
 and the $q$-difference analogue of Theorem 7.3 in \cite{Noumi}. 
 In the classical (and hence commutative) case, 
 the Sato-Wilson formalism of the birational
 Weyl group action of type $A_{n-1}$ immediately leads to 
 the regularity of the classical $\tau$-functions, 
 because a determinant is a polynomial in its entries.
 But, in the quantum (and hence non-commutative) case, 
 the Sato-Wilson formalism does not 
 immediately lead to the regularity of the quantum $\tau$-fucntions,
 because a non-commutative determinant is 
 not always a non-commutative polynomial 
 but a non-commutative rational function in its entries, 
 owing to the theory of quasi-determinants (\cite{GR}, \cite{GGRW}).
 In \cite{Kuroki2012a}, the regularity of the quantum $\tau$-functions
 are shown by the theory of the translation functors 
 in representation theory.
 It is an open problem whether or not the quantum $\tau$-functions 
 admit non-commutative polynomial-type determinant representations. 
 \qed
\end{remark}

%%%%%%%%%%%%%%%%%%%%%%%%%%%%%%%%%%%%%%%%%%%%%%%%%%%%%%%%%%%%%%%%%%%%%%%%%%%%
%%%%%%%%%%%%%%%%%%%%%%%%%%%%%%%%%%%%%%%%%%%%%%%%%%%%%%%%%%%%%%%%%%%%%%%%%%%%

\section{The cases of type $A_\infty$ and of type $A^{(1)}_{n-1}$}
\label{sec:A_{infinity}-A^{(1)}_{n-1}}

%%%%%%%%%%%%%%%%%%%%%%%%%%%%%%%%%%%%%%%%%%%%%%%%%%%%%%%%%%%%%%%%%%%%%%%%%%%%

\subsection{The case of type $A_\infty$}
\label{sec:A_{infinity}}

We define the $R$-matrix $R$ of type $A_\infty$ 
by the formula \eqref{eq:def-R} in which $E_{ij}$'s denote
the matrix units of size $\Z$ and $i,j$ run through all integers.
Then $R$-matrix $R$ satisfies the Yang-Baxter equation.
Let $L$ be the upper triangular matrix of size $\Z$ 
with non-commutative indeterminate entries $L_{ij}$, 
namely $L=\sum_{i\leqq j} L_{ij} E_{ij}$.
We call $L$ the $L$-operator of type $A_\infty$.

Let $\B_{-,\infty}$ be the associative algebra over $\C(q)$ generated by 
$L_{ij}$ ($i,j\in\Z$ with $i\leqq j$) and $L_{ii}^{-1}$ ($i\in\Z$)
with fundamental relation \eqref{eq:RLL=LLR}.
Denote by $D_L$ the diagonal part of $L$, 
namely set $D_L=\sum_{i\in\Z}L_{ii}E_{ii}$. 
We define the quasi $L$-operator of type $A_\infty$ 
by $\tL=D_L^{-1}L=1+\sum_{i<j}f_{ij}E_{ij}$. 
Let $\N_{-,\infty}$ be the subalgebra 
of $\B_{-,\infty}$ generated by $f_{ij}$ ($i<j$).
We define $f_i\in\N_{-,\infty}$ by $f_i=(q-q^{-1})^{-1}f_{i,i+1}$.

The algebra $\N_{-,\infty}$ can be regarded as an inductive limit 
as $n\to\infty$ of the algebras $\N_-$ of type $A_{n-1}$ 
defined in \secref{sec:A_{n-1}}.
Therefore $\N_{-,\infty}$ is the algebra generated by $f_i$ ($i\in\Z$)
with the fundamental relations \eqref{eq:f_i-q-Serre}
and is an Ore domain.
In the sequel, we shall denote $\N_{-,\infty}$ by $U_{-,\infty}$.

The Weyl group $W_\infty=W(A_\infty)$ of type $A_\infty$ 
is defined to be the group generated by $s_i$ ($i\in\Z$) 
with the fundamental relations \eqref{eq:Weyl}.
The extended Weyl group $\tW_\infty$ is given by
the semi-direct product $\tW_\infty=W_\infty\rtimes\bra\pi\ket$ with
defining relations $\pi s_i = s_{i+1}\pi$ ($i\in\Z$).

Let $\Qv_\infty$ be the free $\Z$-module generated 
by $\deltav$, $\epsv_i$ ($i\in\Z$).
Put $\tP_\infty=\Hom(\Qv_\infty,\Z)$
and denote the canonical pairing 
between $\Qv_\infty$ and $\tP_\infty$ by $\bra\,\ ,\ \ket$.
Define $\Lambda_0,\eps_i\in\tP_\infty$ ($i\in\Z$) by
\begin{equation*}
 \bra\epsv_i,\eps_j\ket = \delta_{ij}, \quad
 \bra\deltav,\eps_j\ket = 0, \quad
 \bra\epsv_i,\Lambda_0\ket = 
 \begin{cases}
  1 & (i\leqq 0), \\
  0 & (i>0),
 \end{cases}
 \quad
 \bra\deltav,\Lambda_0\ket = 1.
\end{equation*}
Denote by $P_\infty$ the submodule of $\tP_\infty$ 
generated by $\Lambda_0$ and $\eps_i$ ($i\in\Z$).

The actions of $\pi$ on $\Qv_\infty$ and $P_\infty$ are defined by
\begin{equation*}
 \pi(\epsv_i) = \epsv_{i+1}, \quad
 \pi(\deltav) = \deltav, \quad
 \pi(\eps_i) = \eps_{i+1}, \quad
 \pi(\Lambda_0) = \Lambda_0+\eps_1.
\end{equation*}
The actions of $\pi$ preserve the canonical pairing
between $\Qv_\infty$ and $P_\infty$.
We define the simple coroots $\av_i$, 
the simple roots $\alpha_i$, 
and the fundamental weights $\Lambda_i$ by 
\begin{align*}
 \av_i = \epsv_i-\epsv_{i+1}, \quad
 \alpha_i = \eps_i - \eps_{i+1},
 \quad
 \Lambda_i = 
 \begin{cases}
  \Lambda_0 + \eps_1 + \cdots + \eps_i     & (i\geqq 0), \\
  \Lambda_0 - \eps_0 - \cdots - \eps_{i+1} & (i<0). \\
 \end{cases}
\end{align*}
Then we have $\pi(\av_i)=\av_{i+1}$, $\pi(\alpha_i)=\alpha_{i+1}$,
and $\pi(\Lambda_i)=\Lambda_{i+1}$.
The set $\{\Lambda_i\}_{i\in\Z}$ is a free $\Z$-basis of $P_\infty$.
We also define the Weyl vector $\rho\in\tP_\infty$ 
by $2\rho = - \sum_{i\in\Z}\eps_i$.
Then we obtain the formulas in \eqref{eq:a_{ij}}.
The generalized Cartan matrix $[a_{ij}]_{i\in\Z}$ of type $A_\infty$
is defined by $a_{ij}=\bra\av_i,\alpha_j\ket$.

The extended Weyl group actions on $\Qv_\infty$ and $P_\infty$ 
by the following formulas with the above of $\pi$ given above:
\begin{equation*}
 s_i(\beta) = \beta - \bra\beta,\alpha_i\ket\av_i 
 \quad (\beta\in\Qv_\infty), \qquad
 s_i(\lambda) = \lambda - \bra\av_i,\lambda\ket\alpha_i 
 \quad (\lambda\in P_\infty).
 \label{eq:Weyl_infinity-Q-P}
\end{equation*}
More explicitly, we have
\begin{align*}
 &
 s_i(\epsv_i)     = \epsv_{i+1}, \quad
 s_i(\epsv_{i+1}) = \epsv_i, \quad
 s_i(\epsv_j) = \epsv_j \quad (j\ne i,i+1),
 \\ &
 s_i(\deltav) = \deltav, \quad
 \\ &
 s_i(\eps_i) = \eps_{i+1}, \quad
 s_i(\eps_{i+1}) = \eps_i, \quad
 s_i(\eps_j) = \eps_j \quad (j\ne i,i+1),
 \\ &
 s_i(\Lambda_i) 
 = \Lambda_i - \alpha_i 
 = \Lambda_{i-1}-\Lambda_i+\Lambda_{i+1}, \quad
 s_i(\Lambda_j) = \Lambda_j \quad (j\ne i).
\end{align*}

We consider the Laurent polynomial ring 
$U_{-,\infty}^\pa=U_{-,\infty}[q^\beta|\beta\in\Qv_\infty]$ 
spanned by $\{q^\beta\}_{\beta\in\Qv_\infty}$ over $U_{-,\infty}$.
Then $U_{-,\infty}^\pa$ is also an Ore domain. 
We obtain the skew field $K_\infty^\pa$ of fractions of $U_{-,\infty}^\pa$.
Let $D(K_\infty^\pa)$ be the algebra generated by $K_\infty^\pa$ 
and $\{\tau^\lambda\}_{\lambda\in P_\infty}$ 
with the defining relations \eqref{eq:tau-def}
in which $P$ is replaced by $P_\infty$.

The following proposition immediately follows 
from \propref{prop:QWGA-A_{n-1}}.

\begin{prop}
\label{prop:QWGA-A_{infinity}}
 The algebra automorphism actions of $s_i$ ($i\in\Z$) and $\pi$ 
 on $D(K_\infty^\pa)$ can be given by
 \begin{align*}
  &
  s_i(f_j)
  = q^{-\av_i} f_j + [\av_i]_q [f_i,f_j]_{q^{-1}} f_i^{-1}
  \quad (j=i\pm1),
 \quad
  s_i(f_j) = f_j 
  \quad (j\ne i\pm 1), 
 \\ &
  s_i(q^\beta) = q^{s_i(\beta)} 
  \quad (\beta\in\Qv_\infty),
 \qquad
  s_i(\tau^\lambda) 
  = f_i^{\bra\av_i,\lambda\ket}\tau^{s_i(\lambda)} 
  \quad (\lambda\in P_\infty),
 \\ &
  \pi(f_i) = f_{i+1}, \qquad
  \pi(q^\beta) = q^{\pi(\beta)} 
  \quad (\beta\in\Qv_\infty), \qquad
  \pi(\tau^\lambda) = \tau^{\pi(\lambda)}
  \quad (\lambda\in P_\infty).
 \end{align*}
 These actions defines the actions of $\tW_\infty$ 
 on $D(K_\infty^\pa)$.
 \qed
\end{prop}

For the notational simplicity, we put
\begin{align*}
 &
 p=q^{-\deltav}, \quad 
 t_i = q^{-\epsv_i}, \quad 
 a_i = q^{-\av_i} = t_i/t_{i+1}, \quad
% \\ &
 \tau_i = \tau^{\Lambda_i}, \quad
 z_i = \tau^{\eps_i} = \tau_i/\tau_{i-1}.
\end{align*}
Then the algebra $U_{-,\infty}^\pa$ is generated 
by $\{\,f_i,t_i^{\pm1},p^{\pm1}\mid i\in\Z\,\}$,
and the algebra $D(K_\infty^\pa)$ is generated by $K_\infty^\pa$
and $\{\tau_i^{\pm1}\}_{i\in\Z}$.
The diagonal matrices $D_t$ and $D_Z$ are given by
\begin{equation*}
 D_t = \sum_{i\in\Z} t_i E_{ii}, \quad
 D_Z = \sum_{i\in\Z} z_i E_{ii}.
\end{equation*}  
We define the matrices $M$, $G_i$, and $\Lambda$ by
\begin{equation}
 M = D_t \tL D_t = [m_{ij}]_{i,j\in\Z}, \quad
 G_i = 1 + g_i E_{i+1,i}, \quad
 \Lambda = \sum_{i\in\Z} E_{i,i+1},
 \label{eq:M-G_i-Lambda}
\end{equation}
where \(
 g_i 
 = (t_i^2-t_{i+1}^2)/m_{i,i+1} 
 = (a_i-a_i^{-1})/f_{i,i+1} 
 = -[\av_i]_q/f_i
\). 
The matrix shall be called {\em the shift matrix}.
There exists a unique upper triangular matrix 
$U=1+\sum_{i<j}u_{ij}E_{ij}$ with $M=UD_t^2U^{-1}$.
Then we have $u_{i,i+1}=-g_i^{-1}$. 
Put $Z=U D_Z$.
We define the matrices $S_i^g$ and $S_i$ by
\begin{align*}
 &
 S_i^g = g_i^{-1} E_{i,i+1} - g_i E_{i+1,i} 
       + \sum_{k\ne i,i+1} E_{kk},
 \\ &
 S_i = -[\av_i-1]_q^{-1} E_{i,i+1} + [\av_i+1]_q E_{i+1,i}
     + \sum_{k\ne i,i+1} E_{kk}.
\end{align*}
The following theorem follows from the results of 
Sections \ref{sec:Lax-A_{n-1}} and \ref{sec:Sato-Wilson-A_{n-1}}.

\begin{theorem}[Lax and Sato-Wilson formalisms for type $A_\infty$]
\label{theorem:Lax-Sato-Wilson-A_{infinity}}
 We have
 \begin{align*}
  &
  s_i(M) = G_i M G_i^{-1}, \quad
  s_i(U) = G_i U S_i^g, 
  \\ &
  s_i(D_Z) = (S_i^g)^{-1} D_Z S_i, \quad
  s_i(Z) = G_i Z S_i, \quad
  s_i(D_t) = S_i^{-1} D_t S_i
  \\ &
  \pi(X) = \Lambda X \Lambda^{-1} 
  \quad \text{for $X=M,U,D_Z,Z,D_t$}.
 \end{align*}
 These relations with $s_i(p)=\pi(p)=p$,  
 $s_i(\tau_0) = f_0\tau_{-1}\tau_1/\tau_0$, 
 and $\pi(\tau_0)=\tau_1$ uniquely characterize
 the extended Weyl group action on $D(K_\infty^\pa)$.
 \qed
\end{theorem}

%%%%%%%%%%%%%%%%%%%%%%%%%%%%%%%%%%%%%%%%%%%%%%%%%%%%%%%%%%%%%%%%%%%%%%%%%%%%

\subsection{The case of type $A^{(1)}_{n-1}$ for $n\geqq 3$}
\label{sec:A^{(1)}_{n-1}}

In this subsection, we assume that $n\geqq 3$.
Denote by $\overline k$ the image of $k\in\Z$ in $\Z/n\Z$.

The Weyl group $W_n=W(A^{(1)}_{n-1})$ of type $A^{(1)}_{n-1}$ is defined to be
the group generated by $\{s_{\overline{i}}\}_{i=0}^{n-1}$
with the fundamental relations
\(
  s_{\overline{i}}s_{\overline{i+1}}s_{\overline{i}}
= s_{\overline{i+1}}s_{\overline{i}}s_{\overline{i+1}}
\), \(
  s_{\overline{i}}s_{\overline{j}}
= s_{\overline{j}}s_{\overline{i}}
\) ($\overline{j}\ne\overline{i\pm1}$), and
$s_{\overline{i}}^2 = 1$.
The extended Weyl group $\tW_n$ is given by the semi-direct
product $\tW_n=W_n\rtimes\bra\pi\ket$ with defining relations
$\pi s_{\overline{i}} = s_{\overline{i+1}} \pi$.

Let $\Qv_n$ (resp.\ $P_n$) be the quotient lattice 
of $\Qv_\infty$ (resp.\ $P_\infty$)
given by the relations $\epsv_{i+n}=\epsv_i-\deltav$
(resp.\ $\Lambda_{i+n}=\Lambda_i$) for $i\in\Z$.
The images of $\beta\in\Qv_\infty$ in $\Qv_n$ 
and $\lambda\in P_\infty$ in $P$ 
shall be denoted by the same symbols.
For example, $\av_i = \epsv_i - \epsv_{i+1}$, $\av_{i+n}=\av_i$ in $\Qv_n$ 
and $\eps_i = \Lambda_{i+1}-\Lambda_i$, $\eps_{i+n}=\eps_i$ in $P_n$.
Denote the induced actions of $\pi$ on $\Qv_n$ and $P_n$ by the same symbols:
\begin{equation*}
 \pi(\epsv_i) = \epsv_{i+1}, \quad
 \pi(\deltav) = \deltav
 \quad \text{in $\Qv_n$}, \qquad
 \pi(\Lambda_i) = \Lambda_{i+1}
 \quad \text{in $P_n$}.
\end{equation*}
The lattice $\Qv_n$ (resp.\ $P_n$) is the free $\Z$-module
spanned by $\deltav,\epsv_1,\ldots,\epsv_n$
(resp.\ $\Lambda_0,\eps_1,\ldots,\eps_n$).

We define the paring between $\Qv_n$ and $P_n$ by
\begin{align*}
 &
 \bra\epsv_i,\eps_j\ket = \delta_{ij} \quad (1\leqq i,j\leqq n), \qquad
 \bra\deltav,\eps_j\ket = 0,
 \\ &
 \bra\epsv_i,\Lambda_0\ket = 0 \quad (1\leqq i\leqq n), \qquad
 \bra\deltav,\Lambda_0\ket = 1.
\end{align*}
The induced actions of $\pi$ on $\Qv_n$ and $P_n$ 
preserve the pairing between them. 
Then we have
\begin{equation*}
 \bra\av_i,\alpha_j\ket =
 \begin{cases}
   2 & (\overline{j}\equiv\overline{i}), \\
  -1 & (\overline{j}\equiv\overline{i}\pm\overline{1}), \\
   0 & (\text{otherwise}), \\
 \end{cases}
 \qquad
 \bra\av_i,\Lambda_j\ket = \delta_{\overline{i},\overline{j}}.
\end{equation*}
The generalized Cartan matrix $[a_{\overline{i},\overline{j}}]_{i,j=0}^{n-1}$
of type $A^{(1)}_{n-1}$ is defined by 
$a_{\overline{i},\overline{j}} = \bra\av_i,\alpha_j\ket$.
The actions of the extended Weyl group $\tW_n$ on $\Qv_n$ and $P_n$
by the following formulas with the induced action of $\pi$:
\begin{equation*}
 s_{\overline{i}}(\beta) = \beta - \bra\beta,\alpha_i\ket\av_i, \quad
 s_{\overline{i}}(\lambda) = \lambda - \bra\av_i,\lambda\ket\alpha_i.
\end{equation*}
These actions also preserve the pairing between $\Qv_n$ and $P_n$. 

For each $i\in\Z$, 
let $\bars_i$ be the automorphism of $\Qv_\infty\oplus P_\infty$ given by
\begin{equation*}
 \bars_i(x) = \prod_{k\in\Z} s_{i+nk}(x)
 \quad (x\in \Qv_\infty\oplus P_\infty).
\end{equation*}
This is well-defined 
because the infinite product in the right-hand side 
reduces to the finite product for each $x$.
The mapping $s_{\overline{i}}\mapsto\bars_i$ ($i=0,1,\ldots,n-1$)
with the action of $\pi$ 
defines the action of $\tW_n$ on $\Qv_\infty\oplus P_\infty$,
which induces the action of $\tW_n$ on $\Qv_n\oplus P_n$ 
given above.
%
Furthermore the actions 
of the infinite product $\bars_i=\prod_{k\in\Z}s_{i+nk}$
for $i\in\Z$ on $D(K_\infty^\pa)$ are also well-defined 
and give the action of $\tW_n$ on $D(K_\infty^\pa)$
with the action of $\pi$.
More explicitly, we have
\begin{align*}
 &
 \bars_i(f_{j\pm1}) 
 = a_j f_{j\pm1} 
 + \frac{a_j^{-1}-a_j}{q-q^{-1}}[f_j,f_{j\pm1}]_{q^{-1}} f_j^{-1}
 \\ &
 \hphantom{\bars_i(f_{j\pm1})}
 = a_j^{-1} f_{j\pm1} 
 + \frac{a_j^{-1}-a_j}{q-q^{-1}}[f_j,f_{j\pm1}]_q        f_j^{-1}
 \\ &
 \hphantom{\bars_i(f_{j\pm1})}
 = \frac{qa_j-q^{-1}a_j^{-1}}{q-q^{-1}}f_{j\pm1}
 + \frac{a_i^{-1}-a_i}{q-q^{-1}}f_j f_{j\pm1} f_j^{-1}
 \qquad (\overline{j}=\overline{i}),
 \\ &
 \bars_i(f_k) = f_k \qquad (\overline{k}\ne\overline{i\pm1}),
 \\ &
 \bars_i(t_j)=t_{j+1}, \quad
 \bars_i(t_{j+1}) = t_j 
 \quad (\overline{j}=\overline{i}), 
 \quad
 \bars_i(t_k)=t_k 
 \quad (\overline{k}\ne\overline{i},\overline{i+1}), 
 \quad
% \\ &
 \bars_i(p) = p, 
 \\ &
 \bars_i(\tau_j) = f_j\frac{\tau_{j-1}\tau_{j+1}}{\tau_j}
 \quad (\overline{j}=\overline{i}), 
 \quad
 \bars_i(\tau_k) = \tau_k
 \quad (\overline{k}\ne\overline{i}).
\end{align*}
See also \secref{sec:Lax-A_{n-1}} and recall that $a_i=t_i/t_{i+1}$.

The multiplicative subset $S$ of $U_{-,\infty}^\pa$ generated by
\begin{equation*}
  f_k,\quad 1-p^{2k},\quad t_i^2-t_j^2 \quad (i,j,k\in\Z, i<j)
\end{equation*}
is an Ore set in $U_{-,\infty}^\pa$. 
We denote by $\tU_{-,\infty}^\pa$ the localization of $U_{-,\infty}^\pa$
with respect to $S$:
\begin{align*}
 &
 U_{-,\infty}^\pa = U_{-,\infty}[\, p^{\pm1},t_k^{\pm1} \mid k\in\Z\,],
 \\ &
 \tU_{-,\infty}^\pa
 = U_{-,\infty}
   [\, f_k^{-1}, p^{\pm1}, (1-p^{2k})^{-1}, t_k^{\pm1}, (t_i^2-t_j^2)^{-1}
   \mid i,j,k\in\Z, i<j \,].
\end{align*} 
Define the algebra $D(U_{-,\infty}^\pa)$ (resp.\ $D(\tU_{-,\infty}^\pa)$) 
to be the subalgebra of $D(K_\infty^\pa)$ generated 
by $U_{-,\infty}^\pa$ (resp.\ $\tU_{-,\infty}^\pa$)
and $\{\tau_i\}_{i\in\Z}$.
For each $i\in\Z$, the infinite product $\bars_i=\prod_{k\in\Z}s_{i+nk}$
maps $D(U_{-,\infty}^\pa)$ into $D(\tU_{-,\infty}^\pa)$.

Let $D(\tU_{-,n}^\pa)$ be the quotient algebra of $D(\tU_{-,\infty}^\pa)$
given by the following relations
\begin{equation}
 f_{i+n} = f_i, \quad t_{i+n} = p^{-1}t_i, \quad \tau_{i+n}=\tau_i
 \qquad (i\in\Z).
 \label{eq:n-reduction}
\end{equation}
Denote the images, in $D(\tU_{-,n}^\pa)$, of 
$U_{-,\infty}^\pa$, $\tU_{-,\infty}^\pa$, and $D(U_{-,\infty}^\pa)$ 
by $U_{-,n}^\pa$, $\tU_{-,n}^\pa$, and $D(U_{-,n}^\pa)$,
respectively.
The algebra $U_{-,n}^\pa$ can be identified with 
the Laurent polynomial ring 
generated by $p^{\pm1},t_1^{\pm1},\ldots,t_n^{\pm1}$
over the lower triangular part $U_{-,n}$ of 
the $q$-difference deformation of the universal enveloping algebra 
of the Kac-Moody algebra of type $A^{(1)}_{n-1}$
and hence is an Ore domain (\cite{Kuroki2008}).
Denote by $K_n^\pa$ the skew field of fractions of $U_{-,n}^\pa$.
Let $D(K_n^\pa)$ be the algebra generated by $K_n^\pa$
and $\{\tau^\lambda\}_{\lambda\in P_n}$ 
with the following defining relations:
\begin{align*}
 &
 \tau^\lambda \tau^\mu = \tau^{\lambda+\mu}, \quad
 \tau^0 = 1
 \quad (\lambda,\mu,0\in P_n)
 \\ &
 \tau^\lambda f_i = f_i \tau^\lambda, \quad
 \tau^\lambda q^\beta = q^{\beta+\bra\beta,\lambda\ket}\tau^\lambda
 \quad (\lambda\in P_n, \beta\in\Qv_n).
\end{align*}
We can naturally regard $D(\tU_{-,n}^\pa)$
as a subalgebra of $D(K_n^\pa)$.
We identify $q^{-\epsv_i},q^{-\deltav},\tau^{\Lambda_i}\in D(K_n^\pa)$
with $t_i,p,\tau_i\in D(U_{-,n}^\pa)$, respectively.
The action of $\pi$ on $D(K_n^\pa)$ is naturally induced
by the action on $D(U_{-,\infty})$.

The injective algebra homomorphism 
$\bars_i=\prod_{k\in\Z}s_{i+nk}:D(U_{-,\infty}^\pa)\to D(\tU_{-,\infty}^\pa)$
induces the algebra automorphism of $D(K_n^\pa)$, 
which shall be also denoted by $\bars_i$.
The mapping $s_{\overline{i}}\mapsto\bars_i$ with the induced action of $\pi$
defines the action of $\tW_n$ on $D(K_n^\pa)$.

Denote by $M_\Z(R)$ the set of all matrices of size $\Z$ over a ring $R$.
We shall denote the image of a matrix $A\in M_\Z(\tU_{-,\infty}^\pa)$
in $M_\Z(\tU_{-,n}^\pa)$ by the same symbol.

Recall that the matrices 
$\tL, D_t, M, G_i, U, D_Z, Z, S_i^g, S_i \in M_\Z(\tU_{-,\infty}^\pa)$
are given in \secref{sec:A_{infinity}}.
We have $\tL\in M_\Z(U_{-,\infty})$, $D_t,M\in M_\Z(U_{-,\infty}^\pa)$,
$G_i, S_i, S_i^g \in M_\Z(\tU_{-,\infty}^\pa)$, 
and $D_Z\in M_\Z(D(U_{-,\infty}^\pa))$.
From the formula \eqref{eq:u_{ij}}, we obtain $U\in M_\Z(\tU_{-,\infty}^\pa)$
and hence $Z\in M_\Z(D(\tU_{-,\infty}^\pa))$.
The relations in \eqref{eq:n-reduction} are equivalent to
\begin{equation*}
 \Lambda^n \tL \Lambda^{-n} = \tL, \quad
 \Lambda^n D_t \Lambda^{-n} = p^{-1} D_t, \quad
 \Lambda^n D_Z \Lambda^{-n} = D_Z.
\end{equation*}
These relations implies $\Lambda^n M \Lambda^{-n}=p^{-2}M$, 
$\Lambda^n U \Lambda^{-n} = U$, and $\Lambda^n Z \Lambda^{-n} = Z$.

We define the matrices 
$\barG_i,\barS_i^g,\barS_i\in M_\Z(\tU_{-,\infty}^\pa)$ by
\begin{align*}
 &
 \barG_i 
 = \prod_{j\in i+n\Z} G_j
 = 1 + \sum_{j\in i+n\Z} g_j E_{j+1,j},
 \\ &
 \barS_i^g 
 = \prod_{j\in i+n\Z} S_j
 = \sum_{j\in i+n\Z} (g_j^{-1}E_{j,j+1} - g_j E_{j+1,j})
 + \sum_{\overline{k}\ne\overline{i},\overline{i+1}} E_{kk},
 \\ &
 \barS_i
 = \prod_{j\in i+n\Z} S_j^g
 = \sum_{j\in i+n\Z} (-[\av_j-1]_q^{-1} E_{j,j+1}+[\av_j+1]_q E_{j+1,j})
 + \sum_{\overline{k}\ne\overline{i},\overline{i+1}} E_{kk},
\end{align*}
where $g_j=-[\av_j]_q/f_j$.
Since $[\av_{j+n}]_q=[\av_j]_q$ and $g_{j+n}=g_j$ in $\tU_{-,n}^\pa$, 
we have $\barG_{i+n}=\barG_i$, $\barS_{i+n}^g=\barS_i^g$, 
and $\barS_{i+n}=\barS_i$.
\theoremref{theorem:Lax-Sato-Wilson-A_{infinity}} 
immediately leads to the following proposition.

\begin{prop}
\label{prop:Lax-Sato-Wilson-A^{(1)}_{n-1}}
 We have, in $M_\Z(D(\tU_{-,\infty}^\pa))$, 
 \begin{align*}
  &
  \bars_i(M) = \barG_i M \barG_i^{-1}, \quad
  \bars_i(U) = \barG_i U \barS_i^g, 
  \\ &
  \bars_i(D_Z) = (\barS_i^g)^{-1} D_Z \barS_i, \quad
  \bars_i(Z) = \barG_i Z \barS_i, \quad
  s_i(D_t) = \barS_i^{-1} D_t \barS_i,
  \\ &
  \pi(X) = \Lambda X \Lambda^{-1} 
  \quad \text{for $X=M,U,D_Z,Z,D_t$}.
 \end{align*}
 These relations with $s_i(p)=\pi(p)=p$,  
 $s_i(\tau_0) = f_0\tau_{-1}\tau_1/\tau_0$, 
 and $\pi(\tau_0)=\tau_1$ uniquely characterize
 the action of the extended Weyl group $\tW_n$ on $D(K_n^\pa)$.
 \qed
\end{prop}

The above proposition is
the infinite matrix version of  
the Lax and the Sato-Wilson formalisms for the $A^{(1)}_{n-1}$ case
with $n\geqq 3$.

Let $R$ be an associative algebra with unit $1$.
Assume that $c$ is an invertible elements of the center of $R$
and is not a root of unity.
Recall that $\Lambda\in M_\Z(R)$ denotes 
the shift matrix given in \eqref{eq:M-G_i-Lambda}.
For each $k\in\Z$, let $M_\Z(R)_{n,c}^k$ be the set of 
all infinite matrices $X=[x_{ij}]_{i,j\in\Z}\in M_\Z(R)$ such that
$\Lambda^n X \Lambda^{-n}=c^{-k} X$ and
there exists an integer $N$ with $x_{ij}=0$ if $i-j>N$.
Then $M_\Z(R)_{n,c} = \bigoplus_{k\in\Z}M_\Z(R)_{n,c}^k$ can be 
naturally regarded as an algebra.
Define the infinite diagonal matrix $D_{n,c}$ by 
\begin{equation*}
 D_{n,c} = \sum_{k\in\Z} c^{-k} \sum_{i=1}^n E_{i+nk,i+nk}.
\end{equation*}
Then we have
\begin{equation*}
 \Lambda D_{n,c} \Lambda^{-1} =  
 \left(
  \sum_{i\in\Z} c^{-\delta_{\overline{i},\overline{0}}} E_{ii}
 \right) D_{n,c},
 \quad
 \Lambda^n D_{n,c} \Lambda^{-n} = c^{-1} D_{n,c}.
\end{equation*}
Therefore we obtain $D_{n,c}\in M_\Z(R)_{n,c}^1$ 
and $M_\Z(R)_{n,c}^k = M_\Z(R)_{n,c}^0 D_{n,c}^k$.
%
Denote by $M_n(R)$ the set of all square matrices of size $n$ 
over an associative algebra $R$ with $1$.
We introduce the spectral parameter $z$.
(Do not confuse the spectral parameter $z$ 
with the $z$-variables $z_i=\tau_i/\tau_{i-1}$.)
Let $c^{d/dz}$ be the difference operator acting 
on $R[z^{\pm1}]=R[z,z^{-1}]$ given by $c^{d/dz}(z)=cz$.
We obtain the matrix difference operator algebra 
$M_n(R[z^{\pm1},c^{\pm d/dz}])$.
We define {\em the shift matrix} $\Lambda(z)\in M_n(R[z])$ by
\begin{equation*}
 \Lambda(z) 
 = \sum_{i=1}^{n-1} E_{i,i+1} + z E_{n1}
 = \diag(1,\ldots,1,z) \Lambda(1).
\end{equation*}
Then we have, in $M(R[z^{\pm1},c^{\pm d/dz}])$, 
\begin{equation*}
 \Lambda(z) c^{d/dz} \Lambda(z)^{-1} = \diag(1,\ldots,1,c^{-1})c^{d/dz}, \quad
 \Lambda(z)^n c^{d/dz} \Lambda(z)^{-n} = c^{-1} c^{d/dz}.
\end{equation*}
We can define the algebra isomorphism 
$\iota_{n,c} : M_\Z(R)_{n,c}\to M_n(R[z^{\pm1},c^{\pm d/dz}])$ by
\begin{align*}
 &
 \iota_{n,c}(D_{n,c}) = c^{d/dz}, \quad
 \\ &
 \iota_{n,c}(X) 
 = \sum_{k\in\Z} z^k \sum_{i,j=1}^n x_{i,j+nk} E_{ij}
 \quad (X=[x_{ij}]_{i,j\in\Z}\in M_\Z(R)_{n,c}^0).
\end{align*}
In particular, we have $\iota_{n,c}(\Lambda)=\Lambda(z)$.

Let us apply the isomorphism $\iota_{n,c}$ for $R=D(\tU_{-,n}^\pa)$ and $c=p$
to the formulas in \propref{prop:Lax-Sato-Wilson-A^{(1)}_{n-1}}.
Denote the $\iota_{n,p}$-images 
of $\tL$, $D_t$, $M$, $U$, $D_Z$, $Z$, 
$\barG_i$, $\barS_i^g$, and $\barS_i$
by $\tL(z)$, $D_{t,n}p^{d/dz}$, $M(z)$, $U(z)$, $D_{Z,n}$, $Z(z)$,
$\barG_i(z)$, $\barS_i(z)$, and $\barS_i(z)$,
respectively. Then we have
\begin{align*}
 &
 \tL(z) 
 = 1 + \sum_{1\leqq i<j\leqq n} f_{ij} E_{ij} 
   + \sum_{k=1}^\infty z^k \sum_{i,j=1}^n f_{i,j+nk} E_{ij},
 \\ &
 D_{t,n} = \diag(t_1,\ldots,t_n), 
 \\ &
 M(z) = D_{t,n}p^{d/dz} \tL(z) D_{t,n}p^{d/dz}
 = D_{t,n}\tL(pz)D_{t,n}p^{2d/dz},
 \\ &
 U(z) 
 = 1 + \sum_{1\leqq i<j\leqq n} u_{ij} E_{ij} 
   + \sum_{k=1}^\infty z^k \sum_{i,j=1}^n u_{i,j+nk} E_{ij},
 \\ &
 M(z) = U(z) D_{t,n}^2 p^{2d/dz} U(z)^{-1}
 = U(z) D_{t,n}^2 U(p^2z)^{-1} p^{2d/dz}, 
 \\ &
 D_{Z,n} = \diag(z_1,\ldots,z_n), 
 % \\ &
 \quad
 Z(z) = U(z) D_{Z,n},
 \\ &
 \barG_i(z) = 1 + g_i E_{i+1,i} \quad (i=1,\ldots,n-1), \quad
 \barG_0(z) = 1 + z^{-1} g_0 E_{1n},
 \\ &
 \barS_i^g(z)
 = g_i^{-1} E_{i,i+1} - g_i E_{i+1,i} + \sum_{k\ne i,i+1} E_{kk}
 \quad (i=1,\ldots,n-1),
 \\ &
 \barS_0^g(z) 
 = z g_0^{-1} E_{n1}  - z^{-1} g_0 E_{1n} + \sum_{k=1}^{n-1} E_{kk},
 \\ &
 \barS_i(z)
 = -[\av_i-1]_q^{-1} E_{i,i+1} + [\av_i+1]_q E_{i+1,i} + \sum_{k\ne i,i+1} E_{kk}
 \quad (i=1,\ldots,n-1),
 \\ &
 \barS_0(z) 
 = - z [\av_0-1]_q^{-1} E_{n1} + z^{-1} [\av_0+1]_q E_{1n} + \sum_{k=2}^{n-1} E_{kk}.
\end{align*}
\propref{prop:Lax-Sato-Wilson-A^{(1)}_{n-1}} immediately leads to
the following theorem.

\begin{theorem}
[Lax and Sato-Wilson formalisms for type $A^{(1)}_{n-1}$, $n\geqq3$]
\label{theorem:Lax-Sato-Wilson-A^{(1)}_{n-1}}
 We have, in $M_n\left(D(\tU_{-,\infty}^\pa)[z^{\pm1},p^{\pm d/dz}]\right)$, 
 \begin{align*}
  &
  \bars_i(M(z)) = \barG_i(z) M(z) \barG_i(z)^{-1}, \quad
  \bars_i(U(z)) = \barG_i(z) U(z) \barS_i^g(z), 
  \\ &
  \bars_i(D_{Z,n}) = \barS_i^g(z)^{-1} D_{Z,n} \barS_i(z), \quad
  \bars_i(Z(z)) = \barG_i(z) Z(z) \barS_i(z), \quad
  \\ &
  \bars_i(D_{t,n}) = \barS_i(z)^{-1} D_{t,n} \barS_i(pz),
  \\ &
  \pi(X) = \Lambda(z) X \Lambda(z)^{-1} 
  \quad \text{for $X=M(z),U(z),D_{Z,n},Z(z),D_{t,n}$}.
 \end{align*}
 These relations with $s_i(p)=\pi(p)=p$,  
 $s_i(\tau_0) = f_0\tau_{-1}\tau_1/\tau_0$, 
 and $\pi(\tau_0)=\tau_1$ uniquely characterize
 the action of the extended Weyl group $\tW_n$ on $D(K_n^\pa)$.
 \qed
\end{theorem}


%%%%%%%%%%%%%%%%%%%%%%%%%%%%%%%%%%%%%%%%%%%%%%%%%%%%%%%%%%%%%%%%%%%%%%%%%%%%

\subsection{The case of type $A^{(1)}_{1}$}
\label{sec:A^{(1)}_1}

%%%%%%%%%%%%%%%%%%%%%%%%%%%%%%%%%%%%%%%%%%%%%%%%%%%%%%%%%%%%%%%%%%%%%%%%%%%%
%%%%%%%%%%%%%%%%%%%%%%%%%%%%%%%%%%%%%%%%%%%%%%%%%%%%%%%%%%%%%%%%%%%%%%%%%%%%

\begin{thebibliography}{99}

\bibitem{BK2000}
Berenstein, Arkady and Kazhdan, David. 
Geometric and unipotent crystals. 
GAFA 2000 (Tel Aviv, 1999). 
Geom.\ Funct.\ Anal.\ 2000, Special Volume, Part I, 188--236.
\arxivref{math/9912105}

%\bibitem{BK2007}
%Berenstein, Arkady and Kazhdan, David. 
%Geometric and unipotent crystals. II. 
%From unipotent bicrystals to crystal bases. 
%Quantum groups, 13–88, Contemp.\ Math., 433, Amer.\ Math.\ Soc., Providence, RI, 2007.
%\arxivref{math/0601391}

%\bibitem{DGK}
%Deodhar, Vinay V., Gabber, Ofer, and Kac, Victor. 
%Structure of some categories of representations 
%of infinite-dimensional Lie algebras. 
%Adv.\ in Math.\ 45 (1982), no.~1, 92--116.

\bibitem{DF}
Ding, Jin Tai and Frenkel, Igor B. 
Isomorphism of two realizations of quantum affine algebra $U_q(\widehat{\mathfrak{gl}(n)})$. 
Comm.\ Math.\ Phys.\ 156 (1993), no.~2, 277--300.

%\bibitem{Dixmier}
%Dixmier, Jacques.
%Enveloping Algebras,
%The 1996 Printing of the 1977 English Translation.
%Graduate Studies in Mathematics, Volume 11, 
%American Mathematical Society, 1996, 379~pp.
%
%\bibitem{EK}
%Etingof, Pavel and Kazhdan, David. 
%Quantization of Lie bialgebras. 
%VI. Quantization of generalized Kac-Moody algebras. 
%Transform.\ Groups 13 (2008), no.~3--4, 527--539. 

\bibitem{GR}
Gelfand, I.\ and Retakh, V. 
Quasideterminants. I. 
Selecta Math.\ (N.S.) 3 (1997), no.~4, 517--546.
\arxivref{q-alg/9705026}

\bibitem{GGRW}
Gelfand, Israel, Gelfand, Sergei, Retakh, Vladimir, and Wilson, Robert Lee. Quasideterminants. 
Adv.\ Math.\ 193 (2005), no.~1, 56--141.
\arxivref{math/0208146}

%\bibitem{GW-2004}
%Goodearl, K.~R.\ and Warfield, R.~B., Jr.,
%An introduction to noncommutative Noetherian rings, Second edition.
%London Mathematical Society Student Texts, 61, 
%Cambridge University Press, Cambridge, 2004, xxiv+344~pp. 
%
%\bibitem{Hasegawa2007}
%Hasegawa, Koji. 
%Quantizing the B\"acklund transformations of Painlev\'e equations 
%and the quantum discrete Painlev\'e VI equation. 
%Exploring new structures and natural constructions in mathematical physics, 275--288, 
%Adv.\ Stud.\ Pure Math., 61, Math.\ Soc.\ Japan, Tokyo, 2011.
%\arxivref{math/0703036}
%
%\bibitem{Hirota}
%Hirota, Ryogo.
%Discrete analogue of a generalized Toda equation. 
%J.\ Phys.\ Soc.\ Japan 50 (1981), no.~11, 3785--3791. 
%
%\bibitem{Jat-1986}
%Jategaonkar, A.~V.
%Localization in Noetherian rings.
%London Mathematical Society lecture note series, 98, 
%Cambridge University Press, 1986, xii+323~pp.
%
%\bibitem{JNS}
%Jimbo, M., Nagoya, H., and  Sun, J. 
%Remarks on the confluent KZ equation for $\mathfrak{sl}_2$ 
%and quantum Painlev\'e equations. 
%J.\ Phys.\ A 41 (2008), no.~17, 175205, 14~pp. 

\bibitem{Jos-1995}
Joseph, Anthony.
Quantum groups and their primitive ideals.
Ergebnisse der Mathematik und ihrer Grenzgebiete (3) 
[Results in Mathematics and Related Areas (3)], 29. 
Springer-Verlag, Berlin, 1995. x+383~pp. 

%\bibitem{KW}
%Kac, Victor G.\ and Wakimoto, Minoru.
%Modular invariant representations of infinite-dimensional Lie algebras and superalgebras.
%Proc.\ Nat.\ Acad.\ Sci.\ U.S.A.\ 85 (1988), no.~14, 4956--4960.
%
%\bibitem{KNY}
%Kajiwara, Kenji, Noumi, Masatoshi, and Yamada, Yasuhiko. 
%A study on the fourth $q$-Painlev\'e equation. 
%J.\ Phys.\ A 34 (2001), no.~41, 8563–8581.
%\arxivref{nlin/0012063}

\bibitem{Kuroki2008}
Kuroki, Gen.
Quantum groups and quantization of Weyl group symmetries of Painlev\'e systems. 
Exploring new structures and natural constructions in mathematical physics, 
289--325, Adv.\ Stud.\ Pure Math., 61, Math.\ Soc.\ Japan, Tokyo, 2011.
\arxivref{0808.2604}

\bibitem{Kuroki2012a}
Kuroki, Gen.
Regularity of quantum $\tau$-functions generated by 
quantum birational Weyl group actions.
Preprint 2012. \\
\href
{http://www.math.tohoku.ac.jp/~kuroki/LaTeX/RegularityOfQuantumTau.pdf}
{http://www.math.tohoku.ac.jp/\textasciitilde kuroki/LaTeX/RegularityOfQuantumTau.pdf}

%\bibitem{Lusztig}
%Lusztig, George. 
%Introduction to quantum groups. 
%Reprint of the 1994  edition.  
%Modern Birkhauser Classics. 
%Birkhauser/Springer, New York, 2010.\ xiv+346 pp. 
%
%\bibitem{MR-2001}
%McConnell, J.~C.\ and Robson, J.~C., 
%Noncommutative Noetherian rings,
%With the cooperation of L.~W.~Small, 
%Revised edition, 
%Graduate Studies in Mathematics, 30, 
%American Mathematical Society, Providence, RI, 2001, xx+636~pp.

%\bibitem{Malikov}
%Malikov, Fyodor.
%Quantum groups: singular vectors and BGG resolution. 
%Infinite analysis, Part A, B (Kyoto, 1991), 623--643, 
%Adv.\ Ser.\ Math.\ Phys., 16, World Sci.\ Publ., River Edge, NJ, 1992. 

%\bibitem{Miwa}
%Miwa, Tetsuji.
%On Hirota's difference equations. 
%Proc.\ Japan Acad.\ Ser.~A Math. Sci.\ 58 (1982), no.~1, 9--12. 

%\bibitem{Nagoya2009}
%Nagoya, Hajime.
%A quantization of the sixth Painlev\'e equation. 
%Noncommutativity and singularities, 291–298, 
%Adv.\ Stud.\ Pure Math., 55, Math.\ Soc.\ Japan, Tokyo, 2009. 

%\bibitem{Nagoya2011}
%Nagoya, H.
%Hypergeometric solutions to Schr\"odinger equations 
%for the quantum Painlev\'e equations. 
%J.\ Math.\ Phys.\ 52 (2011), no.~8, 083509, 16~pp. 
%\arxivref{1109.1645}
%
%\bibitem{Nagoya2012}
%Nagoya, Hajime.
%Realizations of Affine Weyl Group Symmetries 
%on the Quantum Painlev\'e Equations by Fractional Calculus.
%Lett.\ Math.\ Phys.\ Online First, 29 March 2012,
%DOI: 10.1007/s11005-012-0557-6.
%\\ \href
%{http://www.springerlink.com/content/1470478223ugw4k1/}
%{http://www.springerlink.com/content/1470478223ugw4k1/}
%
%\bibitem{NGR}
%Nagoya, Hajime, Grammaticos, Basil, and Ramani, Alfred.
%Quantum Painlev\'e equations: from continuous to discrete. 
%SIGMA Symmetry Integrability Geom.\ Methods Appl.\ 4 (2008), Paper 051, 9 pp. 
%\arxivref{0806.1466}

\bibitem{Noumi}
Noumi, Masatoshi. 
Painlev\'e equations through symmetry. 
Translated from the 2000 Japanese original by the author. 
Translations of Mathematical Monographs, 223. 
American Mathematical Society, Providence, RI, 2004. x+156 pp. 

%\bibitem{NY9708018}
%Noumi, Masatoshi and Yamada, Yasuhiko.
%Symmetries in the fourth Painlev\'e equation and Okamoto polynomials.  
%Nagoya Math.\ J.\ 153 (1999), 53--86. 
%\arxivref{q-alg/9708018}

\bibitem{NY0012028}
Noumi, Masatoshi and Yamada, Yasuhiko.
Birational Weyl group action arising from a nilpotent Poisson algebra. 
Physics and combinatorics 1999 (Nagoya), 287--319, 
World Sci.\ Publ., River Edge, NJ, 2001. 
\arxivref{math.QA/0012028}

%\bibitem{Okamoto1981}
%Okamoto, Kazuo. 
%On the $\tau$-function of the Painlev\'e equations. 
%Phys.\ D 2 (1981), no.~3, 525--535.
%
%\bibitem{OkamotoIII}
%Okamoto, Kazuo. 
%Studies on the Painlev\'e equations. 
%III. Second and fourth Painlev\'e equations, PII and PIV. 
%Math.\ Ann.\ 275 (1986), no.~2, 221--255. 

%\bibitem{Okamoto1996}
%Okamoto, Kazuo.
%Algebraic relations among six adjacent $\tau$-functions 
%related to the fourth Painlev\'e system. 
%Kyushu J.\ Math.\ 50 (1996), no.~2, 513–532. 

%\bibitem{Oshima}
%Oshima, Toshio.
%Fractional calculus of Weyl algebra and Fuchsian differential equations.
%Preprint \arxivref{1102.2792}
%
%\bibitem{RCW}
%Rocha-Caridi, Alvany and Wallach, Nolan R., 
%Characters of irreducible representations of the Virasoro algebra,  
%Math.\ Z.\ 185 (1984), no.~1, 1--21. 
%
%\bibitem{Sato-Sato}
%Sato, Mikio and Sato, Yasuko. 
%Soliton equations as dynamical systems on infinite-dimensional Grassmann manifold. 
%Nonlinear partial differential equations in applied science (Tokyo, 1982), 259--271, 
%North-Holland Math. Stud., 81, North-Holland, Amsterdam, 1983.

\bibitem{S-1971}
Smith, P.~F.
Localization and the AR property.
Proc.\ London Math.\ Soc.\ (3) 22 (1971), 39--68.

%\bibitem{Yamada9808002}
%Yamada, Yasuhiko. 
%Determinant formulas of the $\tau$-functions of the Painlev\'e equations of type A. 
%Nagoya Math.\ J.\ 156 (1999), 123–134.
%\arxivref{math/9808002}
%
%\bibitem{Yamada-SP}
%Yamada, Yasuhiko.
%Special polynomials and generalized Painlev\'e equations. 
%Combinatorial methods in representation theory (Kyoto, 1998), 391–400, 
%Adv.\ Stud.\ Pure Math., 28, Kinokuniya, Tokyo, 2000. 
%%\\
%\href
%{http://www.math.kobe-u.ac.jp/~yamaday/comb.ps}
%{http://www.math.kobe-u.ac.jp/\textasciitilde yamaday/comb.ps}
%
%\bibitem{Yamakawa}
%Yamakawa, Daisuke.
%Quiver varieties with multiplicities, 
%Weyl groups of non-symmetric Kac-Moody algebras, and Painlev\'e equations.
%SIGMA Symmetry Integrability Geom.\ Methods Appl.\ 6 (2010), Paper 087, 43~pp. 
%\arxivref{1003.3633}

\end{thebibliography}

%%%%%%%%%%%%%%%%%%%%%%%%%%%%%%%%%%%%%%%%%%%%%%%%%%%%%%%%%%%%%%%%%%%%%%%%%%%%
\end{document}
%%%%%%%%%%%%%%%%%%%%%%%%%%%%%%%%%%%%%%%%%%%%%%%%%%%%%%%%%%%%%%%%%%%%%%%%%%%%
