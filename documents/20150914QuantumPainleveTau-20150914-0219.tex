%%%%%%%%%%%%%%%%%%%%%%%%%%%%%%%%%%%%%%%%%%%%%%%%%%%%%%%%%%%%%%%%%%%%%%%%%%%%%%
\def\TITLE{\bf Painlev\'e系とその $\tau$ 函数の正準量子化}
\def\PDFTITLE{量子 Painlev\'e $\tau$ 函数}
\def\PDFAUTHOR{黒木玄}
\def\PDFSUBJECT{数学}
\def\AUTHOR{黒木玄 (Gen Kuroki)}
\def\INSTITUTE{東北大学数学教室
\\ \quad
\\ 日本数学会 2015年度秋季総合分科会 
\\ 京都産業大学 2015年9月13日(日)~16日(水)
\\ \VERSION
\\ \URL
}
\def\DATE{講演日2015年9月14日(月)
}
\def\VERSION{%
2015/09/14 Version 1.0
}
\def\URL{%
\href{http://www.math.tohoku.ac.jp/~kuroki/LaTeX/20150914QuantumPainleveTau.pdf}
{\tt http://www.math.tohoku.ac.jp/{\textasciitilde}kuroki/LaTeX/20150914QuantumPainleveTau.pdf}
}
%%%%%%%%%%%%%%%%%%%%%%%%%%%%%%%%%%%%%%%%%%%%%%%%%%%%%%%%%%%%%%%%%%%%%%%%%%%%%%
%\documentclass[20pt,dvipdfm]{beamer}
%\documentclass[17pt,dvipdfm]{beamer}
\documentclass[14pt,dvipdfm]{beamer}
%\documentclass[dvipdfm]{beamer}
% pdfの栞の字化けを防ぐ
% \AtBeginDvi{\special{pdf:tounicode EUC-UCS2}}
% テーマ
%\usetheme{AnnArbor}
\usetheme{Madrid}
\usecolortheme{beaver}
%\usecolortheme{crane}
\usecolortheme{dove}
%\usecolortheme{seagull}
%%%%%%%%%%%%%%%%%%%%%%%%%%%%%%%%%%%%%%%%%%%%%%%%%%%%%%%%%%%%%%%%%%%%%%%%%%%%%%
% navi. symbolsは目立たないが,dvipdfmxを使うと機能しないので非表示に
\setbeamertemplate{navigation symbols}{} 
\usepackage{graphicx}
\usepackage{amsmath}
\usepackage{amssymb}
\usepackage{amscd}
%%%%%%%%%%%%%%%%%%%%%%%%%%%%%%%%%%%%%%%%%%%%%%%%%%%%%%%%%%%%%%%%%%%%%%%%%%%%%%
% フォントはお好みで
\usepackage{txfonts}
\mathversion{bold}
\renewcommand{\familydefault}{\sfdefault}
\renewcommand{\kanjifamilydefault}{\gtdefault}
\setbeamerfont{title}{size=\large,series=\bfseries}
\setbeamerfont{frametitle}{size=\large,series=\bfseries}
\setbeamertemplate{frametitle}[default][center]
\usefonttheme{professionalfonts} 
%%%%%%%%%%%%%%%%%%%%%%%%%%%%%%%%%%%%%%%%%%%%%%%%%%%%%%%%%%%%%%%%%%%%%%%%%%%%%%
\newcommand\red{\color{red}}
\newcommand\blue{\color{blue}}
\newcommand\green{\color{green}}
\newcommand\magenta{\color{magenta}}
\newcommand\cyan{\color{cyan}}
\newcommand\yellow{\color{yellow}}
\newcommand\white{\color{white}}
\newcommand\black{\color{black}}
\renewcommand\r{\red}
\renewcommand\b{\blue}
%%%%%%%%%%%%%%%%%%%%%%%%%%%%%%%%%%%%%%%%%%%%%%%%%%%%%%%%%%%%%%%%%%%%%%%%%%%%
\newcommand\tW{{\widetilde W}}
\newcommand\WxW{\tW(A^{(1)}_{m-1})\times\tW(A^{(1)}_{n-1})}
\newcommand\Peq[1]{\mathrm{P}_{\mathrm{#1}}}
\newcommand\PIV{\Peq{IV}}
\newcommand\qPeq[1]{q\mathrm{P}_{\mathrm{#1}}}
\newcommand\qPIV{\qPeq{IV}}
\newcommand\eps{\varepsilon}
\newcommand\MOD{\mathop{\mathrm{mod}}\nolimits}
\renewcommand\pmod[1]{\;(\MOD #1)}
\renewcommand\d{\partial}
\newcommand\ad{\mathop{\mathrm{ad}}\nolimits}
\newcommand\Ad{\mathop{\mathrm{Ad}}\nolimits}
\newcommand\Hom{\mathop{\mathrm{Hom}}\nolimits}
\newcommand\bra{\langle}
\newcommand\ket{\rangle}
\newcommand\av{\alpha^\vee}
\newcommand\ev{\eps^\vee}
\newcommand\A{{\mathcal A}}
\newcommand\K{{\mathcal K}}
\newcommand\B{{\mathcal B}}
\newcommand\N{{\mathcal N}}
\newcommand\cl{{\mathrm{cl}}}
\newcommand\h{{\mathfrak h}}
\newcommand\n{{\mathfrak n}}
%\renewcommand\b{{\mathfrak b}}
\newcommand\g{{\mathfrak g}}
\newcommand\Spec{\mathop{\mathrm{Spec}}\nolimits}
\newcommand\lie{\mathrm}
\newcommand\trace{\mathop{\mathrm{tr}}\nolimits}
\makeatletter\newcommand\qbinom{\genfrac[]\z@{}}\makeatother
\newcommand\st{{\tilde s}}
\newcommand\rt{{\tilde r}}
\newcommand\gr{\mathop{\mathrm{gr}}\nolimits}
\newcommand\Lt{\widetilde{L}}
\renewcommand\L{{\mathcal L}}
\newcommand\M{{\mathcal M}}
\newcommand\G{{\mathcal G}}
\newcommand\diag{\mathop{\mathrm{diag}}\nolimits}
\newcommand\At{\widetilde{\A}}
\newcommand\Wt{\widetilde{W}}
\newcommand\minv{{\widetilde m}}
\newcommand\ninv{{\tilde n}}
\newcommand\all{{\mathrm{all}}}
\newcommand\ball{b_\all}
\newcommand\isom{\cong}
\newcommand\LL{{\mathbb L}}
\newcommand\XX{{\mathbb X}}
\newcommand\bigzerol{\smash{\hbox{\large $0$}}}             % 左下の大きなゼロ
\newcommand\bigzerou{\smash{\lower.3ex\hbox{\large $0$}}}   % 右上の大きなゼロ
\newcommand\bigstarl{\smash{\hbox{\Large $*$}}}           % 左下の大きな星
\newcommand\bigstaru{\smash{\lower.3ex\hbox{\Large $*$}}} % 右上の大きな星
%%%%%%%%%%%%%%%%%%%%%%%%%%%%%%%%%%%%%%%%%%%%%%%%%%%%%%%%%%%%%%%%%%%%%%%%%%%%
\newcommand\tL{{\widetilde{L}}}
\newcommand\tM{{\widetilde{M}}}
\newcommand\cG{{\mathcal{G}}}
\newcommand\tg{{\tilde{g}}}
\newcommand\ta{{\tilde{a}}}
\newcommand\tb{{\tilde{b}}}
\newcommand\tc{{\tilde{c}}}
\newcommand\ts{{\tilde{s}}}
\newcommand\tw{{\tilde{w}}}
\newcommand\tC{{\widetilde{C}}}
%\newcommand\tW{{\widetilde{W}}}
\newcommand\hL{{\widehat{L}}}
\newcommand\hM{{\widehat{M}}}
\newcommand\ha{{\hat{a}}}
\newcommand\hb{{\hat{b}}}
\newcommand\hf{{\hat{f}}}
\newcommand\dv{\delta^\vee}
%%%%%%%%%%%%%%%%%%%%%%%%%%%%%%%%%%%%%%%%%%%%%%%%%%%%%%%%%%%%%%%%%%%%%%%%%%%%
%\newcommand\N{{\mathbb N}} % natural numbers
\newcommand\Z{{\mathbb Z}} % rational integers
\newcommand\F{{\mathbb F}} % finite field
\newcommand\Q{{\mathbb Q}} % rational numbers
\newcommand\R{{\mathbb R}} % real numbers
\newcommand\C{{\mathbb C}} % complex numbers
%\renewcommand\P{{\mathbb P}} % projective spaces
%%%%%%%%%%%%%%%%%%%%%%%%%%%%%%%%%%%%%%%%%%%%%%%%%%%%%%%%%%%%%%%%%%%%%%%%%%%%
%\newcommand\arxivref[1]{{\tt\red arXiv:#1}}
\newcommand\arxivref[1]{\href{http://arxiv.org/abs/#1}{{\tt arXiv:#1}}}
%%%%%%%%%%%%%%%%%%%%%%%%%%%%%%%%%%%%%%%%%%%%%%%%%%%%%%%%%%%%%%%%%%%%%%%%%%%%%%
\newcommand\migiue{\text{/}}
\newcommand\migish{\text{\}}
%%%%%%%%%%%%%%%%%%%%%%%%%%%%%%%%%%%%%%%%%%%%%%%%%%%%%%%%%%%%%%%%%%%%%%%%%%%%%%
\title{\magenta\TITLE}
\author{\AUTHOR}
\institute{\INSTITUTE}
\date{\DATE}
%%%%%%%%%%%%%%%%%%%%%%%%%%%%%%%%%%%%%%%%%%%%%%%%%%%%%%%%%%%%%%%%%%%%%%%%%%%%%%
\begin{document}
%%%%%%%%%%%%%%%%%%%%%%%%%%%%%%%%%%%%%%%%%%%%%%%%%%%%%%%%%%%%%%%%%%%%%%%%%%%%%%
\frame{\titlepage}
%%%%%%%%%%%%%%%%%%%%%%%%%%%%%%%%%%%%%%%%%%%%%%%%%%%%%%%%%%%%%%%%%%%%%%%%%%%%%%%
\frame{
\frametitle{真のタイトル}

\begin{center}
{\Large\bf\magenta だれでもできる

\bigskip

Painlev\'e系とその ``$\tau_i$'' の正準量子化}

\bigskip\bigskip

量子化によって

古典の場合には曖昧にすませていたことを

まじめに考え直さざるを得なくなる.

\end{center}

}
%%%%%%%%%%%%%%%%%%%%%%%%%%%%%%%%%%%%%%%%%%%%%%%%%%%%%%%%%%%%%%%%%%%%%%%%%%%%%%%
\frame{
\frametitle{正準量子化とは}

Classical mechanics $\longrightarrow$ Quantum mechanics

\bigskip

Poisson brackets $\longrightarrow$ non-commutativities
\bigskip

$\{p,x\}=1$ $\longrightarrow$ $[p,x]=px-xp=1$
\quad ($p=\d/\d x$, 微分)

\bigskip

$\{\tau,x\}=\tau$ $\longrightarrow$ $\tau x \tau^{-1} = x+1$ 
\quad ($\tau=e^{\d/\d x}$, 差分)

\bigskip

$\{\tau,a\}=\tau a$ $\longrightarrow$ $\tau a\tau^{-1}=qa$ 
\quad ($a=q^x$, $q$ 差分)

\bigskip\bigskip

古典系の様々な良い性質を保ちながら量子化したい.

}
%%%%%%%%%%%%%%%%%%%%%%%%%%%%%%%%%%%%%%%%%%%%%%%%%%%%%%%%%%%%%%%%%%%%%%%%%%%%%%
\frame{
\frametitle{例: 微分版 Painlev\'e $\PIV$ の量子化}
\begin{small}

{\red $A^{(1)}_2$} 型の場合.

\bigskip

従属変数 $f_{i+3}=f_i$, \quad パラメーター変数 $\av_{i+3}=\av_i$, \quad

\medskip

$[f_i,f_{i+1}]=1$, \quad
$[\av_i,\av_j]=0$, \quad 
$[\av_i,f_j]=0$.

\bigskip

$\PIV$: \qquad\qquad
\(
\dfrac{\d f_i}{\d t}=f_i f_{i+1} - f_{i-1}f_i+\av_i
\)

\bigskip

対称性(Weyl群作用):
\\
\qquad\qquad\qquad
\( s_i(f_i)=f_i \), \quad
\( s_i(f_{i\pm1})=f_{i\pm1}\pm\dfrac{\av_i}{f_i} \),
\\[\medskipamount]
\qquad\qquad\qquad
\( s_i(\av_i)=-\av_i \), \quad
\( s_i(\av_{i\pm1})=\av_i+\av_{i\pm1} \).

\bigskip

{\blue 非可換性以外は古典の場合と同じ.}

\end{small}
}
%%%%%%%%%%%%%%%%%%%%%%%%%%%%%%%%%%%%%%%%%%%%%%%%%%%%%%%%%%%%%%%%%%%%%%%%%%%%%%
\frame{
\frametitle{例: 量子 $\PIV$ への $\tau$ 変数の導入}
\begin{small}

量子化された $\tau$ 変数 $\tau_i = \exp(\d/\d\av_i)$ (差分作用素):

\medskip
\qquad\qquad
{\red
$\tau_i\av_j\tau_i^{-1}=\av_j+\delta_{ij}$, \quad
$\tau_i\tau_j=\tau_j\tau_i$, \quad
$\tau_i f_j = f_j \tau_i$.
}

\medskip

対称性(Weyl群作用): 
\\[\medskipamount]
\qquad\qquad
\( s_i(\tau_i)=f_i\dfrac{\tau_{i-1}\tau_{i+1}}{\tau_i} \), \qquad
$s_i(\tau_j)=\tau_j$ \quad ($j\not\equiv i\pmod 3)$.

\medskip

対称性を用いて従属変数 $f_i$ を $\tau$ 変数で表示できる:
\\[\medskipamount]
\qquad\qquad\qquad\qquad\qquad
$f_i = \dfrac{s_i(\tau_i)\tau_i}{\tau_{i-1}\tau_{i+1}}$.

\bigskip

{\blue 非可換性以外は古典の場合と同じ.}

\medskip

{\blue しかし, $\tau$ 変数の非可換性の入れ方は新しい.}

\medskip
{\magenta $q$ 差分版では「古典の場合と同じ」とは言い難くなる(後述).}

\medskip

$\tau$ 変数は任意の対称化可能GCMに付随する場合に導入可能.

\end{small}
}
%%%%%%%%%%%%%%%%%%%%%%%%%%%%%%%%%%%%%%%%%%%%%%%%%%%%%%%%%%%%%%%%%%%%%%%%%%%%%%%
\frame{
\frametitle{まずWeyl群双有理作用の部分を量子化したい}
%\begin{small}

古典Painlev\'e系の対称性の典型的形は
\begin{align*}
s_i(f_j) = f_j \pm \frac{\av_i}{f_i}, \quad
s_i(\tau_i) = f_i\frac{\tau_{i-1}\tau_{i+1}}{\tau_i}, \quad
\text{etc.}
\end{align*}
$f_i$ は従属変数, \\
$\av_i$ はパラメーター変数 \quad (← simple coroot に対応), \\
$\tau_i$ は $\tau$ 変数 \quad (← $\exp(\text{fundamental weight})$ に対応).

\bigskip

これらを(正準)量子化したい.

\bigskip

{\blue $\tau$ 変数も適切に非可換化する(New!)}

%\end{small}
}
%%%%%%%%%%%%%%%%%%%%%%%%%%%%%%%%%%%%%%%%%%%%%%%%%%%%%%%%%%%%%%%%%%%%%%%%%%%%%%%
\frame{
\frametitle{Weyl群作用を作るための基本アイデア}
\begin{small}
\begin{center}

{\bf\blue Serre関係式}\\
{\red $[f_1,[f_1,f_2]]=0$} \quad (例 $f_1=\d/\d x$, $f_2=x-1$) \quad や \\
{\red $f_1^2f_2-(q+q^{-1})f_1f_2f_1+f_2f_1^2=0$} (量子展開環)のとき,
\[
\text{\bf\blue Verma関係式}\quad
{\red f_1^k f_2^{k+l} f_1^l = f_2^l f_1^{k+l} f_2^k}.
\]
$\downarrow$

{\bf\blue パラメーター変数へのWeyl群作用}\\
$\ts_i(\av_i)=\ts_i\av_i\ts_i^{-1}=-\av_i$, \\ 
$\ts_i(\av_j)=\ts_i\av_j\ts_i^{-1}=\av_i+\av_j$ \quad ($i,j=1,2$, $i\ne j$),
$\ts_i(f_j)=\ts_i f_j \ts_i^{-1}=f_j$ と仮定して, \\
{\red $\sigma_i := f_i^{\av_i}\ts_i$}

$\downarrow$

{\bf\blue braid 関係式}\quad
{\red $\sigma_1\sigma_2\sigma_1 = \sigma_2\sigma_1\sigma_2$}

\end{center}
\end{small}
}
%%%%%%%%%%%%%%%%%%%%%%%%%%%%%%%%%%%%%%%%%%%%%%%%%%%%%%%%%%%%%%%%%%%%%%%%%%%%%%%
\frame{
\frametitle{非可換環の要素の変数べき}
\begin{small}

$\sigma:=f_i^{\av_i}\ts_i$. \quad 代数自己同型 \vspace{-2\medskipamount}
\[
x \mapsto s_i(x)=\sigma_i x {\sigma_i}^{-1} 
= {\red f_i^{\av_i} \ts_i(x) f_i^{-\av_i}}
\]
\vspace{-2\medskipamount}でWeyl群作用を構成可能.

\bigskip
\bigskip
\bigskip

環 $A$ の可逆元 $f$ の変数 $\gamma$ によるべき {\red $f^\gamma$} とは何か?

\bigskip
\bigskip

(1) 可算直積環 $A^\Z$ に $A$ を対角的に埋め込んで同一視:

\medskip

\qquad\qquad
$A\ni a=(a)_{k\in\Z}\in A^\Z$.

\medskip

(2) $f^\gamma$ の定義と $\gamma$ の $A^\Z$ への埋め込み方:

\medskip

\qquad\qquad
${\red f^\gamma = (f^k)_{k\in\Z}}\in A^\Z$, \quad 
$\gamma = (k)_{k\in\Z}\in A^\Z$.


\end{small}
}
%%%%%%%%%%%%%%%%%%%%%%%%%%%%%%%%%%%%%%%%%%%%%%%%%%%%%%%%%%%%%%%%%%%%%%%%%%%%%%%
\frame{
\frametitle{従属変数への $s_i$ の作用の有理性}
\begin{small}

(1) $[f_1,[f_1,f_2]]=0$ \quad (例 $f_1=\d/\d x$, $f_2=x-a$) \quad のとき
\[
{\red
{f_1}^\gamma f_2 {f_1}^{-\gamma} 
= f_2 + [f_1,f_2]\frac{\gamma}{f_1}
}
= (1-\gamma)f_2 + \gamma f_1 f_2 {f_1}^{-1}.
\]
特に \quad $[f_1,f_2]=\pm1$ \quad 
ならば \quad ${f_1}^\gamma f_2 {f_1}^{-\gamma}=f_2\pm\dfrac{\gamma}{f_1}$.

{\blue 欲しい形の公式が出て来た!}

\bigskip\bigskip

(2) ${f_1}^2f_2-(q+q^{-1})f_1f_2f_1+f_2{f_1}^2=0$ \quad のとき
\begin{align*}
{\red {f_1}^\gamma f_2 {f_1}^{-\gamma}}
&\,{\red = q^{\pm\gamma} f_2 + (f_1 f_2 - q^{\pm1}f_2 f_1)\frac{[\gamma]_q}{f_1}}
\\
&= [1-\gamma]_q f_2 + [\gamma]_q f_1 f_2 {f_1}^{-1}
\end{align*}
ここで $[\gamma]_q=\dfrac{q^{\gamma}-q^{-\gamma}}{q-q^{-1}}$.
\quad {\blue $q$ 差分版でもよい公式が得られる!}

\end{small}
}
%%%%%%%%%%%%%%%%%%%%%%%%%%%%%%%%%%%%%%%%%%%%%%%%%%%%%%%%%%%%%%%%%%%%%%%%%%%%%%%
\frame{
\frametitle{$\tau$ 変数の非可換性をどう定めるか?}
\begin{small}

量子従属変数 $f_i$ $\longrightarrow$ Serre関係式もしくはVerma関係式

\bigskip

量子パラメーター変数 $\av_i$ $\longrightarrow$ 互いにおよび従属変数と可換

\bigskip\bigskip

{\blue 量子 $\tau$ 変数は何とどのように非可換であるべきか?}

\bigskip\bigskip

量子 $\tau$ 変数 $\tau_i$ $\longrightarrow$
互いにおよび $f_j$ と可換. 

\bigskip

しかし, {\blue 量子 $\tau$ 変数は量子パラメーター変数とは非可換(New!)}
\[
{\red \tau_i \av_j \tau_i^{-1} = \av_j + \delta_{ij}}.
\]
すなわち \quad {\red $\tau_i = \exp(\d/\d\av_i)$} 
\quad{\blue パラメーターの差分作用素!}

\end{small}
}
%%%%%%%%%%%%%%%%%%%%%%%%%%%%%%%%%%%%%%%%%%%%%%%%%%%%%%%%%%%%%%%%%%%%%%%%%%%%%%%
\frame{
\frametitle{$\tau$ 変数へのWeyl群作用の拡張}
\begin{small}

$\ts_i$ の $\tau_j$ への作用を定めれば, 
$s_i$ の $\tau_j$ への作用も定まる.

$\ts_i$ への $\av_j$ への作用は
$\d/\d \av_j$ への作用に自然に拡張される.

\bigskip

$A_3$ 型: \quad
$\ts_2(\av_2)=-\av_2$, \quad 
$\ts_2(\av_j)=\av_2+\av_j$ \quad ($j=1,3$) \quad ならば

\medskip

$\ts_2\left(\frac{\d}{\d\av_2}\right) 
= \frac{\d}{\d\av_1}-\frac{\d}{\d\av_2}+\frac{\d}{\d\av_3}$ であり, \quad
$\tau_i=\exp\left(\frac{\d}{\d\av_i}\right)$ なので
\[
\ts_2(\tau_2)=\frac{\tau_1\tau_3}{\tau_2}.
\]
さらに \quad $\tau_2 f_2^{\av_2} \tau_2^{-1} = f_2^{\av_2+1}$ \quad
($\tau_2$ は $\av_2$ を1ずらす) \quad を使うと
\begin{align*}
{\red s_2(\tau_2)}
&
=f_2^{\av_2}\ts_2(\tau_2)f_2^{-\av_2}
=f_2^{\av_2}\frac{\tau_1\tau_3}{\tau_2}f_2^{-\av_2}
\\ &
=\frac{\tau_1\tau_3}{\tau_2}f_2^{\av_2+1}f_2^{-\av_2}
=\frac{\tau_1\tau_3}{\tau_2}f_2
\,{\red = f_2\frac{\tau_1\tau_3}{\tau_2}}
\end{align*}
{\blue 欲しい形の公式が出て来た!}

\end{small}
}
%%%%%%%%%%%%%%%%%%%%%%%%%%%%%%%%%%%%%%%%%%%%%%%%%%%%%%%%%%%%%%%%%%%%%%%%%%%%%%%
\frame{
\frametitle{一般の対称化可能GCMでOK}
\begin{small}

{\blue $q$ 差分版でも}OK.

\bigskip

任意の対称化可能GCMに付随する場合

\bigskip

$U_q(\g)$ $\longrightarrow$ 
下三角 $U_q(\n_-)=\bra{\red f_1,\ldots,f_m}\ket$ $\longrightarrow$
{\blue 従属変数}.

\medskip

simple coroot {\red $\av_i$} $\longrightarrow$
{\blue パラメーター変数}
\quad {\footnotesize\magenta 従属変数とは可換とみなす}

\medskip

fundamental weight $\Lambda_i$ $\longrightarrow$
$\d/\d \av_i$ $\longrightarrow$
{\red $\tau_i=\exp(\d/\av_i)$} $\longrightarrow$
{\blue $\tau$ 変数}

\bigskip

$\av_i$, $\tau_j$ たちには自然にWeyl群が作用(それを {\red $\ts_i$} と書く).

\bigskip

欲しいWeyl群 $W=\bra s_1,\dots,s_m\ket$ の作用は
\[
{\red
s_i(x) = f_i^{\av_i} \ts_i(x) f_i^{-\av_i} 
}
\]

\end{small}
}
%%%%%%%%%%%%%%%%%%%%%%%%%%%%%%%%%%%%%%%%%%%%%%%%%%%%%%%%%%%%%%%%%%%%%%%%%%%%%%%
\frame{
\frametitle{$\tau$ 変数へのWeyl群作用の結果の多項式性}
\begin{small}

整ウェイト $\lambda=\sum_i\lambda_i\Lambda_i\in P$ ($\lambda_i\in\Z$) に対して
$\tau^\lambda=\prod_i\tau_i^{\lambda_i}$ とおく.

\bigskip
{\bf\magenta 定理:} $w\in W$ と dominant integral weight $\mu\in P_+$ に対して
\[
w(\tau^\mu) = (\text{$f_i$, $q^{\pm\av_i}$ たちの非可換多項式})\times\tau^{w(\mu)}.
\]
Kac-Moody版では
\[
w(\tau^\mu) = (\text{$f_i$, $\av_i$ たちの非可換多項式})\times\tau^{w(\mu)}.
\]

\bigskip

証明には

Kac-Moody代数の表現論の {\blue translation functor} 
{\tiny(Deodhar-Gabber-Kac 1982)} と

{\blue quantization of Lie bialgebras} {\tiny (Etingof-Kazhdan 2008, ``VI'')} を使う.

\end{small}
}
%%%%%%%%%%%%%%%%%%%%%%%%%%%%%%%%%%%%%%%%%%%%%%%%%%%%%%%%%%%%%%%%%%%%%%%%%%%%%%%
\frame{
\frametitle{translation functor の応用}
\begin{small}

$\mu,\lambda\in P_+$ (dominant integral weight),\quad $w\in W$ \quad とする. \\
$w\circ\lambda=w(\lambda+\rho)-\rho$, \quad shifted action.

\bigskip

可積分表現 $L(\mu)$ をtensorして部分加群を取る操作\\
で定義されるtranslation functor を $T=T_\lambda^\mu$ と書く[DGK]. \\
$w\in W$ とVerma加群 $M(\lambda)$ について
\begin{align*}
&
M(w\circ\lambda)\subset M(\lambda),  \quad
T(M(w\circ\lambda))\subset M(w\circ\lambda)\otimes L(\mu), 
\\ &
{\red T(M(w\circ\lambda)) = M(w\circ(\lambda+\mu))},
\end{align*}
ゆえに次の可換図式が得られる:
\[
\begin{CD}
  M(\lambda)\otimes L(\mu) @<<< M(w\circ\lambda)\otimes L(\mu) \\
  @AAA                          @AAA \\
  M(\lambda+\mu)           @<<< M(w\circ(\lambda+\mu)). \\
\end{CD}
\]
[EK]の結果よりこの形の可換図式は $U_q$ の場合にも存在する.

\end{small}
}
%%%%%%%%%%%%%%%%%%%%%%%%%%%%%%%%%%%%%%%%%%%%%%%%%%%%%%%%%%%%%%%%%%%%%%%%%%%%%%%
\frame{
\frametitle{$w(\tau^\mu)$ の多項式性の証明のスケッチ}
\begin{small}

\[
\begin{CD}
  M(\lambda)\otimes L(\mu) @<<< M(w\circ\lambda)\otimes L(\mu) \\
  @AAA                          @AAA \\
  M(\lambda+\mu)           @<<< M(w\circ(\lambda+\mu)). \\
\end{CD}
\]
この可換図式から $w(\tau^\mu)$ の多項式性が得られる.

\bigskip

$M(w\circ\lambda)$ のh.w.\ vectorの $M(\lambda)$ での
像を $f_w(\lambda)\,v_\lambda$ と書く. 

\bigskip

$w(\tau^\mu)$ の中の変数 $\av_i$ に 
$\lambda_i=\lambda(\av_i)\in\Z_{\geqq0}$ を代入したものは\\
本質的に \quad {\red $f_w(\lambda+\mu)f_w(\lambda)^{-1}$} \quad 
(2つのsingular vectorsの比) \quad に一致し, 
\[
\text{上の可換図式} \quad\implies\quad
{\red f_w(\lambda+\mu) \in f_w(\lambda) U_q(\n_-)}.
\]
$f_w(\lambda+\mu)f_w(\lambda)^{-1}$ は割り切れ, $w(\tau^\mu)$ は多項式になる.

\end{small}
}
%%%%%%%%%%%%%%%%%%%%%%%%%%%%%%%%%%%%%%%%%%%%%%%%%%%%%%%%%%%%%%%%%%%%%%%%%%%%%%%
\frame{
\frametitle{共形場理論と量子群}
\begin{small}
\begin{center}

以上の単純な枠組みは\\
野性的に登場する Painlev\'e 系の量子化\\
を理解するためにはまだ不十分!

\bigskip

B\"acklund変換(Weyl群作用)の部分だけではなく, \\
Painlev\'e 方程式の部分はどうなっているのか?

\bigskip

Painlev\'e系のLax表示やSato-Wilson表示は?

\bigskip

それらもろもろの量子化の表現論的な理解?

\bigskip
\bigskip

以下では共形場理論や量子群との関係について説明する.

\end{center}
\end{small}
}
%%%%%%%%%%%%%%%%%%%%%%%%%%%%%%%%%%%%%%%%%%%%%%%%%%%%%%%%%%%%%%%%%%%%%%%%%%%%%%
\frame{
\frametitle{微分版 Painlev\'e 系の量子化 $=$ 共形場理論}
\begin{small}

量子 Schlesinger 系(1階連立の場合)の量子化 \\
$=$ Knizhnik-Zamolodchikov方程式 (WZW model)

\bigskip\bigskip

量子 Garnier 系(2階単独の場合)の量子化 \\
$=$ 退化場 $\varphi_{1,2}$, $\varphi_{2,1}$ に付随するBPZ方程式

\bigskip

特異点の量子化 $=$ primary field $\varphi$

\medskip

2階単独の場合のみかけの特異点の量子化 $=$ 退化場 $\varphi_{1,2}$

\medskip

{\footnotesize
\href{https://twitter.com/genkuroki/status/448159501808439296}
{\tt https://twitter.com/genkuroki/status/448159501808439296}
}

\bigskip\bigskip

{\bf\magenta 予想:
任意の共形場理論は Painlev\'e 系の量子化とみなせる!}

\bigskip

{\footnotesize\blue 共形場理論でのWeyl群作用と量子 $\tau$ 変数のことはよくわかっていない.}

\end{small}
}
%%%%%%%%%%%%%%%%%%%%%%%%%%%%%%%%%%%%%%%%%%%%%%%%%%%%%%%%%%%%%%%%%%%%%%%%%%%%%%%
\frame{
\frametitle{$q$ 差分化版 Painlev\'e IV $\qPIV$ の古典版}
\begin{small}

例として以下の場合を扱おう.

\bigskip

{\red $A^{(1)}_2$} 型の場合.

\bigskip

従属変数 $F_{i+3}=F_i$, \quad パラメーター変数 $a_{i+3}=a_i$

\bigskip

$\qPIV$: \quad
\(
T_{\qPIV}(F_i)
=
a_i a_{i+1} F_{i+1}
\dfrac
{1+a_{i-1}F_{i-1}+a_{i-1}a_i F_{i-1}F_i}
{1+a_i F_i + a_i a_{i+1} F_i F_{i+1}}
\), 
\\
\qquad\quad\;\,
\( T_{\qPIV}(a_i)=a_i \).

\bigskip

対称性(Weyl群作用):
\\
\qquad\qquad
$s_i(F_i)=F_i$, \quad
$s_i(F_{i\pm1})=F_{i\pm1}\left(\dfrac{1+a_iF_i}{a_i+F_i}\right)^{\pm1}$, \quad
\\
\qquad\qquad
$s_i(a_i) = a_i^{-1}$, \quad
$s_i(a_{i\pm1}) = a_ia_{i\pm1}$.

\bigskip

{\blue 見た目が微分版と全然違う!}

\end{small}
}
%%%%%%%%%%%%%%%%%%%%%%%%%%%%%%%%%%%%%%%%%%%%%%%%%%%%%%%%%%%%%%%%%%%%%%%%%%%%%%%
\def\QqPIV{
$F_i F_{i+1}=q^2 F_{i+1} F_i$, \quad
$a_i a_j=a_j a_i$, \quad
$a_i F_j = F_j a_i$.

\bigskip

量子 $\qPIV$ (離散時間発展):
\begin{align*}
T_{\qPIV}(F_i)
&=(1+q^2 a_{i-1}F_{i-1}+q^2 a_{i-1}a_i F_{i-1}F_i)
\\
& \,\times a_i a_{i+1} F_{i+1}
\\
& \,\times(1+q^2 a_i F_i + q^2 a_i a_{i+1} F_i F_{i+1})^{-1}
\\
T_{\qPIV}(a_i)
&=a_i.
\end{align*}

対称性(Weyl群作用):
\\[\medskipamount]
\qquad\qquad
$s_i(F_i)=F_i$, 
\\
\qquad\qquad
$s_i(F_{i-1})=F_{i-1}\dfrac{a_i+F_i}{1+a_i F_i}$, \quad
$s_i(F_{i+1})=\dfrac{1+a_i F_i}{a_i+F_i}F_{i+1}$, 
\\
\qquad\qquad
$s_i(a_i) = a_i^{-1}$, \quad
$s_i(a_{i\pm1}) = a_ia_{i\pm1}$.
}
%%%%%%%%%%%%%%%%%%%%%%%%%%%%%%%%%%%%%%%%
\frame{
\frametitle{$q$ 差分版 Painlev\'e IV $\qPIV$ の量子化}
\begin{small}

\QqPIV

\bigskip

{\blue 見た目が微分版と全然違う! 量子群を使って非自明に構成した!}

\end{small}
}
%%%%%%%%%%%%%%%%%%%%%%%%%%%%%%%%%%%%%%%%%%%%%%%%%%%%%%%%%%%%%%%%%%%%%%%%%%%%%%%
\frame{
\frametitle{量子化だけではなく, $q$ 差分化も!}

微分版の量子化の公式は古典の場合とほぼ同じ.\\
特別な道具を使わない直接的な計算で色々わかる.

\bigskip

$q$ 差分版の量子化を直接的構成は難しい.\\
適切な非可換性の入れ方さえわからないことが多い.

\bigskip

{\blue 量子群の助けを借りる!}

\bigskip

$q$ 差分版 Painlev\'e IV $\qPIV$ を例に説明する. 

}
%%%%%%%%%%%%%%%%%%%%%%%%%%%%%%%%%%%%%%%%%%%%%%%%%%%%%%%%%%%%%%%%%%%%%%%%%%%%%%
\frame{
%\frametitle{}

\begin{center}
\large

以下の構成は{\magenta まだ}かなり複雑.

\bigskip\bigskip

1. 量子群の $L$-operator を定義

\bigskip

{\red ``$RLL=LLR$''}

\end{center}

}
%%%%%%%%%%%%%%%%%%%%%%%%%%%%%%%%%%%%%%%%%%%%%%%%%%%%%%%%%%%%%%%%%%%%%%%%%%%%%%%
\frame{
\frametitle{量子$L$-operatorから$\qPIV$の量子化へ 1}
\begin{small}

$A^{(1)}_2$ 型の $R$ 行列:
\begin{align*}
 R(z)
 & = (q-q^{-1}z) \sum_i E_{ii}\otimes E_{ii}
 + (1-z)       \sum_{i\ne j} E_{ii}\otimes E_{jj}
 \\ & \,
 + (q-q^{-1})  \sum_{i<j} (E_{ij}\otimes E_{ji} + z E_{ji}\otimes E_{ij}).
\end{align*}
$i,j$ は $1,2,3$ を動く. $E_{ij}$ は $3\times3$ の行列単位.

\bigskip

$A^{(1)}_2$ 型の量子 $L$-operator の定義は {\red ``$RLL=LLR$''}: 
\\
$3\times3$ 行列 $L(z)$ の成分は非可換環の元,
\begin{align*}
&
{\red R(z/w)L(z)^1 L(w)^2 = L(w)^2 L(z)^1 R(z/w)}, \quad
\\ &
L(z)^1=L(z)\otimes 1, \quad L(w)^2=1\otimes L(w).
\end{align*}


\end{small}
}
%%%%%%%%%%%%%%%%%%%%%%%%%%%%%%%%%%%%%%%%%%%%%%%%%%%%%%%%%%%%%%%%%%%%%%%%%%%%%%
\frame{
%\frametitle{}

\begin{center}
\large
2-1. 二重対角型上三角 $L$-operators の積

\bigskip

$L(z)=L_1(z)L_2(z)$

\bigskip

2-2. {\red 上三角な $L(z)$ の対角部分 $L_0$ を二重化}

\bigskip

{\red $\tL(z) = L_1(z)L_2(z)L_0$}


\end{center}

}
%%%%%%%%%%%%%%%%%%%%%%%%%%%%%%%%%%%%%%%%%%%%%%%%%%%%%%%%%%%%%%%%%%%%%%%%%%%%%%%
\frame{
\frametitle{量子$L$-operatorから$\qPIV$の量子化へ 2}
\begin{small}

次のような二重対角型の上三角 $L$-operator を考える:
\[
L_k(z) =
\begin{bmatrix}
a_{1k}   & b_{1k} & 0      \\ 
0        & a_{2k} & b_{2k} \\
z b_{3k} & 0      & a_{3k} \\
\end{bmatrix}
\]
より正確に言えば, 各々の $L_k(z)$ に関する ``$RLL=LLR$'' 関係式と\\ $L_k(z)^1L_l(w)^2=L_l(w)^2L_k(z)^1$ \quad ($k\ne l$) \quad 成分の可換性 \\
を定義関係式とする代数を考える.

\medskip


$L_0:=(\text{$L_1(z)L_2(z)$ の対角部分})=\diag(\ta_1,\ta_2,\ta_3)$ 
\;($\ta_i=a_{i1}a_{i2}$).

\medskip

{\red $L_1(z)L_2(z)$ の対角部分 $L_0=\diag(\ta_1,\ta_2,\ta_3)$ を``二重化''}:
\[
\tL(z) = L_1(z)L_2(z){\red L_0} =
\begin{bmatrix}
 \ta_1^{{\red 2}} & b_1     & c_1     \\
 z c_2   & \ta_2^{{\red 2}} & b_2     \\
 z b_3   & z c_3   & \ta_3^{{\red 2}} \\
\end{bmatrix}.
\] 

\end{small}
}
%%%%%%%%%%%%%%%%%%%%%%%%%%%%%%%%%%%%%%%%%%%%%%%%%%%%%%%%%%%%%%%%%%%%%%%%%%%%%%
\frame{
%\frametitle{}

\begin{center}
\large
3-1. $\tL(z)$ の対角行列による相似変換で\\
$c_i$ の部分を $1$ または中心元 $r$ にする.
{\small
\[
\tL(z) =
\begin{bmatrix}
 \ta_1^2 & b_1     & c_1     \\
 z c_2   & \ta_2^2 & b_2     \\
 z b_3   & z c_3   & \ta_3^2 \\
\end{bmatrix}
\mapsto
\tC \tL(z) \tC^{-1}
\begin{bmatrix}
 t_1^2   & \hb_1 & 1     \\
 rz      & t_2^2 & \hb_2 \\
 rz\hb_3 & z c_3 & t_3^2 \\
\end{bmatrix}
\] 
}

\bigskip

3-2. {\red 二重対角行列の積 $X(z)Y(rz)$ に分解する.}
{\small
\[
\tC \tL(z) \tC^{-1} = X(z) Y(rz)
\]
}

\end{center}
}
%%%%%%%%%%%%%%%%%%%%%%%%%%%%%%%%%%%%%%%%%%%%%%%%%%%%%%%%%%%%%%%%%%%%%%%%%%%%%%
\frame{
\frametitle{量子$L$-operatorから$\qPIV$の量子化へ 3}
\begin{small}

$\tC:=\diag(\tc_1,\tc_2,\tc_3):=\diag(1,c_1 c_3, c_1)$, 
\\[\medskipamount]
$\tC':=\diag(\tc_3 b_{31}, \tc_1 b_{11}, \tc_2 b_{21})$,
\\[\medskipamount]
$r:=c_1 c_3 c_2 \in \text{center}$,
\\[\medskipamount]
$t_i := \tc_i \ta_i \tc_i^{-1}$.

\bigskip

\(X(z) := \tC L_1(z) \tC'^{-1}\),\quad
\(Y(rz) := \tC' L_2(z)L_0 \tC^{-1}\).
\\
\(
X(z) =
\begin{bmatrix}
 x_1 & 1   & 0 \\
 0   & x_2 & 1 \\
 z   & 0   & x_3 \\
\end{bmatrix}
\),
\quad
\(
Y(rz) =
\begin{bmatrix}
 y_1 & 1   & 0 \\
 0   & y_2 & 1 \\
 rz  & 0   & y_3 \\
\end{bmatrix}
\).

\bigskip

このとき \(
X(z)Y(rz) = \tC\tL(z)\tC^{-1} =
\begin{bmatrix}
 t_1^2              & x_1+y_2 & 1       \\
 rz                 & t_2^2   & x_2+y_3 \\
 rz(x_3+r^{-1}y_1)  & z       & t_3^2   \\
\end{bmatrix}
\).

\end{small}
}
%%%%%%%%%%%%%%%%%%%%%%%%%%%%%%%%%%%%%%%%%%%%%%%%%%%%%%%%%%%%%%%%%%%%%%%%%%%%%%%
\frame{
%\frametitle{}
\begin{small}

前ページの続き.

\bigskip

インデックスの拡張: \quad
$x_{i+3}=r^{-1}x_i$, \quad
$y_{i+3}=r^{-1}y_i$, \quad
$t_{i+3}=r^{-1}t_i$.

\bigskip

基本関係式: \quad $\mu=1,2$ に対して, 
\begin{alignat*}{2}
 &
 x_i y_i = y_i x_i = t_i^2, \quad
 \\ &
 x_i x_{i+\mu} = q^{(-1)^{\mu-1}2} x_{i+\mu} x_i, \qquad
 & &
 x_i y_{i+\mu} = q^{-(-1)^{\mu-1}2} y_{i+\mu} x_i,
 \\ &
 y_i y_{i+\mu} = q^{(-1)^{\mu-1}2} y_{i+\mu} y_i, \qquad
 & &
 y_i x_{i+\mu} = q^{-(-1)^{\mu-1}2} x_{i+\mu} y_i,
 \\[\smallskipamount] &
 \text{$t_i$ は $t_j$, $x_j$, $y_j$ と可換}.
\end{alignat*}
この関係式は ``$3$'' を $3$ 以上の奇数に一般化しても成立している.

\bigskip

以上は $(m,n)=(3,2)$ の場合. \\
互いに素な $(m,n)$ の場合に一般化可能.\\
$n>2$ の場合にも ``$xy=q^{2a} yz$'' 型の関係式になる($a=0,\pm 1$).

\end{small}
}
%%%%%%%%%%%%%%%%%%%%%%%%%%%%%%%%%%%%%%%%%%%%%%%%%%%%%%%%%%%%%%%%%%%%%%%%%%%%%%
\frame{
%\frametitle{}

\begin{center}
\large
4. 量子化された変数 $t_i$, $x_i$, $y_i$ には\\
   {\red $\tW(A^{(1)}_2)\times\tW(A^{(1)}_1)$ が双有理作用!}

\bigskip\bigskip

$\tW(A^{(1)}_2)=\bra s_0,s_1,s_2,\pi\ket$

\bigskip

$\tW(A^{(1)}_1)=\bra r_0,r_1,\varpi\ket$

\end{center}

}
%%%%%%%%%%%%%%%%%%%%%%%%%%%%%%%%%%%%%%%%%%%%%%%%%%%%%%%%%%%%%%%%%%%%%%%%%%%%%%%
\frame{
\frametitle{量子$L$-operatorから$\qPIV$の量子化へ 4}
\begin{small}

$\tW(A^{(1)}_2)\times\tW(A^{(1)}_1)$ の作用:
\begin{align*}
 &
 s_i(x_i) 
% = x_i - (y_i x_i - y_{i+1}x_{i+1})(y_i+x_{i+1})^{-1}
 = (x_i+y_{i+1})x_{i+1}(y_i+x_{i+1})^{-1},
 \\ &
 s_i(x_{i+1})
% = x_{i+1} +(x_i+y_{i+1})^{-1}(y_i x_i - y_{i+1}x_{i+1})
 =(x_i+y_{i+1})^{-1} x_i (y_i+x_{i+1}),
 \\ &
 s_i(y_i) 
% = y_i - (x_i y_i - x_{i+1}y_{i+1})(x_i+y_{i+1})^{-1}
 = (y_i+x_{i+1})y_{i+1}(x_i+y_{i+1})^{-1},
 \\ &
  s_i(y_{i+1})
% = y_{i+1} +(y_i+x_{i+1})^{-1}(x_i y_i - x_{i+1}y_{i+1})
 =(y_i+x_{i+1})^{-1} y_i (x_i+y_{i+1}),
 \\ &
 s_i(x_{i+2})=x_{i+2},
 \quad
 s_i(y_{i+2})=y_{i+2}, 
 \\ &
 s_i(t_i)=t_{i+1},  \quad
 s_i(t_{i+1})=t_i,  \quad
 s_i(t_{j+2})=t_{j+2},
 \\ &
 \pi(x_i) = x_{i+1},  \quad
 \pi(y_i) = y_{i+1},  \quad
 \pi(t_i) = t_{i+1}.
 \\[\medskipamount]
 &
 Q_i := y_{i+2}y_{i+1} + y_{i+2}x_i + x_{i+1}x_i,
 \\ &
 r_1(x_i) 
% = x_i - r Q_{i+1}^{-1}(x_{i+3}x_{i+2}x_{i+1}-y_{i+4}y_{i+3}y_{i+2})
 = r^{-1} Q_{i+1}^{-1}y_i Q_i,
 \\ &
 r_1(y_i)
% = y_i + r(x_{i+2}x_{i+1}x_i-y_{i+3}y_{i+2}y_{i+1})Q_i^{-1}
 = r Q_{i+1} x_i Q_i^{-1},
 \\ &
 r_1(t_i) = t_i,  \quad
 \varpi(x_i)=y_i,   \quad
 \varpi(y_i)=x_i,   \quad
 \varpi(t_i)=t_i.
\end{align*}

\end{small}
}
%%%%%%%%%%%%%%%%%%%%%%%%%%%%%%%%%%%%%%%%%%%%%%%%%%%%%%%%%%%%%%%%%%%%%%%%%%%%%%
\frame{
%\frametitle{}

\begin{center}
\large
5. \quad
$a_i:=\dfrac{t_i}{t_{i+1}}$ \quad (パラメーター変数), 
\\[\medskipamount]
$F_i:=\dfrac{x_{i+1}x_i}{t_{i+1}t_i}$ \quad (従属変数),
\\[\medskipamount]
$T_{\qPIV} := r_1 \varpi$ \quad (離散時間発展).

\bigskip

このようにおくと次ページの公式が成立.

\end{center}

}
%%%%%%%%%%%%%%%%%%%%%%%%%%%%%%%%%%%%%%%%%%%%%%%%%%%%%%%%%%%%%%%%%%%%%%%%%%%%%%%
\frame{
\frametitle{量子$L$-operatorから$\qPIV$の量子化へ 5(終)}
\begin{small}

周期性: \quad $F_{i+3}=F_i$, \quad $a_{i+3}=a_i$.

\medskip

\QqPIV

\end{small}
}
%%%%%%%%%%%%%%%%%%%%%%%%%%%%%%%%%%%%%%%%%%%%%%%%%%%%%%%%%%%%%%%%%%%%%%%%%%%%%%%
\frame{
\frametitle{``変数べき''を用いたWeyl群作用の表示}
\begin{small}

$f_i := (q^{-1}-q)^{-1}\ta_i^{-1}b_i\ta_{i+1}^{-1}$, \\
$\hf_i := (q^{-1}-q)^{-1}t_i^{-1}(x_i + y_{i+1})t_{i+1}^{-1}$, \\
{\blue $\hL(z) := X(z)Y(rz)$} とおくと, 
\[
\hL(z)=
\begin{bmatrix}
t_1^2                     & (q^{-1}-q)t_1 t_2\hf_1 & 1 \\
rz                        & t_2^2                  & (q^{-1}-q)t_2 t_3\hf_2 \\
rz(q^{-1}-q)t_3 t_4 \hf_3 & z                      & t_3^2 \\
\end{bmatrix}.
\]

$\ts_i(t_i)=t_{i+1}$, $\ts_i(t_{i+1})=t_i$, $\ts_i(t_{i+2})=t_{i+2}$, 
$\ts_i(\hf_i)=\hf_i$ と定めると,
\[
s_i(\hL(z))
= f_i^{\av_i} \hL(z) f_i^{-\av_i}
= \hf_i^{\av_i} \ts_i(\hL(z)) \hf_i^{\av_i}.
\]
ここで $a_i = t_i/t_{i+1} = q^{-\av_i}$.

\end{small}
}
%%%%%%%%%%%%%%%%%%%%%%%%%%%%%%%%%%%%%%%%%%%%%%%%%%%%%%%%%%%%%%%%%%%%%%%%%%%%%%
\frame{
\frametitle{Lax表示とは}

微分版:
\[
\frac{\d L}{\d t} = AL-LA.
\]

\bigskip\bigskip

離散版: \quad 離散変換 $\rho$ について
\[
\rho(L) = g L g^{-1}.
\]

}
%%%%%%%%%%%%%%%%%%%%%%%%%%%%%%%%%%%%%%%%%%%%%%%%%%%%%%%%%%%%%%%%%%%%%%%%%%%%%%%
\frame{
\frametitle{Weyl群作用のLax表示 1}
\begin{small}

$T_{z,r}:=(\text{差分作用素 $z\mapsto rz$})$ とし, 
\[
 {\blue \hM(z) := X(z)T_{z,r}Y(z)T_{z,r} = \hL(z)T_{z,r}^2}
\]
とおく. このとき, 
\[
s_i(\hM(z)) = G_i(z) \hM(z) G_i(z)^{-1}.
\]
ここで \vspace{-2\medskipamount}
\begin{align*}
&
G_1(z)=
\begin{bmatrix}
 1 & 0 & 0 \\
g_1& 1 & 0 \\
 0 & 0 & 1 \\
\end{bmatrix},
\;
G_2(z)=
\begin{bmatrix}
 1 & 0 & 0 \\
 0 & 1 & 0 \\
 0 &g_2& 1 \\
\end{bmatrix},
\;
G_3(z)=
\begin{bmatrix}
 1 & 0 &rz^{-1}g_3 \\
 0 & 1 & 0 \\
 0 & 0 & 1 \\
\end{bmatrix},
\\ &
g_i = \frac{[\av_i]_q}{\hf_i}, \quad
[\av_i]_q = \frac{a_i^{-1}-a_i}{q-q^{-1}}, \quad
q^{-\av_i} = a_i = \frac{t_i}{t_{i+1}}.
\end{align*}

\end{small}
}
%%%%%%%%%%%%%%%%%%%%%%%%%%%%%%%%%%%%%%%%%%%%%%%%%%%%%%%%%%%%%%%%%%%%%%%%%%%%%%%
\frame{
\frametitle{Weyl群作用のLax表示 2}

$\pi(t_i)=t_{i+1}$, \; $ \pi(\hf_i)=\hf_{i+1}$, \quad 
$t_{i+3}=r^{-1}t_i$,\; \quad $\hf_{i+3}=r\hf_i$.
\[
\Lambda_3(z) =
\begin{bmatrix}
 0 & 1 & 0 \\
 0 & 0 & 1 \\
 z & 0 & 0 \\
\end{bmatrix}
\]
とおくと, 
\begin{align*}
\pi(\hM(z)) 
&= (\Lambda_3(z)T_{z,r}) \hM(z) (\Lambda_3(z)T_{z,r})^{-1}
\\
&= \Lambda_3(z) \hL(rz) \Lambda_3(r^2z)^{-1} T_{z,r}^2.
\end{align*}
\[
{\red \pi \longleftrightarrow \Lambda_3(z)T_{z,r}}
\]

}
%%%%%%%%%%%%%%%%%%%%%%%%%%%%%%%%%%%%%%%%%%%%%%%%%%%%%%%%%%%%%%%%%%%%%%%%%%%%%%%%
\frame{
\frametitle{Sato-Wilson表示とは 1}
\begin{center}
\begin{Large}

Sato-Wilson表示とは

\bigskip

Gauss 分解 $G_-\times G_+ \to G$ を通して \\

\bigskip

群 $G$ 上の簡単な方程式を

\bigskip

群 $G_-$, $G_+$ 上の\\
複雑な方程式に書き直したもの

\end{Large}

\bigskip\bigskip

{\magenta ソリトン系やPainlev\'e系の基礎!}

\end{center}
}
%%%%%%%%%%%%%%%%%%%%%%%%%%%%%%%%%%%%%%%%%%%%%%%%%%%%%%%%%%%%%%%%%%%%%%%%%%%%%%
\frame{
\frametitle{Sato-Wilson表示とは 2}
\begin{footnotesize}

Lie群 $G$ とその部分群 $G_{\pm}$ について, 
$G_-\times G_+\to G$, $(x_-,x_+)\mapsto x=x_-^{-1}x_+$ が \\
$G$ の単位元の近傍での局所微分同相を定めると仮定する.\\
$G$, $G_\pm$ のLie環を $\g$, $\g_\pm$ と書くと, $\g=\g_+\oplus\g_-$.\\
単位元の近傍で以下が成立している.

\bigskip

微分方程式: \quad $x_-P x_-^{-1}=B_+-B_-$, \quad 
$P\in\g$, \quad $B_\pm\in\g_\pm$ \quad のとき
\[
\frac{\d x}{\d t}=Px
\quad\iff\quad
\frac{\d x_+}{\d t} = B_+ x_+ \;\;\text{and}\;\;
\frac{\d x_-}{\d t} = B_+ x_- - x_- P.
\]
$P$ は時間に依存しないとする.
$B_\pm$ は初期条件と時間に依存する.

\bigskip

離散対称性: \quad $x_+\sigma=g_-^{-1}g_+$, \quad
$\sigma\in G$, \quad $g_\pm\in G_\pm$ \quad のとき
\[
\rho(x) = x \sigma
\quad\iff\quad
{\red \rho(x_+) = g_- x_+ \sigma} \;\;\text{and}\;\;
\rho(x_-) = g_- x_-.
\]
$\sigma$ は時間に依存しないとする.\\
時間発展 $\d x/\d t=Px$ と離散変換 $\rho(x)=x\sigma$ は互いに可換.

\end{footnotesize}
}
%%%%%%%%%%%%%%%%%%%%%%%%%%%%%%%%%%%%%%%%%%%%%%%%%%%%%%%%%%%%%%%%%%%%%%%%%%%%%%%
\frame{
\frametitle{Sato-Wilson表示 $\implies$ Lax表示}
\begin{small}

前ページの設定のもとで, \\
$x_-P x_-^{-1}=B_+-B_-$, \quad 
$P\in\g$, \quad $B_\pm\in\g_\pm$, \\
$x_+\sigma=g_-^{-1}g_+$, \quad
$\sigma\in G$, \quad $g_\pm\in G_\pm$ \quad $h\in G_+$ \quad かつ
\[
 {\red \rho(h)=\sigma^{-1} h \sigma}
\]
のとき{\small(たとえば $h$ は対角行列で $\sigma$ は置換行列)},
\[
  {\red L := x_+ h x_+^{-1}}
\]
とおくと, \quad $\d x/\d t=Px$,\quad $\rho(x)=x\sigma$ \quad のとき,
\[
 \frac{\d L}{\d t} = [B_+, L], \qquad
 {\red \rho(L) = g_- L g_-^{-1}}.
\]
微分方程式が差分方程式の場合も同様.

\end{small}
}
%%%%%%%%%%%%%%%%%%%%%%%%%%%%%%%%%%%%%%%%%%%%%%%%%%%%%%%%%%%%%%%%%%%%%%%%%%%%%%%
\frame{
\frametitle{Lax表示からSato-Wilson表示へ}
\begin{small}

上三角行列 $L$ の対角部分を $h=D$ と書く.

\medskip

$L=Z D Z^{-1}$ を満たす上三角行列 $x_+=Z$ は一意的ではない.

\medskip

$Z$ の対角部分は $L$ から決まらない($Z$ の不定性はちょうどそれ).

\medskip

$\tau$ 変数は本質的に上三角行列 $Z=x_+$ の対角成分の座標である.

\medskip

すべてが可換な場合は易しい.

\medskip

量子化された非可換な場合には整合性が色々非自明になる.

\bigskip

しかし、我々が扱っている場合について\\
実際に計算してみると色々うまく行っている!


\end{small}
}
%%%%%%%%%%%%%%%%%%%%%%%%%%%%%%%%%%%%%%%%%%%%%%%%%%%%%%%%%%%%%%%%%%%%%%%%%%%%%%%
\frame{
\frametitle{$\tau$ 変数の導入}
\begin{small}

$\tau$ 変数 $\tau_0$, $z_1,z_2,z_3$ を次の関係式で導入する:
\begin{align*}
&
\tau_0 r = q^{-1}r \tau_0, \quad
\tau_0 t_j = t_j \tau_0, \quad
z_i r = r z_i, \quad
z_i t_j = q^{-\delta_{ij}} t_j z_i,
\\ &
\tau_0 z_i = z_i \tau_0, \quad
z_i z_j = z_j z_i, \quad
\tau_0 \hf_j = \hf_j \tau_0, \quad
z_i \hf_j = \hf_j z_i.
\end{align*}
さらに \quad $z_{i+3}=z_i$, \quad $\tau_i=\tau_{i-1}z_i$ \quad によってインデックスを拡張.

$s_i$, $\pi$ の作用を以下のように $\tau$ 変数に拡張できる:
\begin{align*}
&
s_i(\tau_i) = \hf_i \frac{\tau_{i-1}\tau_{i+1}}{\tau_i}, \quad
s_i(\tau_j) = \tau_j \quad (i,j=0,1,2,\; i\ne j),
\\ &
\pi(\tau_i)=\tau_{i+1}.
\end{align*}

\[
{\red s_i(x) = \hf_i^{\av_i} \ts_i(x) \hf_i^{-\av_i}}
\]

\end{small}
}
%%%%%%%%%%%%%%%%%%%%%%%%%%%%%%%%%%%%%%%%%%%%%%%%%%%%%%%%%%%%%%%%%%%%%%%%%%%%%%%
\frame{
\frametitle{Sato-Wilson表示 1}
\begin{small}

$D(t):=\diag(t_1,t_2,t_3)$.

\medskip

$\exists! U(z)$: $z$ の形式べき級数を成分に持つ $3\times 3$ 行列, \\
$U(0)$ は対角成分がすべて $1$ の上三角行列,
\[
 \hM(z) = U(z) (D(t)T_{z,r})^2 U(z)^{-1}.
\]

$D_Z:=\diag(z_1,z_2,z_3)$. 

\medskip

$Z(z) := U(z)D_Z$.

\bigskip

このとき\quad $\hM(z) = Z(z) (D(qt)T_{z,r})^2 Z(z)^{-1}$.

\end{small}
}
%%%%%%%%%%%%%%%%%%%%%%%%%%%%%%%%%%%%%%%%%%%%%%%%%%%%%%%%%%%%%%%%%%%%%%%%%%%%%%%
\frame{
\frametitle{Sato-Wilson表示 2(終)}
\begin{small}

行列 $S^g_i$, $S_i$ を次のように定める:{\footnotesize
\begin{align*}
 &
 S^g_1 =
 \begin{bmatrix}
   0    & g_1^{-1} & 0 \\
   -g_1 & 0        & 0 \\
   0    & 0        & 1 \\
 \end{bmatrix},
 \quad
 S^g_2 =
 \begin{bmatrix}
   1 & 0 & 0 \\
   0 & 0 & g_2^{-1} \\
   0 & -g_2 & 0 \\
 \end{bmatrix},
 \\ &
 S_1 =
 \begin{bmatrix}
   0           & -[\av_1+1]_q & 0 \\
   [\av_1-1]_q & 0        & 0 \\
   0           & 0        & 1 \\
 \end{bmatrix},
 \quad
 S_2 =
 \begin{bmatrix}
   1 & 0           & 0 \\
   0 & 0           & -[\av_2+1]_q \\
   0 & [\av_2-1]_q & 0 \\
 \end{bmatrix}.
\end{align*}}
このとき以下が成立している:
\begin{align*}
 &
 \hM(z)
 = Z(z)(D(qt)T_{z,r})^2 Z(z)^{-1}
 = Z(z)D(qt)^2Z(r^2z)^{-1}T_{z,r}^2,
 \\[\medskipamount] &
 s_i(U(z)) = G_i U(z) S^g_i, 
 \\ &
 s_i(D_Z) = (S^g_i)^{-1} D_Z S_i, \
 \\ &
 s_i(D(t)T_{z,r}) = S_i^{-1} D(t)T_{z,r} S_i
 \\ &
 {\red s_i(Z(z)) = G_i Z(z) S_i}.
\end{align*}

\end{small}
}
%%%%%%%%%%%%%%%%%%%%%%%%%%%%%%%%%%%%%%%%%%%%%%%%%%%%%%%%%%%%%%%%%%%%%%%%%%%%%%%%
\frame{
\frametitle{量子 $q$ 差分 Painlev\'e 系と量子群の関係}
\begin{small}

$m,n$ は互いに素と仮定

\medskip

(1) $L_k(z)$ ← $A^{(1)}_{m-1}$ 型の二重対角上三角 $L$-operators

\medskip

(2) $L(z)=L_1(z)\cdots L_n(z)$ ← 上三角 $L$-operator

\medskip

(3) $\tL(z)=L(z)L_0$ ← 対角部分 $L_0$ を二重化

\medskip

(4) $\hL(z)=\tC \tL(z) \tC^{-1}$ ← 対角行列で最高次成分を定数化

\medskip

(5) $\hL(z)=X_1(z)X_2(rz)\cdots X_n(r^{n-1}z)$ ← 再度 $n$ 個に分解

\medskip

(6) $\tW(A^{(1)}_{m-1})\times\tW(A^{(1)}_{n-1})$ 作用とそのLax表示

\medskip

(7) $\tW(A^{(1)}_{n-1})$ の格子部分の作用が $q$ 差分版量子Painlev\'e系

\medskip

(8) $\tW(A^{(1)}_{m-1})$ の作用がその対称性

\medskip

(9) 対称性のSato-Wilson表示がわかっている

\end{small}
}
%%%%%%%%%%%%%%%%%%%%%%%%%%%%%%%%%%%%%%%%%%%%%%%%%%%%%%%%%%%%%%%%%%%%%%%%%%%%%%%%
%\frame{
%\frametitle{}
%\begin{small}
%
%\end{small}
%}
%%%%%%%%%%%%%%%%%%%%%%%%%%%%%%%%%%%%%%%%%%%%%%%%%%%%%%%%%%%%%%%%%%%%%%%%%%%%%%%%
%\frame{
%\frametitle{}
%\begin{small}
%
%\end{small}
%}
%%%%%%%%%%%%%%%%%%%%%%%%%%%%%%%%%%%%%%%%%%%%%%%%%%%%%%%%%%%%%%%%%%%%%%%%%%%%%%%%
%\frame{
%\frametitle{}
%\begin{small}
%
%\end{small}
%}
%%%%%%%%%%%%%%%%%%%%%%%%%%%%%%%%%%%%%%%%%%%%%%%%%%%%%%%%%%%%%%%%%%%%%%%%%%%%%%%%
%\frame{
%\frametitle{}
%\begin{small}
%
%\end{small}
%}
%%%%%%%%%%%%%%%%%%%%%%%%%%%%%%%%%%%%%%%%%%%%%%%%%%%%%%%%%%%%%%%%%%%%%%%%%%%%%%%%
%\frame{
%\frametitle{}
%\begin{small}
%
%\end{small}
%}
%%%%%%%%%%%%%%%%%%%%%%%%%%%%%%%%%%%%%%%%%%%%%%%%%%%%%%%%%%%%%%%%%%%%%%%%%%%%%%%%
%\frame{
%\frametitle{}
%%\begin{small}
%
%\begin{center}
%{\Large\bf\magenta ここから先は古いスライド原稿}
%\end{center}
%
%%\end{small}
%}
%%%%%%%%%%%%%%%%%%%%%%%%%%%%%%%%%%%%%%%%%%%%%%%%%%%%%%%%%%%%%%%%%%%%%%%%%%%%%%%
%\frame{
%\frametitle{}
%
%}
%%%%%%%%%%%%%%%%%%%%%%%%%%%%%%%%%%%%%%%%%%%%%%%%%%%%%%%%%%%%%%%%%%%%%%%%%%%%%%%
%\frame{
%\frametitle{}
%
%}
%%%%%%%%%%%%%%%%%%%%%%%%%%%%%%%%%%%%%%%%%%%%%%%%%%%%%%%%%%%%%%%%%%%%%%%%%%%%%%
%\frame{
%\frametitle{}
%}
%%%%%%%%%%%%%%%%%%%%%%%%%%%%%%%%%%%%%%%%%%%%%%%%%%%%%%%%%%%%%%%%%%%%%%%%%%%%%%
%\frame{
%\frametitle{}
%}
%%%%%%%%%%%%%%%%%%%%%%%%%%%%%%%%%%%%%%%%%%%%%%%%%%%%%%%%%%%%%%%%%%%%%%%%%%%%%%%
%\frame{
%\frametitle{}
%
%}
%%%%%%%%%%%%%%%%%%%%%%%%%%%%%%%%%%%%%%%%%%%%%%%%%%%%%%%%%%%%%%%%%%%%%%%%%%%%%%%
%\frame{
%\frametitle{}
%
%}
%%%%%%%%%%%%%%%%%%%%%%%%%%%%%%%%%%%%%%%%%%%%%%%%%%%%%%%%%%%%%%%%%%%%%%%%%%%%%%%
%\frame{
%\frametitle{}
%
%}
%%%%%%%%%%%%%%%%%%%%%%%%%%%%%%%%%%%%%%%%%%%%%%%%%%%%%%%%%%%%%%%%%%%%%%%%%%%%%%%
%\frame{
%\frametitle{}
%
%}
%%%%%%%%%%%%%%%%%%%%%%%%%%%%%%%%%%%%%%%%%%%%%%%%%%%%%%%%%%%%%%%%%%%%%%%%%%%%%%%
%\frame{
%\frametitle{}
%
%}
%%%%%%%%%%%%%%%%%%%%%%%%%%%%%%%%%%%%%%%%%%%%%%%%%%%%%%%%%%%%%%%%%%%%%%%%%%%%%%%
%\frame{
%\frametitle{}
%
%}
%%%%%%%%%%%%%%%%%%%%%%%%%%%%%%%%%%%%%%%%%%%%%%%%%%%%%%%%%%%%%%%%%%%%%%%%%%%%%%
\end{document}
%%%%%%%%%%%%%%%%%%%%%%%%%%%%%%%%%%%%%%%%%%%%%%%%%%%%%%%%%%%%%%%%%%%%%%%%%%%%%%
