%%%%%%%%%%%%%%%%%%%%%%%%%%%%%%%%%%%%%%%%%%%%%%%%%%%%%%%%%%%%%%%%%%%%%%%%%%%%%%
%
% Quantum Groups and Quantizations of Isomonodromic Systems
%
% Gen KUROKI  (Tohoku University, Japan)
%
% 5 March 2007, 
%
% Exploration of New Structures and Natural Constructions
% in Mathematical Physics
%
% Graduate School of Mathematics (Room 509), 
% Nagoya University, March 5--8, 2007. 
% On the occasion of Professor Akihiro Tsuchiya's retirement 
%
% Version 1.0  2006-03-05
% Version 1.1  2008-09-14 minor corrections
% Version 1.2  2008-09-19 p.33 corrected + minor corrections
%
%%%%%%%%%%%%%%%%%%%%%%%%%%%%%%%%%%%%%%%%%%%%%%%%%%%%%%%%%%%%%%%%%%%%%%%%%%%%%%
\def\TITLE{%
  \red{Quantum Groups and} \\
%  \magenta{$q$-Difference Versions of} \\
  \red{Quantizations of Isomonodromic Systems}
  }
\def\AUTHOR{\blue{Gen KUROKI \quad (Tohoku University, Japan)}}
\def\DATE{
 5 March 2007
 \\[5mm]
 Exploration of New Structures and Natural Constructions \\
 in Mathematical Physics
 \\[5mm]
 Graduate School of Mathematics (Room 509), \\
 Nagoya University, March 5--8, 2007. \\
 On the occasion of Professor Akihiro Tsuchiya's retirement 
}
\def\MYLOGO{}
\def\RIGHTHEADER{Quantum Groups and Quantizations of Isomonodromic Systems}
\def\LEFTHEADER{Gen KUROKI \quad (Tohoku Univ.)}
%%%%%%%%%%%%%%%%%%%%%%%%%%%%%%%%%%%%%%%%%%%%%%%%%%%%%%%%%%%%%%%%%%%%%%%%%%%%%%
\documentclass[25pt,a4paper,landscape]{foils}
\usepackage[dvipdfm]{color,graphicx}
\usepackage{amsmath,amssymb}
\AtBeginDvi{\special{papersize=\the\paperwidth,\the\paperheight}}
%\AtBeginDvi{\special{dviout !p}} % Presentation switch [ESC]
\AtBeginDvi{\special{dviout df}} % Fit Display Size [5]
%%%%%%%%%%%%%%%%%%%%%%%%%%%%%%%%%%%%%%%%%%%%%%%%%%%%%%%%%%%%%%%%%%%%%%%%%%%%%%
\setlength{\foilheadskip}{0mm}
\setlength{\parindent}{0mm}
\setlength{\fboxrule}{1.5pt}
%%%%%%%%%%%%%%%%%%%%%%%%%%%%%%%%%%%%%%%%%%%%%%%%%%%%%%%%%%%%%%%%%%%%%%%%%%%%%%
\setlength{\topmargin}{-20mm}
\setlength{\textheight}{170mm}
\setlength{\oddsidemargin}{-10mm}
\setlength{\evensidemargin}{-10mm}
\setlength{\textwidth}{266mm}
%%%%%%%%%%%%%%%%%%%%%%%%%%%%%%%%%%%%%%%%%%%%%%%%%%%%%%%%%%%%%%%%%%%%%%%%%%%%%%
\allowdisplaybreaks
%%%%%%%%%%%%%%%%%%%%%%%%%%%%%%%%%%%%%%%%%%%%%%%%%%%%%%%%%%%%%%%%%%%%%%%%%%%%%%
\newcommand\black[1]{{\color{black}#1}}
\newcommand\blue[1]{{\color{blue}#1}}
\newcommand\red[1]{{\color{red}#1}}
\newcommand\magenta[1]{{\color{magenta}#1}}
\newcommand\green[1]{{\color{green}#1}}
\newcommand\cyan[1]{{\color{cyan}#1}}
\newcommand\yellow[1]{{\color{yellow}#1}}
\newcommand\white[1]{{\color{white}#1}}
%%%%%%%%%%%%%%%%%%%%%%%%%%%%%%%%%%%%%%%%%%%%%%%%%%%%%%%%%%%%%%%%%%%%%%%%%%%%%%
\newcommand\Z{{\mathbb Z}} % rational integers
\newcommand\Q{{\mathbb Q}} % rational numbers
\newcommand\R{{\mathbb R}} % real numbers
\newcommand\C{{\mathbb C}} % complex numbers
%%%%%%%%%%%%%%%%%%%%%%%%%%%%%%%%%%%%%%%%%%%%%%%%%%%%%%%%%%%%%%%%%%%%%%%%%%%%%%
\newcommand\A{{\mathcal A}}
\newcommand\ac{\alpha^\vee}
\newcommand\Ad{\mathop{\mathrm{Ad}}\nolimits}
\newcommand\ad{\mathop{\mathrm{ad}}\nolimits}
\newcommand\Atilde{{\widetilde\A}}
\newcommand\affgl{\widehat{\lie{gl}}}
\newcommand\Aut{\mathop{\mathrm{Aut}}\nolimits}
\newcommand\B{{\mathcal B}}
\renewcommand\b{{\mathfrak b}}
\newcommand\ball{b_{\mathrm{all}}}
\newcommand\bigzerol{\smash{\hbox{\large $0$}}}             % �����̑傫�ȃ[��
\newcommand\bigzerou{\smash{\lower.3ex\hbox{\large $0$}}}   % �E��̑傫�ȃ[��
\newcommand\bigstarl{\smash{\hbox{\Large $*$}}}           % �����̑傫�Ȑ�
\newcommand\bigstaru{\smash{\lower.3ex\hbox{\Large $*$}}} % �E��̑傫�Ȑ�
\newcommand\bra{\langle}
\newcommand\cH{{\mathcal H}}
\renewcommand\d{\partial}
\newcommand\diag{\mathop{\mathrm{diag}}\nolimits}
\newcommand\dummyITEM{\phantom{\ITEM}}
\newcommand\dummyItem{\phantom{\Item}}
\newcommand\ec{\eps^\vee}
\newcommand\eps{\varepsilon}
\newcommand\h{{\mathfrak h}}
\newcommand\G{{\mathcal G}}
\newcommand\g{{\mathfrak g}}
%\renewcommand\iff{\Leftrightarrow}
%\renewcommand\implies{\Rightarrow}
\newcommand\isom{\cong}
\newcommand\isomto{\overset\sim\rightarrow}
\newcommand\ITEM{\blue{$\bullet\;$}}
\newcommand\Item{\blue{$\circ\;$}}
\newcommand\K{{\mathcal K}}
\newcommand\KK{{\mathbb K}}
\newcommand\Ktilde{{\widetilde\K}}
\newcommand\ket{\rangle}
\renewcommand\L{{\mathbb L}}
\newcommand\lie{\mathrm}
\newcommand\minv{{\widetilde m}}
\newcommand\n{{\mathfrak n}}
\newcommand\ninv{{\widetilde n}}
\newcommand\MOD{\mathop{\mathrm{mod}}\nolimits}
\renewcommand\pmod[1]{\;(\MOD #1)}
\newcommand\PP{{\widetilde P}}
\newcommand\QQ{{\widetilde Q}}
\newcommand\qP[1]{$q{\mathrm P}_{\mathrm #1}$}
\newcommand\rtilde{{\tilde r}}
\newcommand\stilde{{\tilde s}}
\newcommand\W[1]{{W(A_{#1-1})}}
\newcommand\WAffine[1]{{W\bigl(A^{(1)}_{#1}\bigr)}}
\newcommand\Waffine[1]{\WAffine{#1-1}}
\newcommand\WTilde[1]{{\widetilde W\bigl(A^{(1)}_{#1}\bigr)}}
\newcommand\Wtilde[1]{\WTilde{#1-1}}
\newcommand\x{{\mathbf x}}
\newcommand\X{{\mathbb X}}
\newcommand\Xspace{{\mathcal X}}
\newcommand\Yspace{{\mathcal Y}}
%%%%%%%%%%%%%%%%%%%%%%%%%%%%%%%%%%%%%%%%%%%%%%%%%%%%%%%%%%%%%%%%%%%%%%%%%%%%%%
%
% \DDOTS
%
\makeatletter
\def\DDOTS{\mathinner{%
  \mkern1mu\raise10\p@\vbox{\kern7\p@\hbox{.}}
  \mkern2mu\raise4\p@\hbox{.}%
  \mkern2mu\raise-2\p@\hbox{.}\mkern1mu}}
\def\VDOTS{\vbox{\baselineskip4\p@ \lineskiplimit\z@
    \kern6\p@\hbox{.}\hbox{\vphantom{.}}\hbox{.}\hbox{\vphantom{.}}\hbox{.}}}
\makeatother
%%%%%%%%%%%%%%%%%%%%%%%%%%%%%%%%%%%%%%%%%%%%%%%%%%%%%%%%%%%%%%%%%%%%%%%%%%%%%%
%
% claim environment
%
\makeatletter
\newenvironment{claim}[1][\theoremname]{\par
  \normalfont
  \topsep6\p@\@plus6\p@ \trivlist
  \item[\hskip\labelsep{\bfseries\blue{#1}}]%
  \ignorespaces
}{%
  \endtrivlist
}
\newcommand{\theoremname}{Claim.}
\makeatother
%%%%%%%%%%%%%%%%%%%%%%%%%%%%%%%%%%%%%%%%%%%%%%%%%%%%%%%%%%%%%%%%%%%%%%%%%%%%%%
%
% proof environment
%
\makeatletter
\newenvironment{proof}[1][\proofname]{\par
  \normalfont
  \topsep6\p@\@plus6\p@ \trivlist
  \item[\hskip\labelsep{\bfseries\magenta{#1}}\@addpunct{\bfseries\magenta{.}}]%
  \ignorespaces
}{%
  \endtrivlist
}
\newcommand{\proofname}{Proof}
\makeatother
%%%%%%%%%%%%%%%%%%%%%%%%%%%%%%%%%%%%%%%%%%%%%%%%%%%%%%%%%%%%%%%%%%%%%%%%%%%%%%
%
% \qed
%
\makeatletter
\def\qedsymbol{\RIfM@\bgroup\else$\bgroup\aftergroup$\fi
%  \vcenter{\hrule\hbox{\vrule\@height.8em\kern.6em\vrule}\hrule}\egroup}
\vcenter{%
  \hrule\@height1.5pt%
  \hbox{\vrule\@width1.5pt\@height0.8em\kern.6em\vrule\@width1.5pt}%
  \hrule\@height1.5pt%
  }\egroup}
\def\qed{\blue{\RIfM@\else\unskip\nobreak\fi\quad\qedsymbol}}
\makeatother
%%%%%%%%%%%%%%%%%%%%%%%%%%%%%%%%%%%%%%%%%%%%%%%%%%%%%%%%%%%%%%%%%%%%%%%%%%%%%%
\leftheader{\LEFTHEADER}
\rightheader{\RIGHTHEADER}
%\rightfooter{quad\textsf{\thepage}} % default
%%%%%%%%%%%%%%%%%%%%%%%%%%%%%%%%%%%%%%%%%%%%%%%%%%%%%%%%%%%%%%%%%%%%%%%%%%%%%%
\begin{document}
%%%%%%%%%%%%%%%%%%%%%%%%%%%%%%%%%%%%%%%%%%%%%%%%%%%%%%%%%%%%%%%%%%%%%%%%%%%%%%
\title{\TITLE}
\author{\bf\AUTHOR}
\date{\bf\DATE}
\MyLogo{\MYLOGO} 
%%%%%%%%%%%%%%%%%%%%%%%%%%%%%%%%%%%%%%%%%%%%%%%%%%%%%%%%%%%%%%%%%%%%%%%%%%%%%%
\maketitle
%%%%%%%%%%%%%%%%%%%%%%%%%%%%%%%%%%%%%%%%%%%%%%%%%%%%%%%%%%%%%%%%%%%%%%%%%%%%%%

%\foilhead[-10mm]{}
%
%\smallskip\smallskip\smallskip
%
%\begin{center}
%\begin{Large}
%\red{\bf Quantum Groups and} \\
%\magenta{\bf $\mathbf{q}$-Difference Versions of} \\[3mm]
%\red{\bf Quantizations of Isomonodromic Systems}
%\end{Large}
%\end{center}
%
%\begin{center}
%\blue{\bf\AUTHOR}
%\end{center}
%
%\begin{center}
%{\bf\DATE}
%\end{center}

%%%%%%%%%%%%%%%%%%%%%%%%%%%%%%%%%%%%%%%%%%%%%%%%%%%%%%%%%%%%%%%%%%%%%%%%%%%%%%

%\foilhead[-10mm]{Contents}
%
%{\bf\S1. Introduction}
%\\ 
%\ITEM
%\red{Quantizations of isomonodromic deformations}
%\\
%\ITEM
%\red{Quantizations of discrete symmetries}
%\\[3mm]
%{\bf\S2. Complex powers of Chevalley generators in quantum groups}
%\\
%\ITEM 
%\red{quantum $q$-difference version of the Noumi-Yamada 
%\blue{({\tt math.QA/0012028})} birational Weyl group action 
%arising from a nilpotent Poisson algebra}
%\\
%\ITEM
%\red{reconstruction of the quantum $q$-difference version of the 
%birational Weyl group action constructed by Koji Hasegawa 
%\blue{({\tt math.QA/0703036})}}
%\\[3mm]
%{\bf\S3. Quantization of the $\Wtilde{m}\times\Wtilde{n}$ action of KNY}
%\\
%\ITEM
%\red{Assume that $m,n$ are mutually prime.}
%\\
%\ITEM
%\red{$\K_{m,n} = Q(\A_{m,n}) =$ 
%Quantization of $\C(x_{ik}|1\leqq i\leqq m, 1\leqq k\leqq n)$}
%\\
%\ITEM
%\red{$\Wtilde{m}\times\Wtilde{n}$ action on $\K_{m,n}$
%as algebra automorphisms}
%\\
%\ITEM
%\red{Complex powers of corrected Chevalley generators}

%%%%%%%%%%%%%%%%%%%%%%%%%%%%%%%%%%%%%%%%%%%%%%%%%%%%%%%%%%%%%%%%%%%%%%%%%%%%%%

\foilhead[-10mm]{\S1. Introduction}

\red{\bf Isomonodromic Systems
\\
= Isomonodromic Deformations + Discrete Symmetries}

Jimbo-Miwa-Ueno, \blue{Physica 2D, 1981}.
\\
Jimbo-Miwa, \blue{Physica 2D, 4D, 1981}.

\ITEM Isomonodromic deformations
\\ \dummyITEM
$=$ monodromy preserving deformations (differential equations) of 
\\ \dummyItem\phantom{$=$}
rational connections on ${\mathbb P}^1_\C$ (or on compact Riemann surfaces).
\\[4mm]
\Item
Deformation parameters $=$ time variables
\\ \dummyItem
$=$ positions of singularities
and irregular types of irregular singularities

\ITEM Discrete symmetries
\\ \dummyITEM
$=$ discrete group actions compatible with isomonodromic deformations.
\\ \dummyITEM
$=$ B\"acklund transformations of deformation differential equations

%%%%%%%%%%%%%%%%%%%%%%%%%%%%%%%%%%%%%%%%%%%%%%%%%%%%%%%%%%%%%%%%%%%%%%%%%%%%%%

\foilhead[-10mm]{\red{Quantizations of isomonodromic deformations}}

\ITEM
the Schlesinger equations
$\longrightarrow$ the Knizhnik-Zamolodchikov equations 
\\ \dummyITEM
\quad(Reshetikhin \blue{(LMP26, 1992)}, Harnad \blue{({\tt hep-th/9406078})})
\\
The KZ equations have \red{hypergeometric integral solutions}. 

\ITEM
the generalized Schlesinger equations 
\ (rank-$1$ irreg.\ sing.\ at $\infty$)
\\ 
$\longrightarrow$ 
the generalized Knizhnik-Zamolodchikov equations 
\\ \phantom{$\longrightarrow$}
(Babujian-Kitaev \blue{(for $\lie{sl}_2$, JMP39, 1998)},
\\ \phantom{$\longrightarrow$} \ 
Felder-Markov-Tarasov-Varchenko \blue{(for any $\g$, math.QA/0001184)})
\\
The gen.\ KZ equations have \red{confluent hypergeometric integral solutions}.

\blue{\bf Conjecture.}
Any quantum isomonodromic system has (confluent or non-confluent)
\red{hypergeometric integral solutions}.

\blue{\bf Problem.}
Quantize \red{the discrete symmetries} \ %
(the Schlesinger transfor\-mations, 
the birational Weyl group actions, $\ldots$).

%%%%%%%%%%%%%%%%%%%%%%%%%%%%%%%%%%%%%%%%%%%%%%%%%%%%%%%%%%%%%%%%%%%%%%%%%%%%%%

\foilhead[-10mm]{\red{Quantizations of discrete symmetries}}

\ITEM
the $q$-difference version of the birational Weyl group action 
\\ \dummyITEM
(Kajiwara-Noumi-Yamada \blue{({\tt nlin.SI/0012063})}) 
\\
$\longrightarrow$ 
the quantum $q$-difference version of the birat.\ Weyl group action
\\ \phantom{$\longrightarrow$}
(Koji Hasegawa \blue{({\tt math.QA/0703036})})

\ITEM
the higher Painlev\'e equation of type $A^{(1)}_l$ 
with $\WTilde{l}$ symmetry
\\ \dummyITEM
(rank-2 irr.\ sing.\ at $\infty$)
(Noumi-Yamada \blue{({\tt math.QA/9808003})})
\\
$\longrightarrow$ 
the quantum higher Painlev\'e equation type $A^{(1)}_l$ 
with $\WTilde{l}$ sym.
\\ \phantom{$\longrightarrow$}
(Hajime Nagoya \blue{(math.QA/0402281)})

\ITEM
the birational Weyl group action arising from a nilpotent Poisson algebra 
\\ \dummyITEM
(Noumi-Yamada \blue{({\tt math.QA/0012028})})
\\
\ \ \qquad $\downarrow$
\red{complex powers of Chevalley generators} in the Kac-Moody algebra
\qquad
the Weyl group action on the quotient skew field of $U(\n)\otimes U(\h)$

%%%%%%%%%%%%%%%%%%%%%%%%%%%%%%%%%%%%%%%%%%%%%%%%%%%%%%%%%%%%%%%%%%%%%%%%%%%%%%

\newpage

\ITEM
the dressing chains
\ (Shabat-Yamilov \blue{(LMJ2, 1991)}, 
\\ \dummyITEM
(Veselov-Shabat \blue{(FAA27, 1993)}, V.~E.~Adler \blue{(Phys.D73, 1994)})
\\
$\longrightarrow$ the quantum dressing chains
\ (Lipan-Rasinariu \blue{(hep-th/0006074)})

\Item
$R(z) := z+P^{12}$, \ %
\(
 L_k(z) :=
 \begin{bmatrix}
  x_k                   & 1    \\
  x_k \d_k - \eps_k + z & \d_k \\
 \end{bmatrix}
\), \ %
$\d_k=\d/\d x_k$.
\\[4mm]
\Item
\(
 R(z-w)L_k(z)^1L_k(w)^2 = L_k(w)^2 L_k(z)^1 R(z-w)
\).
\\[2mm]
\Item
Assume \red{$n=2g+1$}, \ $x_{k+\red{n}}=x_{k}$, \ %
$\eps_{k+\red{n}}=\eps_k+\kappa$ \ (quasi-periodicity).
\\[2mm]
\Item
The fundamental algebra of the quantum dressing chain is 
\red{not} the algebra generated by $x_k,\d_k$ 
\red{but} the algebra generated by $f_k:=\d_k+x_{k+1}$.
The Hamiltonian of the dressing chain can be expressed with $f_k$.

\red{\bf Duality.}
the quantum quasi-periodic dressing chain with period \red{$n$}
\\
\phantom{\bf Duality.}
$\isom$ the quantum higher Painlev\'e equation of type $A^{(1)}_{\red{n}-1}$.

\Item
Thus the $\WTilde{2g}$ symmetry of the dressing chain is also quantized. 

%%%%%%%%%%%%%%%%%%%%%%%%%%%%%%%%%%%%%%%%%%%%%%%%%%%%%%%%%%%%%%%%%%%%%%%%%%%%%%

\newpage

\begin{center}
{\bf Quantizations of Isomonodromic Systems}

\begin{tabular}{|c||c|}
\hline
 \blue{\bf Classical}
 &
 \blue{\bf Quantum}
 \\ \hline\hline
 \vphantom{$\big|$}
 Poisson algebra $S(\g)=\C[\g^*]$
 &
 Non-commutative algebra $U(\g)$
 \\ \hline\hline
 (generalized) Schlesinger eq.
 &
 (generalized) KZ eq.
 \\ \hline
 $A^{(1)}_l$ higher Painlev\'e eq.
 &
 quantum $A^{(1)}_l$ higher Painlev\'e eq.
 \\
 with $\WTilde{l}$ symmetry
 &
 with $\WTilde{l}$ symmetry
 \\ \hline
 dressing chain
 &
 quantum dressing chain
 \\
 with quasi-period $2g+1$
 &
 with quasi-period $2g+1$
 \\
 ($\isom$ $A^{(1)}_{2g}$ higher Painlev\'e eq.)
 &
 ($\isom$ quantum $A^{(1)}_{2g}$ higher Painlev\'e eq.)
 \\
 and its $\WTilde{2g}$-symmetry
 &
 and its $\WTilde{2g}$-symmetry
 \\ \hline
 birational Weyl group action
 &
 the ``$U_q(\g)\to U(\g)$'' limit of
 \\
 arising from nilpotent Poisson 
 &
 \red{\underline{the Weyl group action on}}
 \\
 algebra of NY
 &
 \red{\underline{$Q(U_q(\n)\otimes U_q(\h))$ constructed in {\bf\S2}}}
 \vphantom{$\Big|$}
 \\ \hline
\end{tabular}
\\[1mm]
{\small
(As far as the speaker knows, 
the \red{\underline{red-colored}} results are new.)
}
\end{center}

%%%%%%%%%%%%%%%%%%%%%%%%%%%%%%%%%%%%%%%%%%%%%%%%%%%%%%%%%%%%%%%%%%%%%%%%%%%%%%

\newpage

\begin{center}
{\bf Quantum $q$-difference Versions of Discrete Symmetries}

\begin{tabular}{|c||c|}
\hline
 \blue{\bf $q$-difference Classical}
 &
 \blue{\bf $q$-difference Quantum}
 \\ \hline\hline
 \vphantom{$\big|$}
 Poisson algebra $\C[G^*]$
 &
 Non-commutative algebra $U_q(\g)$
 \\
 ($G=$ Poisson Lie group)
 &
 (quantum universal enveloping alg.)
 \\ \hline\hline
 \red{\underline{$q$-difference version of}} the
 &
 \red{\underline{Weyl group action on the quotient}}
 \\
 NY birat. Weyl group action 
 &
 \red{\underline{skew field $Q(U_q(\n_-)\otimes U_q(\h))$}}
 \\
 arising from nilp.\ Poisson alg.
 &
 \red{\underline{constructed in {\bf\S2}}}
 \vphantom{$\Big|$}
 \\ \hline
 $q$-difference version of
 &
 quantum $q$-difference version of
 \\
 birational Weyl Group action
 &
 birational Weyl Group action
 \\
 of KNY \blue{({\tt nlin.SI/0012063})}
 &
 of Hasegawa \red{(\underline{reconstructed in {\bf\S2}})}
 \vphantom{$\Big|$}
 \\ \hline
 $\Wtilde{m}\times\Wtilde{n}$ 
 &
 \red{\underline{quantum $\Wtilde{m}\times\Wtilde{n}$}}
 \vphantom{$\Big|$}
 \\
 action of KNY
 &
 \red{\underline{action of {\bf\S3}}}
 \vphantom{$\Big|$}
 \\ \hline
\end{tabular}
\\[1mm]
{\small
(As far as the speaker knows, 
the \red{\underline{red-colored}} results are new.)
}
\end{center}

%%%%%%%%%%%%%%%%%%%%%%%%%%%%%%%%%%%%%%%%%%%%%%%%%%%%%%%%%%%%%%%%%%%%%%%%%%%%%%

\foilhead[-10mm]
{\S2. Complex powers of Chevalley generators \\ in quantum groups}

\blue{\bf Problem 1.}
Find a quantum $q$-difference version of
the Noumi-Yamada birational Weyl group action 
arising from a nilpotent Poisson algebra
\blue{({\tt math.QA/0012028})}.
\\[2mm]
\blue{\bf Answer.}
Using \red{complex powers of Chevalley generators in quantum groups},
we can naturally construct the quantum $q$-difference version
of the NY birational action arising from a nilpotent Poisson algebra.

\blue{\bf Problem 2.}
Find a quantum group interpretation of
the quantum $q$-difference version of the birational Weyl group action
constructed by Koji Hasegawa \blue{({\tt math.QA/0703036})}.
\\[2mm]
\blue{\bf Answer.}
Using \red{complex powers of Chevalley generators in quantum groups},
we can reconstruct the Hasegawa quantum birat.\ action. 

%%%%%%%%%%%%%%%%%%%%%%%%%%%%%%%%%%%%%%%%%%%%%%%%%%%%%%%%%%%%%%%%%%%%%%%%%%%%%%

\foilhead[-10mm]{\red{Complex powers of Chevalley generators}}

\ITEM $A=[a_{ij}]_{i,j\in I}$, symmetrizable GCM.
\ $d_ia_{ij}=d_ja_{ji}$.
\ $q_i:=q^{d_i}$.
\\[2mm]
\ITEM
$U_q(\n_-)=\bra\, f_i\mid i\in I\,\ket :=$ 
maximal nilpotent subalgebra of $U_q(\g(A))$.
\\[2mm]
\ITEM
$U_q(\h) = \bra\, a_\lambda=q^{\lambda}\mid \lambda\in\h\,\ket :=$
Cartan subalgebra of $U_q(\g(A))$.
\\[2mm]
\ITEM
$\ac_i :=$ simple coroot, \ %
$\alpha_i :=$ simple root, \ %
$a_i := a_{\alpha_i} = q^{\alpha_i}=q_i^{\ac_i}$.
\\[2mm]
\ITEM
$\K_A := Q(U_q(\n_-)\otimes U_q(\h)) =$ 
the quotient skew field of $U_q(\n_-)\otimes U_q(\h)$.
\\[2mm]
\ITEM
$a_\lambda=q^{\lambda}$ regarded as a \red{central element} of $\K_A$ 
is called a \blue{\bf parameter}.

\blue{\bf Complex powers of $f_i$:}
\quad (Iohara-Malikov \blue{({\tt hep-th/9305138})})
\\[2mm]
\ITEM
The action of $\Ad(f_i^\lambda)x=f_i^\lambda x f_i^{-\lambda}$ 
on $\K_A$ is well-defined.
\\[4mm]
\Item
\(
 f_i^\lambda f_j f_i^{-\lambda}
 = q_i^{-\lambda} f_j + [\lambda]_{q_i} (f_if_j-q_i^{-1}f_jf_i)f_i^{-1}
 \\[2mm] \phantom{\text{\ITEM $f_i^\lambda f_j f_i^{-\lambda}$}}
 = [1-\lambda]_{q_i} f_j + [\lambda]_{q_i} f_if_jf_i^{-1}
\)
\quad\qquad if $a_{ij}=-1$,
\\[4mm]
where \ $[x]_q := (q^x-q^{-x})/(q-q^{-1})$. 

%%%%%%%%%%%%%%%%%%%%%%%%%%%%%%%%%%%%%%%%%%%%%%%%%%%%%%%%%%%%%%%%%%%%%%%%%%%%%%

\foilhead[-10mm]{\red{Verma relations $\iff$ Coxeter relations}}

\blue{\bf Verma relations of Chevalley generators $f_i$ in $U_q(\n_-)$:}
\\[2mm] \quad\qquad
$f_i^a f_j^{a+b} f_i^{b} = f_j^b f_i^{a+b} f_j^{a}$
\ \ \ ($a,b\in\Z_{\geqq0}$)
\qquad if $a_{ij}a_{ji}=1$.
\\ \quad\qquad
{\small (formulae for non-simply-laced cases are omitted)}
\\
{\small (Lusztig, Introduction to Quantum Groups, Prop.39.3.7 or Lemma 42.1.2.)}
\\[2mm]
\ITEM
Verma relations can be extended to the complex powers $f_i^\lambda$.
\\[2mm]
\ITEM
$\rtilde_i \lambda \rtilde_i^{-1}
= \lambda - \bra\ac_i,\lambda\ket\alpha_i$
\ for $\lambda\in\h$
\quad (Weyl group action on parameters).
\\[4mm]
\ITEM
\red{Verma relations of $f_i$'s 
$\iff$ 
Coxeter relations of $R_i:=f_i^{\ac_i}\rtilde_i$'s.}
\\[2mm]
\Item
\(
R_i^2
=
f_i^{\ac_i}\rtilde_if_i^{\ac_i}\rtilde_i
=f_i^{\ac_i}f_i^{-\ac_i}\rtilde_i^2
=1.
\)
\\[2mm]
\Item
\(
R_iR_jR_i
=
 f_i^{\ac_i}\rtilde_i
 f_j^{\ac_j}\rtilde_j
 f_i^{\ac_i}\rtilde_i
=
 f_i^{\ac_i}f_j^{\ac_i+\ac_j}f_i^{\ac_j}
 \rtilde_i\rtilde_j\rtilde_i
\\ \text{\dummyItem}\!\!\!
=
 f_j^{\ac_j}f_i^{\ac_i+\ac_j}f_j^{\ac_i}
 \rtilde_j\rtilde_i\rtilde_j
=
 f_j^{\ac_j}\rtilde_j
 f_i^{\ac_i}\rtilde_i
 f_j^{\ac_j}\rtilde_j
=R_jR_iR_j
\quad \text{if $a_{ij}a_{ji}=1$}.
\)
\\ \dummyItem $\!\!\!$
{\small (formulae for non-simply-laced cases are omitted)}

%%%%%%%%%%%%%%%%%%%%%%%%%%%%%%%%%%%%%%%%%%%%%%%%%%%%%%%%%%%%%%%%%%%%%%%%%%%%%%

\newpage

\begin{claim}[Theorem.]
 $\Ad(R_i)=\Ad(f_i^{\ac_i}\rtilde_i)$ ($i\in I$) generate the action of 
 the Weyl group on $\K_A$ as algebra automorphisms. 
 This is the \red{quantum $q$-difference version of 
 the Noumi-Yamada birational Weyl group action arising 
 from a nilpotent Poisson algebra}
 \blue{({\tt math.QA/0012028})}.
 \end{claim}

\begin{claim}[Example.]
 If $a_{ij}=-1$, then
 \\[2mm]
 \qquad
 $f_i^2f_j-(q_i+q_i^{-1})f_if_jf_i+f_jf_if_i=0$,
 \\[3mm]
 \(
  \qquad
  \Ad(R_i)f_j = f_i^{\ac_i}f_jf_i^{-\ac_i} 
  = q_i^{-\ac_i} f_j + [\ac_i]_{q_i}(f_if_j-q_i^{-1}f_jf_i)f_i^{-1}
  \\ \phantom{\qquad\Ad(R_i)f_j = f_i^{\ac_i}f_jf_i^{-\ac_i}}
  = [1-\ac_i]_{q_i} f_j + [\ac_i]_{q_i} f_if_jf_i^{-1},
 \)
 \\[5mm] \qquad
 $\Ad(R_i)a_i = \rtilde_i a_i \rtilde_i^{-1} = a_i^{-1}$, \quad
 $\Ad(R_i)a_j = \rtilde_i a_j \rtilde_i^{-1} = a_ia_j$.
 \\[2mm]
 In particular, as the $q\to 1$ limit, we have 
 \\[3mm] \qquad
 \(
  \Ad(R_i)f_j 
  = f_j + \alpha_i^\vee[f_i,f_j]f_i^{-1}
  = (1-\ac_i)f_j + \ac_i f_if_jf_i^{-1}.
 \)
 \end{claim}

%%%%%%%%%%%%%%%%%%%%%%%%%%%%%%%%%%%%%%%%%%%%%%%%%%%%%%%%%%%%%%%%%%%%%%%%%%%%%%

\foilhead[-10mm]
{\red{Truncated $q$-Serre relations and Weyl group actions}}

\blue{\bf Assumptions:}\ \ %
%\ITEM $d_ia_{ij}=d_ja_{ji}$, \ $q_i:=q^{d_i}$. 
\\[2mm]
\ITEM $k_ik_j=k_jk_i$, \ $k_i f_j k_i^{-1} = q_i^{-a_{ij}} f_j$.
\ (the action of the Cartan subalgebra)
\\[2mm]
\ITEM $f_i f_j = q_i^{\pm(-a_{ij})} f_j f_i$ \ ($i\ne j$).
\qquad\blue{(truncated $q$-Serre relations)}
\\[2mm]
\ITEM
$f_{i1}:=f_i\otimes 1$, \ \ %
$f_{i2}:=k_i^{-1}\otimes f_i$.
\quad$\!\!$\red{($f_{i1}+f_{i2}=$ ``coproduct of $f_i$'')}

\blue{\bf Skew field $\K_H$ generated by $F_i, a_i$:}
\\[2mm]
\ITEM
$\K_H :=$ the skew field generated by \ %
$F_i := a_i^{-1} f_{i1}^{-1} f_{i2}$, \ $a_i = q^{\alpha_i}$.
\\[2mm]
\ITEM
Then \quad
$F_i F_j = q_i^{\pm2(-a_{ij})} F_j F_i$ \quad ($i\ne j$), 
\qquad $a_i\in$ center of $\K_H$.
\\[2mm]
\ITEM
$\rtilde_i a_j\rtilde_i^{-1} = a_i^{-a_{ij}}a_j$.
\quad (the action of the Weyl group on parameters).

\blue{\bf Theorem.}
Put $R_i := (f_{i1}+f_{i2})^{\ac_i}\rtilde_i$. \\
Then $\Ad(R_i)$'s generate the action of the Weyl group on $\K_H$. 

%%%%%%%%%%%%%%%%%%%%%%%%%%%%%%%%%%%%%%%%%%%%%%%%%%%%%%%%%%%%%%%%%%%%%%%%%%%%%%

\foilhead[-10mm]{\red{$q$-binomial theorem and explicit formulae of actions}}

\ITEM
Applying \blue{the $q$-binomial theorem} 
to $f_{i1}f_{i2}=q_i^{-2}f_{i2}f_{i1}$, 
we obtain 
\\[2mm]\quad
\(
 (f_{i1}+f_{i2})^{\ac_i}
 = \dfrac{(a_i^{-1}F_i)_{i,\infty}}{(a_i F_i)_{i,\infty}} f_{i1}^{\ac_i},
\)
\quad where 
$\displaystyle (x)_{i,\infty} := \prod_{\mu=0}^\infty(1+q_i^{2\mu}x)$.
\\[2mm]
\blue{\bf Explicit Formulae.}
If $i\ne j$, then
\\[2mm]\quad
$\Ad(R_i)F_i = F_i$,
\\[2mm]\quad
\(
 \Ad(R_i)F_j = 
 \begin{cases}
  \displaystyle
  F_j
  \,
  \prod_{\mu=0}^{-a_{ij}-1}\dfrac{1+q_i^{2\mu} a_iF_i}{a_i+q_i^{2\mu}F_i}
  & \text{if}\ F_i F_j = q_i^{+2(-a_{ij})} F_j F_i, \\[4mm]
  \displaystyle
  \,
  \prod_{\mu=0}^{-a_{ij}-1}\dfrac{a_i+q_i^{2\mu}F_i}{1+q_i^{2\mu} a_iF_i}
  \,\,\,
  F_j 
  & \text{if}\ F_i F_j = q_i^{-2(-a_{ij})} F_j F_i. 
 \end{cases}
\)

\ITEM
These formulae coincide with those of 
\red{the quantum $q$-difference Weyl group action 
constructed by Koji Hasegawa} \blue{({\tt math.QA/0703036})}.

%%%%%%%%%%%%%%%%%%%%%%%%%%%%%%%%%%%%%%%%%%%%%%%%%%%%%%%%%%%%%%%%%%%%%%%%%%%%%%

\foilhead[-10mm]
{\S3. Quantization of \\ the $\Wtilde{m}\times\Wtilde{n}$ action of KNY}

\blue{\bf Problem 3.}
For any integers $m,n\geqq2$, construct \\
\blue{\bf(a)} a non-commutative skew field $\K_{m,n}$ and \\
\blue{\bf(b)} an action 
of $\Wtilde{m}\times\Wtilde{n}$ on $\K_{m,n}$ 
as alg.\ automorphisms \\
which is a quantization of the Kajiwara-Noumi-Yamada action \\
of $\Wtilde{m}\times\Wtilde{n}$ on %
$\C(x_{ik}|1\leqq i\leqq m,1\leqq k\leqq n)$.

\blue{\bf Answer.}\ %
If $m,n$ are \red{\bf mutually prime}, \\
then we can construct a quantization of the KNY action.

\blue{\bf Tools.}\\
\blue{\bf(a)} Gauge invariant subalgebras
of quotients of affine quantum groups, \\
\blue{\bf(b)} Complex powers of \red{corrected} Chevalley generators.

%%%%%%%%%%%%%%%%%%%%%%%%%%%%%%%%%%%%%%%%%%%%%%%%%%%%%%%%%%%%%%%%%%%%%%%%%%%%%%

\foilhead[-10mm]{\red{The KNY discrete dynamical systems}}

Kajiwara-Noumi-Yamada, \blue{\tt nlin.SI/0106029}, 
\\ \quad
Discrete dynamical systems 
with $W\bigl(A^{(1)}_{m-1}\times A^{(1)}_{n-1}\bigr)$ symmetry.
\\
Kajiwara-Noumi-Yamada, \blue{\tt nlin.SI/0112045}.
\\
Noumi-Yamada, \blue{\tt math-ph/0203030}.

\blue{(1)} Action of $\Wtilde{m}\times\Wtilde{n}$ 
as algebra automorphisms on the rational function field 
$\C(x_{ik}|1\leqq i\leqq m,1\leqq k\leqq n)$.

\blue{(2)} Lax representations $\implies$
$q$-difference isomonodromic systems.

\blue{(3)} Poisson brackets are, however, \red{\bf not} given.

\begin{claim}[First Problem.]
 Usually quantization replaces Poisson brackets \\ by commutators.
 How to find an appropriate quantization of \\
 $\C(x_{ik}|1\leqq i\leqq m,1\leqq k\leqq n)$
 without Poisson brackets?
\end{claim}

%%%%%%%%%%%%%%%%%%%%%%%%%%%%%%%%%%%%%%%%%%%%%%%%%%%%%%%%%%%%%%%%%%%%%%%%%%%%%%

\foilhead[-10mm]
{\red{Minimal representations of Borel subalgebra of $U_q(\affgl_m)$}}

\ITEM
$\B_{m,n} :=$
the associative algebra over $\C(q,r',s')$ generated by \\
$a_{ik}^{\pm1},b_{ik}^{\pm1}$ ($i,k\in\Z$) with following fundamental relations:
\\ \quad
$a_{i+m,k} = r' a_{ik}$, \ %
$a_{i,k+n} = s' a_{ik}$, \ %
$b_{i+m,k} = r' b_{ik}$, \ %
$b_{i,k+n} = s' b_{ik}$, 
\\[2mm]
\quad
%\begin{large}%
$a_{ik} b_{ik}    = q^{-1} b_{ik}    a_{ik}$, \quad 
$a_{ik} b_{i-1,k} = q      b_{i-1,k} a_{ik}$.
%\end{large}
\\[2mm]
All other combinations 
from $\{a_{ik},b_{ik}\}_{1\leqq i\leqq m,\, 1\leqq k\leqq n}$ commute.

\ITEM
$U_q(\b_-) = \bra\,t_i,f_i\mid i\in\Z\,\ket :=$ 
the lower Borel subalgebra of $U_q(\affgl_m)$ \\
with fundamental relations: \quad
$t_{i+m}=r't_i$, 
$f_{i+m}=f_i$,
\\ \quad
$t_i t_j = t_j t_i$, \quad
$t_i f_i     = q^{-1} f_i     t_i$, \quad
$t_i f_{i-1} = q      f_{i-1} t_i$, 
\\ \quad
$f_i f_j = f_j f_i$ \quad ($j\not\equiv i\pm1\pmod m$),
\\ \quad
$f_i^2f_{i\pm1}-(q+q^{-1})f_if_{i\pm1}f_i+f_{i\pm1}f_i^2=0$
\quad ($q$-Serre relations).

\ITEM
For each $k$, the algebra homomorphism $U_q(\b_-)\to\B_{m,n}$ is given by
\\ \qquad
$t_i\mapsto a_{ik}$, \quad $f_i\mapsto a_{ik}^{-1}b_{ik}$.
\quad \blue{(minimal representations of $U_q(\b_-)$)}


%%%%%%%%%%%%%%%%%%%%%%%%%%%%%%%%%%%%%%%%%%%%%%%%%%%%%%%%%%%%%%%%%%%%%%%%%%%%%%

\foilhead[-10mm]{\red{$RLL=LLR$ relations (Quantum group)}}

%  for (i=1; i<=m; i++) {
%    for (j=1; j<=m; j++) {
%      if (i == j) {
%        R = R + scalar(m^2,q - z/q)*matrix_tensor(EE(m,i,i), EE(m,j,j));
%      } else {
%        R = R + scalar(m^2,1 - z  )*matrix_tensor(EE(m,i,i), EE(m,j,j));
%      }
%    }
%  }
%  for (i=1; i<=m; i++) {
%    for (j=i+1; j<=m; j++) {
%      R = R + scalar(m^2,(q - 1/q)  )*matrix_tensor(EE(m,i,j),EE(m,j,i));
%      R = R + scalar(m^2,(q - 1/q)*z)*matrix_tensor(EE(m,j,i),EE(m,i,j));
%    }
%  }

\blue{$R$-matrix:}
%\begin{small}%
\(\displaystyle
 \quad
 R(z) 
 :=\sum_{i=1}^m (q-z/q) E_{ii}\otimes E_{ii}
 + \sum_{i\ne j} (1-z) E_{ii}\otimes E_{jj}
 \\ 
 \hphantom{\text{\blue{$R$-matrix: }}}
 \hphantom{\quad R(z) :}
 + \sum_{i<j} \left( 
                 (q-q^{-1})  E_{ij}\otimes E_{ji} 
               + (q-q^{-1})z E_{ji}\otimes E_{ij}
   \right).
\)
%\end{small}

\blue{$L$-operators:}\quad
\(
 L_k(z) :=
 \begin{bmatrix}
  a_{1k}    & b_{1k} &        & \\
            & a_{2k} & \DDOTS & \\
            &        & \DDOTS & b_{m-1,k} \\
  b_{mk}\,z &        &        & a_{mk} \\
 \end{bmatrix}.
\)

\blue{$RLL=LLR$ relations:}
\\[2mm]
\qquad\qquad
$R(z/w) L_k(z)^1 L_k(w)^2 = L_k(w)^2 L_k(z)^1 R(z/w)$, 
\\[2mm]
\qquad\qquad
$L_k(z)^1 L_l(w)^2 = L_l(w)^2 L_k(z)^1$
\qquad ($k\not\equiv l\pmod n$), 
\\[2mm]
where 
$L_k(z)^1:=L_k(z)\otimes 1$, 
$L_k(w)^2:=1\otimes L_k(w)$.

%%%%%%%%%%%%%%%%%%%%%%%%%%%%%%%%%%%%%%%%%%%%%%%%%%%%%%%%%%%%%%%%%%%%%%%%%%%%%%

\foilhead[-10mm]
{\red{Gauge invariant subalgebra $\A_{m,n}=\B_{m,n}^\G$ of $\B_{m,n}$}}

\blue{Gauge group:}\quad $\G := (\C^\times)^{mn}\ni g=(g_{ik})$.
\ \ $g_{i+m,k}=g_{ik}$, $g_{i,k+n}=g_{ik}$.
\\[2mm]
\blue{Gauge transformation}: 
The algebra automorphism of $\B_{m,n}$ is given by
\\[2mm] \qquad\qquad
$a_{ik}\mapsto g_{ik}a_{ik}g_{i,k+1}^{-1}$, \quad
$b_{ik}\mapsto g_{ik}b_{ik}g_{i+1,k+1}^{-1}$,
\\[2mm] \quad i.e. \quad
$L_k(z) \mapsto g_k L_k(z) g_{k+1}^{-1}$
\quad ($g_k:=\diag(g_{1k},g_{2k},\ldots,g_{mk})$).
\\[4mm]
\ITEM
\red{Assume that $m,n$ are \red{\bf mutually prime} integers $\geqq2$.}
\\[2mm]
\ITEM
$\minv :=$ mod-$n$ inverse of $m$ \ %
($\minv m \equiv 1 \pmod n$, \ $\minv=1,2,\ldots,n-1$).
\\[2mm]
\ITEM
The gauge invariant subalgebra $\A_{m,n}:=\B_{m,n}^\G$ of $\B_{m,n}$ 
is generated by
\\[2mm] \qquad
$x_{ik}^{\pm1} := 
\left(a_{ik}(b_{ik}b_{i+1,k+1}\cdots b_{i+\minv m-1})^{-1}\right)^{\pm1}$,
\\[2mm] \qquad
$\ball^{\pm1} := 
\left(\prod_{i=1}^m\prod_{k=1}^n b_{ik}\right)^{\pm1} \in$ 
center of $\B_{m,n}$.

\ITEM
\red{$\K_{m,n} :=$ the quotient skew field of $\A_{m,n}$ 
is an {\bf appropriate quanti\-zation} 
of the rational function field 
$\C(x_{ik}|1\leqq i\leqq m,1\leqq k\leqq n)$.}

%%%%%%%%%%%%%%%%%%%%%%%%%%%%%%%%%%%%%%%%%%%%%%%%%%%%%%%%%%%%%%%%%%%%%%%%%%%%%%

\foilhead[-10mm]
{\red{$q$-commutation relations of $x_{ik}$'s}}

\ITEM
\(
 B := \{\,(\mu\MOD m,\;\mu\MOD n)\in\Z/m\Z\times\Z/n\Z
      \mid \mu=0,1,\ldots,\minv m-1 \,\}
\).
\\[2mm]
\ITEM
\(
 p_{\mu\nu} := 
 \begin{cases}
 q & \text{if $(\mu\MOD m,\;\nu\MOD n)\in B$,}
 \\
 1   & \text{otherwise.}
 \end{cases}
\)
\\[2mm]
\ITEM
$q_{\mu\nu} := (p_{\mu\nu}/p_{\mu-1,\nu})^2 \in \{1,q^{\pm2}\}$.
\qquad
\blue{(definition of $q_{\mu\nu}$)}

\blue{\bf Fundamental relations of $x_{ik}$'s:}
\\[2mm]\qquad
$x_{i+m,k} = r x_{ik}, \quad x_{i,k+n} = s x_{ik}$
\qquad $\!$($r:=r'^{1-\minv m}$, $s:=s'^{1-\minv m}$),
\\[2mm]\qquad
$x_{i+\mu,k+\nu} x_{ik} = q_{\mu\nu} x_{ik} x_{i+\mu,k+\nu}$
\qquad\ ($0\leqq\mu<m,\ 0\leqq\nu<n$).
\\[4mm]
\blue{\bf Example.}
If \red{$(m,n)=(2,3)$}, then $\minv = 2$ and 
\\
\(
  [p_{\mu\nu}]
  =
  \begin{bmatrix}
   q & 1 & q \\
   q & q & 1   \\
  \end{bmatrix},
  \quad
  [q_{\mu\nu}]
  =
  \begin{bmatrix}
   1 & q^{-2} & q^2    \\
   1 & q^2    & q^{-2} \\
  \end{bmatrix}
  \quad
  \begin{small}
  \left(
   \begin{array}{l}
    \mu=0,1   \\
    \nu=0,1,2 \\
   \end{array}
  \right).
  \end{small}
\)
\\
 $x_{11}x_{11}=      x_{11}x_{11}$, \quad
 $x_{12}x_{11}=q^{-2}x_{11}x_{12}$, \quad $\!\!\!\!\!\!$
 $x_{13}x_{11}=q^2   x_{11}x_{13}$, \quad \\
 $x_{21}x_{11}=      x_{11}x_{21}$, \quad
 $x_{22}x_{11}=q^2   x_{11}x_{22}$, \quad
 $x_{23}x_{11}=q^{-2}x_{11}x_{23}$. 

%%%%%%%%%%%%%%%%%%%%%%%%%%%%%%%%%%%%%%%%%%%%%%%%%%%%%%%%%%%%%%%%%%%%%%%%%%%%%%

\newpage

\begin{claim}[Example.]
 \blue{(1)}
 If \red{$(m,n)=(2,2g+1)$}, then $\minv = g+1$ and 
 \begin{equation*}
%  [p_{\mu\nu}]
%  =
%  \begin{bmatrix}
%   q & 1 & q & \cdots & 1 & q \\
%   q & q & 1 & \cdots & q & 1 \\
%  \end{bmatrix},
%  \quad
  [q_{\mu\nu}]
  =
  \begin{bmatrix}
    1 & q^{-2} & q^2    & \cdots & q^{-2} & q^2    \\
    1 & q^2    & q^{-2} & \cdots & q^2    & q^{-2} \\
  \end{bmatrix}
  \quad
  \begin{small}
  \left(
   \begin{array}{l}
    \mu=0,1   \\
    \nu=0,1,2,\ldots,2g-1,2g \\
   \end{array}
  \right).
  \end{small}
 \end{equation*}
 $1<k\leqq n$ $\implies$
 $x_{1k}x_{11} = q^{(-1)^{k-1}2}x_{11}x_{1k}$, 
 $x_{2k}x_{11} = q^{(-1)^k    2}x_{11}x_{2k}$.

 \blue{(2)}
 If \red{$(m,n)=(2g+1,2)$}, then $\minv = 1$ and 
 \begin{equation*}
  [p_{\mu\nu}]
  =
  \begin{bmatrix}
   q & 1 \\
   1 & q \\
   q & 1 \\
   \VDOTS & \VDOTS \\
   1 & q \\
   q & 1 \\
  \end{bmatrix},
  \quad
  [q_{\mu\nu}]
  =
  \begin{bmatrix}
   1      & 1      \\
   q^{-2} & q^2    \\
   q^2    & q^{-2} \\
   \VDOTS & \VDOTS \\
   q^{-2} & q^2    \\
   q^2    & q^{-2} \\
  \end{bmatrix}
  \quad
  \begin{small}
  \left(
   \begin{array}{l}
    \mu=0,1,2,\ldots,2g-1,2g \\
    \nu=0,1 \\
   \end{array}
  \right).
  \end{small}
 \end{equation*}
 \blue{Observation:} %
 $\A_{\red{2,n}}\isom\A_{\red{n,2}}$,
 $x_{ik}\leftrightarrow x_{ki}$, 
 $q\leftrightarrow q$,
 $r\leftrightarrow s$, 
 $s\leftrightarrow r$.
\end{claim}

%%%%%%%%%%%%%%%%%%%%%%%%%%%%%%%%%%%%%%%%%%%%%%%%%%%%%%%%%%%%%%%%%%%%%%%%%%%%%%

\newpage

\begin{claim}[Example.]
 \blue{(1)}
 If \red{$(m,n)=(3,4)$}, then $\minv = 3$ and 
 \begin{equation*}
  [p_{\mu\nu}]
  =
  \begin{bmatrix}
   q & 1 & q & q \\
   q & q & 1 & q \\
   q & q & q & 1 \\
  \end{bmatrix},
  \quad
  [q_{\mu\nu}]
  =
  \begin{bmatrix}
    1 & q^{-2} & 1      & q^2    \\
    1 & q^2    & q^{-2} & 1      \\
    1 & 1      & q^2    & q^{-2} \\
  \end{bmatrix}
  \quad
  \begin{small}
  \left(
   \begin{array}{l}
    \mu=0,1,2 \\
    \nu=0,1,2,3 \\
   \end{array}
  \right).
  \end{small}
 \end{equation*}
 $x_{12}x_{11}=q^{-2}x_{11}x_{12}$, 
 $x_{13}x_{11}=      x_{11}x_{13}$, 
 $x_{14}x_{11}=q^2   x_{11}x_{14}$, 
 $\ldots$

 \blue{(2)}
 If \red{$(m,n)=(4,3)$}, then $\minv = 1$ and 
 \begin{equation*}
  [p_{\mu\nu}]
  =
  \begin{bmatrix}
   q & 1 & 1 \\
   1 & q & 1 \\
   1 & 1 & q \\
   q & 1 & 1 \\
  \end{bmatrix},
  \quad
  [q_{\mu\nu}]
  =
  \begin{bmatrix}
    1      & 1      & 1      \\
    q^{-2} & q^2    & 1      \\
    1      & q^{-2} & q^2    \\
    q^2    & 1      & q^{-2} \\
  \end{bmatrix}
  \quad
  \text{
  \begin{small}%
  \(
  \left(
   \begin{array}{l}
    \mu=0,1,2,3 \\
    \nu=0,1,2 \\
   \end{array}
  \right).
  \)
  \end{small}
  }
 \end{equation*}
 \blue{Observation:} %
 $\A_{\red{3,4}}\isom\A_{\red{4,3}}$, \quad
 $x_{ik}\leftrightarrow x_{ki}$, 
 $q\leftrightarrow q$,
 $r\leftrightarrow s$, 
 $s\leftrightarrow r$.
\end{claim}

%%%%%%%%%%%%%%%%%%%%%%%%%%%%%%%%%%%%%%%%%%%%%%%%%%%%%%%%%%%%%%%%%%%%%%%%%%%%%%

\newpage

\begin{claim}[Example.]
 \blue{(1)}
 If \red{$(m,n)=(3,5)$}, then $\minv = 2$ and 
 \begin{equation*}
  [p_{\mu\nu}]
  =
  \begin{bmatrix}
   q & 1 & 1 & q & 1 \\
   1 & q & 1 & 1 & q \\
   q & 1 & q & 1 & 1 \\
  \end{bmatrix},
  \quad
  [q_{\mu\nu}]
  =
  \begin{bmatrix}
    1      & 1      & q^{-2} & q^2    & 1      \\
    q^{-2} & q^2    & 1      & q^{-2} & q^2    \\
    q^2    & q^{-2} & q^2    & 1      & q^{-2} \\
  \end{bmatrix}.
 \end{equation*}
 $x_{12}x_{11}=      x_{11}x_{12}$, 
 $x_{13}x_{11}=q^{-2}x_{11}x_{13}$, 
 $x_{14}x_{11}=q^2   x_{11}x_{14}$, 
% $x_{15}x_{11}=      x_{11}x_{15}$, 
 $\ldots$

 \blue{(2)}
 If \red{$(m,n)=(5,3)$}, then $\minv = 2$ and 
 \begin{equation*}
  [p_{\mu\nu}]
  =
  \begin{bmatrix}
   q & 1 & q \\
   q & q & 1 \\
   1 & q & q \\
   q & 1 & q \\
   q & q & 1 \\
  \end{bmatrix},
  \quad
  [q_{\mu\nu}]
  =
  \begin{bmatrix}
    1      & q^{-2} & q^2    \\
    1      & q^2    & q^{-2} \\
    q^{-2} & 1      & q^2    \\
    q^2    & q^{-2} & 1      \\
    1      & q^2    & q^{-2} \\
  \end{bmatrix}.
 \end{equation*}
 \blue{Observation:} %
 $\A_{\red{3,5}}\isom\A_{\red{5,3}}$, \quad
 $x_{ik}\leftrightarrow x_{ki}$, 
 $q\leftrightarrow q$,
 $r\leftrightarrow s$, 
 $s\leftrightarrow r$.
\end{claim}

%%%%%%%%%%%%%%%%%%%%%%%%%%%%%%%%%%%%%%%%%%%%%%%%%%%%%%%%%%%%%%%%%%%%%%%%%%%%%%

\foilhead[-10mm]{\red{Symmetries of $\A_{m,n}$}}

\begin{claim}[\red{Duality.}]
 The algebra isomorphism $\A_{\red{m,n}} \isom \A_{\red{n,m}}$ is given by
 \begin{center}
 \(
  x_{ik}\leftrightarrow x_{ki}, \quad
  q\leftrightarrow q, \quad
  r\leftrightarrow s, \quad
  s\leftrightarrow r.
 \)
 \end{center}
\end{claim}

\begin{claim}[Reversal.]
 The algebra involution of $\A_{m,n}$ is given by
 \begin{center}
 \(
  x_{ik}\leftrightarrow x_{-i,-k}, \quad
  q\leftrightarrow q^{-1}, \quad
  r\leftrightarrow s^{-1}, \quad
  s\leftrightarrow r^{-1}.
 \)
 \end{center}
\end{claim}

\begin{claim}[Translation.]
 For any integers $\mu,\nu$,\\
 the algebra automorphism of $\A_{m,n}$ is given by
 \begin{center}
 \(
  x_{ik}\mapsto x_{i+\mu,k+\nu}, \quad
  q\mapsto q, \quad
  r\mapsto r, \quad
  s\mapsto s.
 \)
 \end{center}
\end{claim}

%%%%%%%%%%%%%%%%%%%%%%%%%%%%%%%%%%%%%%%%%%%%%%%%%%%%%%%%%%%%%%%%%%%%%%%%%%%%%%

\foilhead[-10mm]{\red{Extended affine Weyl groups $\Wtilde{m}$, $\Wtilde{n}$}}

\ITEM
$\red{\Wtilde{m}}:=\bra r_0,r_1,\ldots,r_{m-1},\omega\ket$
with fundamental relations:
\\[2mm]\quad
$r_ir_j = r_jr_i$ \ ($j\not\equiv i,i+1\pmod m$), \ %
$r_ir_{i+1}r_i = r_{i+1}r_ir_{i+1}$, \ %
$r_i^2 = 1$, %
\\[2mm]\quad
$\omega r_i \omega^{-1} = r_{i+1}$ \ ($r_{i+m}=r_i$).
\\[2mm]
\Item
$T_i := r_{i-1}\cdots r_2r_1 \omega r_{m-1} \cdots r_{i+1}r_i$
\quad (translations).
\\[2mm]
\Item
\(
\Wtilde{m} 
= \bra r_1,r_2,\ldots,r_{m-1}\ket\ltimes\bra T_1,T_2,\ldots,T_m\ket 
\isom S_m\ltimes\Z^m.
\)

\bigskip\bigskip

\ITEM
$\red{\Wtilde{n}}:=\bra s_0,s_1,\ldots,s_{n-1},\varpi\ket$
with fundamental relations:
\\[2mm]\quad
$s_ks_l = s_ls_k$ \ ($l\not\equiv k,k+1\pmod n$), \ %
$s_ks_{k+1}s_k = s_{k+1}s_ks_{k+1}$, \ %
$s_k^2 = 1$, %
\\[2mm]\quad
$\varpi s_k \varpi^{-1} = s_{k+1}$ \ ($s_{k+n}=s_k$).
\\[2mm]
\Item
$U_k := s_{k-1}\cdots s_2s_1 \varpi s_{n-1} \cdots s_{k+1}s_k$
\quad (translations).
\\[2mm]
\Item
\(
\Wtilde{n}
= \bra s_1,s_2,\ldots,s_{n-1}\ket\ltimes\bra U_1,U_2,\ldots,U_n\ket
\isom S_n\ltimes\Z^n.
\)

%%%%%%%%%%%%%%%%%%%%%%%%%%%%%%%%%%%%%%%%%%%%%%%%%%%%%%%%%%%%%%%%%%%%%%%%%%%%%%

\foilhead[-10mm]
{\red{Explicit formulae of the action of $\Wtilde{m}$ on $\K_{m,n}$}}

\ITEM
$\Wtilde{m}=\bra r_0,r_1,\ldots,r_{m-1},\omega\ket$
acts on $\K_{m,n}=Q(\A_{m,n})$ by
\begin{align*}
 r_i(x_{il})
 &= x_{il} - s^{-1}\frac{c_{i,l+1}-c_{i+1,l+2}}{P_{i,l+1}}
 = s P_{il} x_{i+1,l} P_{i,l+1}^{-1},
 \\
 r_i(x_{i+1,l})
 &= x_{i+1,l} + s^{-1}\frac{c_{il}-c_{i+1,l+1}}{P_{il}} 
 = s^{-1} P_{il}^{-1} x_{il} P_{i,l+1},
 \\
 r_i(x_{jl})
 &= x_{jl} \quad (j\not\equiv i,i+1\pmod m),
 \\[\bigskipamount]
 \omega(x_{jl}) 
 &= x_{j+1,l},
\end{align*}
where \quad
$c_{ik} := x_{ik}x_{i,k+1}\cdots x_{i,k+n-1}$ 
\quad and
\begin{equation*}
 P_{ik} :=
 \sum_{l=1}^n
 \overbrace{x_{ik}x_{i,k+1}\cdots x_{i,k+l-2}}^{l-1}
 \overbrace{x_{i+1,k+l}x_{i+1,k+l+1}\cdots x_{i+1,k+n-1}}^{n-l}.
\end{equation*}

%%%%%%%%%%%%%%%%%%%%%%%%%%%%%%%%%%%%%%%%%%%%%%%%%%%%%%%%%%%%%%%%%%%%%%%%%%%%%%

\foilhead[-10mm]
{\red{Explicit formulae of the action of $\Wtilde{n}$ on $\K_{m,n}$}}

\ITEM
$\Wtilde{n}=\bra s_0,s_1,\ldots,s_{n-1},\varpi\ket$
acts on $\K_{m,n}=Q(\A_{m,n})$ by
\begin{align*}
 s_k(x_{jk})
 &= x_{jk} - r^{-1}\frac{d_{j+1,k}-d_{j+2,k+1}}{Q_{j+1,k}}
 = r Q_{j+1,k}^{-1} x_{j,k+1} Q_{jk},
 \\
 s_k(x_{j,k+1})
 &= x_{j,k+1} + r^{-1}\frac{d_{jk}-d_{j+1,k+1}}{Q_{jk}} 
 = r^{-1} Q_{j+1,k} x_{jk} Q_{jk},
 \\
 s_k(x_{jl})
 &= x_{jl} \quad (l\not\equiv k,k+1\pmod n),
 \\[\bigskipamount]
 \varpi(x_{jl})
 &= x_{j,l+1},
\end{align*}
where \quad
$d_{ik} := x_{i+m-1,k}\cdots x_{i+1,k}x_{ik}$
\quad and
\begin{equation*}
 Q_{ik} :=
 \sum_{j=1}^m
 \overbrace{x_{i+m-1,k+1}\cdots x_{i+j+1,k+1}x_{i+j,k+1}}^{m-j}
 \overbrace{x_{i+j-2,k}\cdots x_{i+1,k}x_{ik}}^{j-1}.
\end{equation*}

%%%%%%%%%%%%%%%%%%%%%%%%%%%%%%%%%%%%%%%%%%%%%%%%%%%%%%%%%%%%%%%%%%%%%%%%%%%%%%

\foilhead[-10mm]{\red{Duality of the extended affine Weyl group actions}}

\ITEM
$x^{(m,n)}_{ik} := x_{ik} \in \A_{m,n}$, 
%\\
$c^{(m,n)}_{ik} := c_{ik} \in \A_{m,n}$,
$P^{(m,n)}_{ik} := P_{ik} \in \A_{m,n}$, 
%\\
%$d^{(m,n)}_{ik} := d_{ik} \in \A_{m,n}$,
%$Q^{(m,n)}_{ik} := Q_{ik} \in \A_{m,n}$, 
\\
\dummyITEM
$s^{(m,n)}_i:=$ ($s_i$-action on $\K_{m,n}$), 
$\omega^{(m,n)}:=$ ($\omega$-action on $\K_{m,n}$),
etc.

\ITEM
The algebra isomorphism $\theta:\A_{\red{m,n}}\isomto\A_{\red{n,m}}$ is defined by
%\red{%
\begin{center}
\(
 \theta(x^{(m,n)}_{ik}) = x^{(n,m)}_{-k,-i}, \ %
 \theta(q) = q^{-1}, \ %
 \theta(r) = s^{-1}, \ %
 \theta(s) = r^{-1}.
\)
\end{center}
%}

\ITEM
Then 
\\ \qquad
$\theta(c^{(m,n)}_{ik})=d^{(n,m)}_{-k-n+1,-i}$, \quad
$\,\theta(P^{(m,n)}_{ik})=Q^{(n,m)}_{-k-n+1,-i-1}$, 
\\ \qquad
$\theta(d^{(m,n)}_{ik})=c^{(n,m)}_{-k,-i-m+1}$, \quad
$\theta(Q^{(m,n)}_{ik})=P^{(n,m)}_{-k-1,-i-m+1}$.

\ITEM
Therefore%
%\red{
\\ \qquad
$\theta\circ r^{(m,n)}_i = s^{(n,m)}_{-i-1}\circ\theta$, \quad
$\theta\circ\omega^{(m,n)} = (\varpi^{(n,m)})^{-1}\circ\theta$,
\\ \qquad
$\theta\circ s^{(m,n)}_k = r^{(n,m)}_{-k-1}\circ\theta$, \quad
$\theta\circ\varpi^{(m,n)} = (\omega^{(n,m)})^{-1}\circ\theta$.
%}

%%%%%%%%%%%%%%%%%%%%%%%%%%%%%%%%%%%%%%%%%%%%%%%%%%%%%%%%%%%%%%%%%%%%%%%%%%%%%%

\foilhead[-10mm]{\red{Lax representations of the actions of $r_i$ and $s_k$}}

\blue{$X$-operators:}
\(
 X_{ik} = X_{ik}(z) :=
 \begin{bmatrix}
  x_{ik}  & 1         &        & \\
          & x_{i+1,k} & \DDOTS & \\
          &           & \DDOTS & 1 \\
  r^{-k}z &           &        & x_{i+m-1,k} \\
 \end{bmatrix}.
\)

\blue{\bf(1)}
The action of \red{$r_i$} on $\{x_{1k},\ldots,x_{mk}\}$ 
is uniquely characterized by
\\[2mm] \quad\quad\quad\quad\quad\quad\quad\quad
$r_i(X_{1k}) = G^{(i)}_k X_{1k} \bigl(G^{(i)}_{k+1}\bigr)^{-1}$.
\\[2mm]
$G^{(i)}_k := 1 + s^{-1}\dfrac{c_{ik} - c_{i+1,k+1}}{P_{ik}} E_{i+1,i}$ 
\quad ($c_{ik} = x_{ik}x_{i+1,k}\cdots x_{i+m-1,k}$),\\
$G^{(0)}_k := 1 + r^{k-1} z^{-1} s^{-1}\dfrac{c_{mk} - c_{m+1,k+1}}{P_{mk}} E_{1m}$.
\ ($E_{ij}$'s are matrix units.)

\blue{\bf(2)}
The action of \red{$s_k$} is uniquely characterized by
\\ \quad %
\(
 s_k(X_{ik}X_{i,k+1}) = X_{ik}X_{i,k+1}, \ %
 s_k(X_{il}) = X_{il} \ (l\not\equiv k\pmod n),
\)
\\ \quad %
\(
 s_k: d_{ik} \leftrightarrow d_{i+1,k+1}
 \quad (d_{ik} = x_{i+m-1,k}\cdots x_{i+1,k}x_{ik}).
\)

%%%%%%%%%%%%%%%%%%%%%%%%%%%%%%%%%%%%%%%%%%%%%%%%%%%%%%%%%%%%%%%%%%%%%%%%%%%%%%

\foilhead[-10mm]{\red{Quantum $q$-difference isomonodromic systems}}

\blue{Monodromy matrix:}\quad
$\X_{ik}(z):=X_{ik}(z)X_{i,k+1}(z)\cdots X_{i,k+n-1}(z)$.
\\[1mm]
\blue{Matrix $q$-difference shift operator \red{(shift parameter = $s$)}:}\\
$T_{z,s}v(s):=\diag(s^{-1},s^{-2},\ldots,s^{-m})v(s^mz)$
\ \text{($v(z)$ is $m$-vector valued)}.
\\[2mm]
\blue{Linear $q$-difference equation:}\quad
$T_{z,s}v(z) = \X_{11}(z)v(z)$.
\\[2mm]
\blue{Connection matrix preserving transformations:}
\\
(1) $s_k(\X_{11}(z))=\X_{11}(z)$ for $k=1,2,\ldots,n-1$.
\\
(2) \(
 \varpi(\X_{11}(z))
 = X_{11}^{-1}\X_{11}(z)X_{1,n+1}
 = T_{z,s}X_{1,n+1}^{-1}T_{z,s}^{-1}\X_{11}(z)X_{1,n+1}
\). 
%\\
%(3) \(
% r_i(\X_{11}(z)) 
% = G^{(i)}_1\X_{11}(z)G^{(i)}_{n+1}
% = T_{z,s}G^{(i)}_{n+1}T_{z,s}^{-1}\X_{11}(z)G^{(i)}_{n+1}
%\).

\ITEM
$U_k = s_{k-1}\cdots s_2s_1 \varpi s_{n-1} \cdots s_{k+1}s_k$. 
\\[2mm]
\dummyITEM
The action of $\bra U_1,U_2,\ldots,U_n\ket\isom\Z^n$\\
\quad $\longrightarrow$
Quantum $q$-difference isomonodromic dynamical system \\
\quad \phantom{$\longrightarrow$}
with $n$ time variables

\ITEM
The action of $\Wtilde{m}$
$\longrightarrow$ Symmetry of the dynamical system

%The action of $\bra U_1,U_2,\ldots,U_n\ket\isom\Z^n$ can be regarded 
%as a quantum $q$-difference isomonodromic dynamical system with
%$n$ time variables. 
%\\
%Then the action of $\Wtilde{m}$ is symmetry of this dynamical system.

%%%%%%%%%%%%%%%%%%%%%%%%%%%%%%%%%%%%%%%%%%%%%%%%%%%%%%%%%%%%%%%%%%%%%%%%%%%%%%

\newpage

\begin{claim}[Example \red{($(m,n)=(3,2)$)}] 
 \ITEM $x_{i+3,k}=rx_{ik}$, $x_{i,k+2}=sx_{ik}$.
 \\
 \ITEM
 $x_{11}x_{11}=      x_{11}x_{11}$, \quad
 $x_{21}x_{11}=q^{-2}x_{11}x_{21}$, \quad $\!\!\!\!\!\!$
 $x_{31}x_{11}=q^2   x_{11}x_{31}$, \quad
 \\
 \dummyITEM
 $x_{12}x_{11}=      x_{11}x_{12}$, \quad
 $x_{22}x_{11}=q^2   x_{11}x_{22}$, \quad
 $x_{32}x_{11}=q^{-2}x_{11}x_{32}$. 
 \\[3mm]
 \ITEM
 $P_{ik}=x_{i+1,k+1}+x_{ik}$,
 \\
 \dummyITEM
 $Q_{ik}=x_{i+2,k+1}x_{i+1,k+1}+x_{i+2,k+1}x_{ik}+x_{i+1,k}x_{ik}$.
 \\[3mm]
 \ITEM
 \(
  r_1(x_{11})
%  = s P_{11} x_{21} P_{12}^{-1}
  = s (x_{22}+x_{11}) x_{21} (x_{13}+x_{12})^{-1}
 \),
 \\
 \dummyITEM
 \(
  r_1(x_{21})
%  = s^{-1} P_{11}^{-1} x_{11} P_{12}
  = s^{-1} (x_{22}+x_{11})^{-1} x_{21} (x_{13}+x_{12})
 \),
 \\
 \dummyITEM
 $\omega(x_{ik}) = x_{i+1,k}$.
 \\[3mm]
 \ITEM
 \begin{small}%
 \(
  s_1(x_{11})
%  = r Q_{21}^{-1} x_{12} Q_{11}
  = r
  (x_{42}x_{32}+x_{42}x_{21}+x_{31}x_{21})^{-1}
  x_{12}
  (x_{32}x_{22}+x_{32}x_{11}+x_{21}x_{11})
 \),
 \\
 \dummyITEM
 \(
  s_1(x_{12})
%  = r^{-1} Q_{21} x_{12} Q_{11}^{-1}
  = r^{-1}
  (x_{42}x_{32}+x_{42}x_{21}+x_{31}x_{21})
  x_{11}
  (x_{32}x_{22}+x_{32}x_{11}+x_{21}x_{11})^{-1}
 \),
 \end{small}
 \\
 \dummyITEM
 $\varpi(x_{ik}) = x_{i,k+1}$.
 \qquad 
 ($U_1=\varpi r_1$, $U_2=r_1\varpi$)
 \\
 \dummyITEM
 \begin{small}%
 \(
  U_1(x_{11})
%  =\varpi r_1(x_{11}) \\ \quad
  = r
  (x_{43}x_{33}+x_{43}x_{22}+x_{32}x_{22})^{-1}
  x_{13}
  (x_{33}x_{23}+x_{33}x_{12}+x_{22}x_{12})
 \).
 \end{small}
 \\[4mm]
 \ITEM
 $U_1$ generates \red{quantum \qP{IV}} ($q$-difference Panlev\'e IV system). 
 \\
 \dummyITEM
 The action of $\WTilde{2}$ is symmetry of quantum \qP{IV}.
\end{claim}

%%%%%%%%%%%%%%%%%%%%%%%%%%%%%%%%%%%%%%%%%%%%%%%%%%%%%%%%%%%%%%%%%%%%%%%%%%%%%%

\newpage

\foilhead[-10mm]
{\red{Action of $\Wtilde{m}\times\Wtilde{n}$ on $\K_{m,n}$ as alg.\ autom.}}

\blue{\bf Theorem.}
For any \red{mutually prime} integers $m,n\geqq 2$, 
the action of $\Wtilde{m}\times\Wtilde{n}$ 
on $\K_{m,n}=Q(\A_{m,n})$ as algebra automorphisms is constructed.
%
This is a \red{quantization of the KNY action} of \\
$\Wtilde{m}\times\Wtilde{n}$ 
on $\C(x_{ik}|1\leqq i\leqq m,1\leqq k\leqq n)$.

\blue{\bf Easy Part.}
Lax representations $\implies$ braid relations of $r_i$ and $s_k$. 

\blue{\bf Difficult Part.}
To show that \\
$r_i$ and $s_k$ act on $\K_{m,n}=Q(\A_{m,n})$
\red{as algebra automorphisms}.

\blue{\bf Sketch of proof.}
Let $\varphi_i$ be appropriately \red{corrected} Chevalley generators
in $\B_{m,n}$ and put $\rho_i := \varphi_i^{\ac_i}\rtilde_i$.
Then $\Ad(\rho_i)x_{jl}=\rho_i x_{jl} \rho_i^{-1}=r_i(x_{jl})$.
Therefore $r_i$ acts on $\K_{m,n}$ as algebra automorphisms.
The duality leads to that $s_k$ also acts on $\K_{m,n}$ 
as algebra automorphisms.

%%%%%%%%%%%%%%%%%%%%%%%%%%%%%%%%%%%%%%%%%%%%%%%%%%%%%%%%%%%%%%%%%%%%%%%%%%%%%%

\foilhead[-10mm]
{\red{Chevalley generators $F_i$}}

\blue{Monodromy matrix:}
\quad
$\L(z):=L_1(z)L_2(z)\cdots L_n(z)$. 
\\
\ ($\L(z)$ is the product of the $L$-operators of the minimal representations.)
\\ \qquad
{\small
\(
 \L(z) =
 \begin{bmatrix}
  A_1 & B_1 & \DDOTS  & \DDOTS  \\
      & A_2 & \DDOTS  & \DDOTS  \\
      &     & \DDOTS  & B_{m-1} \\
  \bigzerol & &       & A_m     \\
 \end{bmatrix}
 + z
 \begin{bmatrix}
  \DDOTS & \DDOTS & \DDOTS & \DDOTS \\
  \DDOTS & \DDOTS & \DDOTS & \DDOTS \\
  \DDOTS & \DDOTS & \DDOTS & \DDOTS \\
  B_m    & \DDOTS & \DDOTS & \DDOTS \\
 \end{bmatrix}
 + \cdots
\)
}

\ITEM
$R(z/w)\L(z)^1\L(w)^2=\L(w)^2\L(z)^1R(z/w)$ \\
$\implies$ $F_i:=A_i^{-1}B_i$ satisfy the $q$-Serre relations.

\ITEM
$R_i := F_i^{\ac_i}\rtilde_i$ generate the Weyl group action 
on the skew field generated by $A_i$, $B_i$, 
and parameters $a_\lambda = q^{\lambda}$ 
($\lambda\in \h$).

\ITEM
\red{But} the action of $\Ad(R_i)$ does \red{not} preserve the skew field 
generated by $x_{ik}=a_{ik}(b_{ik}b_{i+1,k+1}\cdots b_{i+\minv m-1})^{-1}$,
$\ball$, and parameters $a_\lambda = q^{\lambda}$.

%%%%%%%%%%%%%%%%%%%%%%%%%%%%%%%%%%%%%%%%%%%%%%%%%%%%%%%%%%%%%%%%%%%%%%%%%%%%%%

\foilhead[-10mm]
{\red{Correction factors for $F_i$}}

\ITEM
$\K_{m,n}=Q(\A_{m,n}) \subset Q(\B_{m,n})$.

\ITEM
$x\simeq y$ $\iff$ 
$\exists\, c\in (\text{the center of}\ Q(\B_{m,n}))^\times \ \text{s.t.}\ cx=y$.

\ITEM
$\ninv :=$ mod-$m$ inverse of $n$ \ %
($\ninv n \equiv 1 \pmod m$, \ $\ninv=1,2,\ldots,m-1$).

\ITEM
$v_{ik} := b_{ik}b_{i+1,k+1}\cdots b_{i+\ninv n-1,k+\ninv n-1}$
\quad ($v_{1k}$ are correction factors).
\\[2mm]
\ (cf. $x_{ik}=a_{ik}(b_{ik}b_{i+1,k+1}\cdots b_{i+\minv m-1})^{-1}$,
\ $\minv=$ mod-$n$ inverse of $m$)

\ITEM
\(
 c_{i1}^{-1}P_{i1} 
 \simeq v_{i1}^{-1} F_i = v_{i1}^{-1}A_i^{-1}B_i
\)
\quad (motivation to find $v_{ik}$).

\ITEM
$\varphi_i := v_{i1} F_i = v_{i1}A_i^{-1}B_i \simeq v_{i1}^2c_{i1}^{-1}P_{i1}$
\quad \red{(corrected $F_i$)}.

\ITEM
Using $\varphi_i$ instead of $F_i$, we can construct the action of 
the affine Weyl group $\Waffine{m}$ on $\K_{m,n}=Q(A_{m,n})$
as algebra automorphisms.

%%%%%%%%%%%%%%%%%%%%%%%%%%%%%%%%%%%%%%%%%%%%%%%%%%%%%%%%%%%%%%%%%%%%%%%%%%%%%%

\foilhead[-10mm]
{\red{Generators of the $\Waffine{m}$-action on $\K_{m,n}=Q(\A_{m,n})$}}

\ITEM
$\cH_m:=\C(q,r',s')[q^{\pm2\ec_i}]_{i=1}^m$, \quad
$\ec_i := E_{ii}\in\h$,\quad  $\ac_i := \ec_i - \ec_{i+1}$.
\\[2mm]
\ITEM
$\A_{m,n}\isom (\A_{m,n}\otimes\cH_m)/I$ 
\qquad ($\otimes=\otimes_{\C(q,r',s')}$), 
\\
where $I:=$ the two-sided ideal generated 
by $c_{ii}\otimes 1 - 1\otimes q^{-2\ec_i}$.
\\[2mm]
\ITEM
$\rtilde_i\ec_i\rtilde_i^{-1}=\ec_{i+1}$, \ %
$\rtilde_i\ec_{i+1}\rtilde_i^{-1}=\ec_i$, \ %
$\rtilde_i\ec_j\rtilde_i^{-1}=\ec_j$ ($j\ne i,i+1$).
\\[2mm]
\ITEM
$\rho_i := \varphi_i^{\ac_i}\rtilde_i$.
\quad \blue{(generators of the $\Waffine{m}$-action on $\K_{m,n}$)}
\\[2mm]
\ITEM
$\Ad(\rho_i)$ generate the action of $\Waffine{m}$ on $Q(\A_{m,n}\otimes \cH_m)$. 
\\[2mm]
\ITEM
\red{\bf The actions of $\Ad(\rho_i)$'s on $Q(\A_{m,n}\otimes \cH_m)$ induce \\
the actions of $r_i\in\Waffine{m}$ on $\K_{m,n}=Q(\A_{m,n})$:}
\\[2mm] \qquad
\(
 \Ad(\rho_i)x_{il}
 = r_i(x_{il})
 = s P_{il} x_{i+1,l} P_{i,l+1}^{-1},
\)
\\[2mm] \qquad
\(
 \Ad(\rho_i)x_{i+1,l}
 = r_i(x_{i+1,l})
 = s^{-1} P_{il}^{-1} x_{il} P_{i,l+1},
\)
\\[2mm] \qquad
\(
 \Ad(\rho_i)x_{jl} = r_i(x_{jl})
 = x_{jl} \quad (j\not\equiv i,i+1\pmod m).
\)

%%%%%%%%%%%%%%%%%%%%%%%%%%%%%%%%%%%%%%%%%%%%%%%%%%%%%%%%%%%%%%%%%%%%%%%%%%%%%%

\foilhead[-10mm]{Summary of Results}

\blue{\bf\S2.}
(for any symmetrizable GCM $A=[a_{ij}]$)
\\[2mm]
\ITEM
$\Ad$-action of 
\red{complex powers of Chevalley generators $f_i$ in $U_q(\g)$}
\\[2mm]
$\implies$
the action of the Weyl group on $Q(U_q(\n)\otimes U_q(\h))$
\\ \phantom{$\implies$}
(quantum $q$-difference version of the NY \blue{\tt math.QA/0012028} action)
\\[2mm]
$\implies$
Reconstruction of the Hasegawa \blue{\tt math.QA/0703036} action
\\[3mm]
\blue{\bf\S3.}
(for any mutually prime integers $m,n\geqq 2$)
\\[2mm]
\ITEM
$\B_{m,n} :=$
the minimal representation of 
$U_q(\b)^{\otimes n}\subset U_q(\affgl_m)^{\otimes n}$.
\\[2mm]
\ITEM
$\K_{m,n} := Q($\red{the gauge invariant subalgebra $\A_{m,n}$ of $\B_{m,n}$}$)$
\\
$\implies$
$\K_{m,n} =$ Quantization of $\C(x_{ik}|1\leqq i\leqq m,1\leqq k\leqq n)$.
\\[2mm]
\ITEM
\red{Complex powers of the {\bf corrected} Chevalley generators in $\B_{m,n}$}
\\
$\implies$
$\Wtilde{m}$-action on $\K_{m,n}$
\\
$\implies$
$\Wtilde{m}\times\Wtilde{n}$-action on $\K_{m,n}$
\quad(by the $m\leftrightarrow n$ duality)

%%%%%%%%%%%%%%%%%%%%%%%%%%%%%%%%%%%%%%%%%%%%%%%%%%%%%%%%%%%%%%%%%%%%%%%%%%%%%%

\foilhead[-10mm]{Other Problems}

\blue{\bf Problem.}
Construct commuting Hamiltonians in $U_q(\n)\otimes U_q(\h)$ 
with Weyl group symmetry.
\\[2mm]
\blue{\bf Hint.}
\red{Commuting transfer matrices for ``$AL^1BL^2=CL^2DL^1$'' algebras}. 
($F=q^{-\sum H_i\otimes H^i}$, 
$A=P(F)^{-1}RF$, 
$B=F$, 
$C=P(F)$, 
$D=R$%, 
%$L=P(F)R$
)
\\[4mm]
\blue{\bf Problem.}
Construct commuting Hamiltonians in $\A_{m,n}$ \\
with $\Wtilde{m}\times\Wtilde{n}$ symmetry.
\\[2mm]
\blue{\bf Classical Case.}
\red{$\det(\X^{(m,n)}_{11}(z)-(-1)^nw)=\det(\X^{(n,m)}_{11}(w)-(-1)^mz)$} 
generates the invariants of birational $\Wtilde{m}\times\Wtilde{n}$ action.
\\[4mm]
\blue{\bf Problem.}
Construct solutions of quantum ($q$-)isomonodromic systems.
\\[2mm]
\blue{\bf Conjecture.}
Schr\"odinger equation of any quantum ($q$-)isomonodromic sys\-tem 
has (non-confluent or confluent) ($q$-)\red{hypergeometric} solutions.

%%%%%%%%%%%%%%%%%%%%%%%%%%%%%%%%%%%%%%%%%%%%%%%%%%%%%%%%%%%%%%%%%%%%%%%%%%%%%%
\end{document}
%%%%%%%%%%%%%%%%%%%%%%%%%%%%%%%%%%%%%%%%%%%%%%%%%%%%%%%%%%%%%%%%%%%%%%%%%%%%%%
